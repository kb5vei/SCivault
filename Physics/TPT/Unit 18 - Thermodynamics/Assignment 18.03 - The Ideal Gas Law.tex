\documentclass[letterpaper, 12pt]{article}
\usepackage[top=.5in,bottom=.5in,left=.75in,right=.75in,headheight=30pt, % as per the warning by fancyhdr
includehead,includefoot,
heightrounded, % to avoid spurious underfull messages
]{geometry}
\addtolength{\topmargin}{-.25in}
\usepackage{fancyhdr}
\pagestyle{fancy}
\usepackage{graphicx}
\usepackage{lastpage}
\usepackage{multicol}
\usepackage{qrcode}

\begin{document}
\fancyhead[l]{	\includegraphics[height=0.5in]{../Logo/sp.png} Name:}
\fancyhead[r]{Due Date: \hspace{ 1in}}
\fancyfoot[c]{\thepage\ of \pageref{LastPage}}
\fancyfoot[r]{Assignment 18.03}	


\begin{center} Assignment 18.01: The Ideal Gas Law
\end{center}





\begin{enumerate}
	\item What is the Ideal Gas Law, and what does each variable in the equation represent?
	\vspace{0.5in}
	
	\item Explain how the pressure of a gas changes if the volume of the container is reduced while the temperature is held constant.
	\vspace{0.65in}
	
	\item If the number of moles of a gas is doubled while keeping the temperature and volume constant, how does the pressure change?
	\vspace{0.65in}
	
	\item Why does increasing the temperature of a gas at constant volume increase its pressure?
	\vspace{0.65in}
	
	\item Under what conditions is the Ideal Gas Law most accurate, and why does it break down under extreme conditions?
	\vspace{0.65in}
	
	\item How would the Ideal Gas Law apply to a gas in a container with a variable volume, such as a balloon?
	\vspace{0.65in}
	
	\item Calculate the pressure of 2.0 moles of an ideal gas confined in a 10.0 L container at a temperature of 300 K. 
	\vspace{0.65in}
	
	\item Determine the temperature of 3.0 moles of an ideal gas that exerts a pressure of 5.0 atm in a 15.0 L container. 
	\vspace{0.65in}
	
	\item What volume does 1.5 moles of an ideal gas occupy at a pressure of 2.0 atm and a temperature of 350 K? 
	\vspace{0.65in}
	
	\item A gas at 1.0 atm and 273 K occupies a volume of 22.4 L. How many moles of gas are present? 
	\vspace{0.65in}
	
	\item Calculate the pressure of an ideal gas containing \(2.5 \times 10^{23}\) particles in a 0.1 m³ container at a temperature of 300 K. 
	\vspace{0.65in}
	
	\item Find the temperature of a gas if it contains \(3.0 \times 10^{24}\) particles, occupies a volume of 0.2 m³, and exerts a pressure of \(1.5 \times 10^5\) Pa. 
	\vspace{0.65in}
	
	\item What is the volume of a gas containing \(1.0 \times 10^{25}\) particles at a pressure of \(2.0 \times 10^5\) Pa and a temperature of 400 K? (Use 
	\vspace{0.65in}
	
	\item Determine the number of particles in a gas at a temperature of 350 K, occupying a volume of 0.05 m³, and exerting a pressure of \(2.0 \times 10^5\) Pa.
	\vspace{0.65in}
	
	\item If the volume of a gas is tripled while keeping the temperature constant, how does the pressure change according to the Ideal Gas Law? Explain using the physics version of the equation.
	\vspace{0.65in}
\end{enumerate}

 



\end{document}
