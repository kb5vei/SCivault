\documentclass[letterpaper, 12pt]{article}
\usepackage[top=2cm,bottom=1cm,left=0.75in,right=0.75in,headheight=17pt, % as per the warning by fancyhdr
includehead,includefoot,
heightrounded, % to avoid spurious underfull messages
]{geometry}
\addtolength{\topmargin}{-.25in}
\usepackage{fancyhdr}
\pagestyle{fancy}
\usepackage{graphicx}
\usepackage{lastpage}
\usepackage{gensymb}

\begin{document}
\fancyhead[l]{	\includegraphics[height=1.2cm]{../Logo/sp.png} Name:}
\fancyhead[r]{REFERENCE MATERIAL}
\cfoot{\thepage\ of \pageref{LastPage}}
	


\begin{center}Things to Memorize: Thermodynamics
\end{center}

\subsection*{Modes of Heat Transfer}
\begin{itemize}
	\item There are three modes of heat transfer: 
	\begin{itemize}
		\item \textbf{Conduction} - When objects are in physical contact.  Metals in particular are quite good at this.
		\item \textbf{Convection} - Is due to the flow of a fluid.  Liquids, gasses, and plasmas are all fluids.
		\item \textbf{Thermal Radiation} is heat transfer due to electromagnetic waves (usually infrared, but can be other types as well.)  This is the only mode that does not require a \textit{medium}. 
	\end{itemize}
\end{itemize}

\subsection*{Kinetic Theory}
\begin{itemize}
	\item The temperature of a gas is related to the average kinetic energy of the atoms or molecules in the gas.  
	\begin{itemize}
		\item As the speed of molecules increases, the temperature increases.
		\item When molecules collide with their surroundings (such as a container), some of the energy is transferred out of the gas, causing the temperature to decrease. 
	\end{itemize}
	
	
\end{itemize}

	
\subsection*{PV Diagrams}
\begin{itemize}
			\item The area under a PV diagram is work.
			\item Be extra careful with units and powers of ten on this type of graph.
\end{itemize}

	
\subsection*{{Heat and Internal Energy}}

	\begin{itemize}
	\item Heat (Q) must be transferred from one object to the other.
	\item Internal Energy (or internal thermal energy) is the total kinetic energy of all the molecules in a gas.  It affects temperature.
	
\end{itemize}

\subsection*{Laws of Thermodynamics}
\begin{itemize}
	\item \textbf{0\textsuperscript{th} Law}: 
	\item  \textbf{1\textsuperscript{st} Law}: 
	\item  \textbf{2\textsuperscript{nd} Law}: 
	
\end{itemize}



\subsection*{Gas Laws}
\begin{itemize}
	\item The Combined Gas Law: $\frac{P_1 V_1}{T_1} = \frac{P_2 V_2}{T_2}$ 
	\item The Ideal Gas Law
\end{itemize}
 



\end{document}
