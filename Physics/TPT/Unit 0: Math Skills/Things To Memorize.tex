\documentclass[letterpaper, 12pt]{article}
\usepackage[top=.5in,bottom=.5in,left=.75in,right=.75in,headheight=30pt, % as per the warning by fancyhdr
includehead,includefoot,
heightrounded, % to avoid spurious underfull messages
]{geometry}
\addtolength{\topmargin}{-.25in}
\usepackage{fancyhdr}
\usepackage{cancel}
\usepackage{gensymb}

\pagestyle{fancy}
\usepackage{graphicx}
\usepackage{lastpage}
\usepackage{multicol}
\newcommand{\assnum}{Unit 0: Math Skills}
\newcommand{\assname}{Things To Memorize}

\begin{document}
\fancyhead[l]{	\includegraphics[height=0.5in]{../Logo/sp.png} Name:}
\fancyhead[r]{Due Date: \hspace{ 1in}}
\fancyfoot[c]{\thepage\ of \pageref{LastPage}}
\fancyfoot[r]{\assnum}	


\begin{center} \assnum{} - \assname{}
\end{center}





\begin{enumerate}
	\item Scientific Notation:
		\begin{itemize}
			\item Scientific Notation always has three parts: the \textit{coefficient}, the \textit{base}, and the \textit{exponent}:
			\begin{center}
				Coefficient $\rightarrow 6.022 \times 10^{23 \leftarrow}$ \textsuperscript{Exponent}
				
				\hspace{.7in} $\uparrow$
				
				\hspace{.7in} Base
			\end{center}
		\item In scientific notation the base is always 10.
		\item A negative in front of the coefficient means the whole number is negative.
		\item A negative exponent means the number is very small (close to zero).	
		\item When comparing numbers in scientific notation, look at (in order): 
		\begin{enumerate}
			\item Negatives in front of the coefficient.
			\item Exponents
			\item Coefficients
		\end{enumerate}
	
	\item To multiply, multiply coeffients, then ADD exponents.
	\item To divide, divide coefficients, then SUBTRACT exponents.
	\item To raise to a power, raise the coefficient to the power, then MULTIPLY exponents.
	\item To enter scientific notation on the calculator use the ``EE" key. $6.022 \times 10^{23}$ is entered as 6.022\scriptsize E\normalsize23.  Calculator notation should never be handwritten. 
	\end{itemize}
	
	\vspace{0.25in}
	\item Algebra:
	\begin{itemize}
		\item To solve for something in the top of a fraction, multiply by the bottom.
		\begin{center}
			$A = \frac {B}{C} \hspace{0.2in} \longrightarrow \hspace{0.2in} A \times C = \frac {B}{\cancel{C}} \times \cancel{C} \hspace{0.2in} \longrightarrow  \hspace{0.2in} A C = B	$
		\end{center}
		\item To solve for something in the bottom of a fraction, switch the bottom with the other side:
		\begin{center}
			$A = \frac {B}{C} \rightarrow C = \frac {B}{A}	$
		\end{center}
		\end{itemize}
	\vspace{2in} 
	\item Trigonometry
	\begin{itemize}
		\item Remember: SOH-CAH-TOA.  It means:
		\begin{center}
			$\sin(\theta) = \frac{opp}{hyp} \hspace{0.5in} \cos(\theta) = \frac{adj}{hyp} \hspace{0.5in} \tan(\theta) = \frac{opp}{adj}	$	
		\end{center}
	\item Hypotenuse is always the longest side.
	\item Cut the angle of interest in half and draw a line across the triangle to find the opposite side.
	\item The adjacent side and the hypotenuse create the angle.
	\item To find a side, use regular functions (sin, cos, tan)
	\item To find an angle use inverse functions (called $\arcsin$, $\arccos$ $\arctan$ or $\sin^{-1}$ $\cos^{-1}$ $\tan^{-1}$)
	\item All trigonometric functions need an argument - They never go anywhere without $(\theta)$.	
	\end{itemize}
	
	\item Arc Length
	\begin{itemize}
		\item $2\pi $ radians = $360 \degree $  = 1 full rotation
		\item 1 radian is the angle where the radius = the arc length $\approx 57.2958\degree$
		\item To use the arc-length formula, all angles must be measured in radians.
	\end{itemize}
\end{enumerate}
 



\end{document}
