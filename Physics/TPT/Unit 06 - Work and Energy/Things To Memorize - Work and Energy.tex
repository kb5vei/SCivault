\documentclass[letterpaper, 12pt]{article}
\usepackage[top=2cm,bottom=1cm,left=0.75in,right=0.75in,headheight=17pt, % as per the warning by fancyhdr
includehead,includefoot,
heightrounded, % to avoid spurious underfull messages
]{geometry}
\addtolength{\topmargin}{-.25in}
\usepackage{fancyhdr}
\pagestyle{fancy}
\usepackage{graphicx}
\usepackage{lastpage}
\usepackage{gensymb}

\begin{document}
\fancyhead[l]{	\includegraphics[height=1.2cm]{../Logo/sp.png} Name:}
\fancyhead[r]{REFERENCE MATERIAL}
\cfoot{\thepage\ of \pageref{LastPage}}
	


\begin{center}Things to Memorize: Work and Energy
\end{center}
\subsection*{The Dot Product}
\begin{itemize}
	\item A \textbf{dot product} is a way of multiplying vectors that results in a scalar. 
	\item To calculate the dot product, only the components of the vectors that are in the same direction are multiplied.
	\item This is equivalent to multiplying the magnitude of the first vector times the magnitude of the second vector times the cosine of the angle between them: $ \vec{A} \bullet \vec{B} = A \cdot B \cdot cos(\theta) $
\end{itemize}
\subsection*{Work}
\begin{itemize}
	\item Work is a scalar.  It is symbolized by the letter $ W $.  
	\item Work is defined as the \textbf{dot product} of Force and distance: $W = \vec{F} \bullet \vec{d} = F \cdot d \cdot cos(\theta) $
	\item The units for work are $ \frac{kg \cdot m^2}{s^2}$.  These are usually abbreviated as $J$ (joules). 
	\item Work causes an object's \textbf{energy} to change. 


	
\end{itemize}

\subsection*{Energy}
\begin{itemize}
	\item Energy is the ability to do work.  The total energy of an object is often symbolized by an $E$. 
	\item Energy is a scalar. 
	\item The units for energy are $ \frac{kg \cdot m^2}{s^2}$.  These are usually abbreviated as $J$ (joules). 
	\item Energy comes in many types.  This unit focuses on the following types of energy:
	\begin{itemize}
		\item \textbf{Gravitational Potential Energy} - is energy that is stored due to the location of an object in a gravitational field. 
		\begin{itemize}
			\item Gravitational potential energy is often abbreviated GPE.  The Symbol for gravitational potential energy is $U_g$.
			\item If you are in a uniform gravitational field, (very close to the surface of a planet), gravitational potential energy depends on the \textbf{mass of the object}, the \textbf{gravitational field} of the planet, and the \textbf{height} above the surface.  
		\end{itemize}
		
		\item \textbf{Kinetic Energy} - is energy of motion.
		\begin{itemize}
			\item Kinetic Energy is often abbreviated KE.  The symbol for kinetic energy is $K$.  
			\item An object that is moving does not use up kinetic energy.  It simply has kinetic energy due to its motion (see Newton's First Law).
		\end{itemize}
	\end{itemize}
	

	
	
\end{itemize}

	
\subsection*{Conservation of Energy}
\begin{itemize}
	\item The \textbf{Law of Conservation of Energy} states that energy can neither be created, nor destroyed.\footnote{Einstein was able to prove that this law is not entirely accurate in all situations.  It will be revised when we study nuclear reactions.} 
	\begin{itemize}
		\item This means that whatever total energy a system has at the beginning must equal to the total energy the system has at the end. 
		\item If work causes an object's energy to change, it must be accounted for as well.  
		\item A basic equation for the law of conservation of energy: $E_i + W = E_f $
	\end{itemize}
	\item To apply the law of conservation of energy to a system:
		\begin{enumerate}
			\item Draw the system in its \textit{before} and \textit{after} states.  
			\item Write an energy term for each object that is part of the system, both before and after.
			\item Determine if any work flows into or out of the system.
			\item Plug in the formulas for work and energy to each term.
			\item Manipulate the equation to solve for the variable you want.
			\item Substitute numbers into the equation and calculate the final answer with units. 
		\end{enumerate}
\end{itemize}



 



\end{document}
