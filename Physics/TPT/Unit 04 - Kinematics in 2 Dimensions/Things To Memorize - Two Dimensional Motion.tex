\documentclass[letterpaper, 12pt]{article}
\usepackage[top=2cm,bottom=1cm,left=0.75in,right=0.75in,headheight=17pt, % as per the warning by fancyhdr
includehead,includefoot,
heightrounded, % to avoid spurious underfull messages
]{geometry}
\addtolength{\topmargin}{-.25in}
\usepackage{fancyhdr}
\pagestyle{fancy}
\usepackage{graphicx}
\usepackage{lastpage}
\usepackage{gensymb}

\begin{document}
\fancyhead[l]{	\includegraphics[height=1.2cm]{../Logo/sp.png} Name:}
\fancyhead[r]{REFERENCE MATERIAL}
\cfoot{\thepage\ of \pageref{LastPage}}
	


\begin{center}Things to Memorize: Motion in Two Dimensions
\end{center}

\subsection*{General Ideas:}
\begin{itemize}
	\item Motion in one dimension does not effect the motion in the other dimension.
	\begin{itemize}
		\item Thus, kinematic equations can only be applied to one dimension at a time. 
		\item The only variable that can be applied to each dimension is time.
		\end{itemize}
			
	\item Each Dimension (X, Y) has its own set of kinematic variables. Use a subscript to distinguish them, ie: $v_{fx}$ and $v_{fy}$.
	

	\item Distances, Velocities, and Accelerations can be \textbf{combined as vectors} by using the Pythagorean Theorem.
	\item Distances, Velocities, and Accelerations can be \textbf{decomposed} (broken down) into component vectors by using trigonometric equations. 
	
\end{itemize}

\subsection*{To Solve a Two-Dimensional Problem}

		\begin{enumerate}
	\item Draw a diagram.
	\item Define a positive X direction and a positive Y direction.  Label those directions clearly with arrows: $\longrightarrow X $ and $\uparrow Y$
	\begin{itemize}
		\item X and Y directions must be 90\degree to each other.
		\item X and Y don't have to be right and up.  Instead, choose directions that correspond with the motion described in the problem. 
		
	\end{itemize}
	\item Indicate in words what portion of motion your are considering, (like ``motion from launch to the peak of the flight.”)
	\item Fill out a chart, including signs and units, of the five kinematics variables for each direction.  Remember time is the only varible that can be used in any dimension.
	\begin{center}
		\begin{tabular} {| c | c | }
			\hline
			$d_x$ = \hspace{0.4in}    & $d_y $= \hspace{0.4in} \\
			\hline
			$v_{ix}$  = \hspace{0.4in}  &$v_{iy}$  = \hspace{0.4in} \\
			\hline
			$v_{fx}$  = \hspace{0.4in}  &$v_{fy}$  = \hspace{0.4in} \\
			\hline
			$a_{x}$  = \hspace{0.4in}  &$a_{y}$  = \hspace{0.4in} \\
			\hline 
		 	 \multicolumn{2}{|c|}{	$t$  = \hspace{0.4in} }  \\
			\hline
		\end{tabular} 
	\end{center}
	\item Pick an dimension (X, Y) that will allow you to use a kinematic equation that has only \textbf{ONE} unknown variable.  
		\begin{itemize}
			\item Remember: \textbf{Do not mix X and Y variables in the same equation!}
		\end{itemize}
	\item Manipulate the equation to isolate the unknown variable (if needed).
	\item Plug in the numbers.
	\item Write your answer with units. Add it to the table above.
	\item Continue to solve for variables until you have determined the variable you want. 
	
\end{enumerate}

 

\subsection*{Projectiles Launched Horizontally}
\begin{itemize}
	\item A projectile launched horizontally has an initial vertical velocity of zero: $v_{iy}$  = 0 m/s.  
	\item Since gravity is the only force acting on the object while it is in free-fall, $a_y = 9.8$ $m/s^2$ downward, and $a_x = 0$ $m/s^2$
	\item Be sure to double-check your positive and negative signs to make sure they correspond with the diagram you drew.
	
\end{itemize}

\subsection*{Projectiles Launched at an Angle}
	\begin{itemize}
	\item Use trigonometry to determine the initial X and Y velocities. 
	\item Since gravity is the only force acting on the object while it is in free-fall, $a_y = 9.8$ $m/s^2$ downward, and $a_x = 0$ $m/s^2$
	\item Be sure to double-check your positive and negative signs to make sure they correspond with the diagram you drew.
	\item Determine the time to the top of the path first.  At the top of a projectile's path, $v_{y}$ = 0 m/s.    
	\item Rarely, you will need to solve the equation $d = v_i t + \frac{1}{2}at^2$ for t when neither $v_i$ nor $a$ are zero. 
	\begin{itemize}
		\item This can be done by using the quadratic equation.
		\item  Alternately, the problem can be broken into a rising component and a falling component and the times can be added. 
	\end{itemize} 
	
	\end{itemize}


 
 



\end{document}
