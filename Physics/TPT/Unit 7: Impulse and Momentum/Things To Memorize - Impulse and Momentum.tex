\documentclass[letterpaper, 12pt]{article}
\usepackage[top=2cm,bottom=1cm,left=0.75in,right=0.75in,headheight=17pt, % as per the warning by fancyhdr
includehead,includefoot,
heightrounded, % to avoid spurious underfull messages
]{geometry}
\addtolength{\topmargin}{-.25in}
\usepackage{fancyhdr}
\pagestyle{fancy}
\usepackage{graphicx}
\usepackage{lastpage}
\usepackage{gensymb}

\begin{document}
\fancyhead[l]{	\includegraphics[height=1.2cm]{../Logo/sp.png} Name:}
\fancyhead[r]{REFERENCE MATERIAL}
\cfoot{\thepage\ of \pageref{LastPage}}
	


\begin{center}Things to Memorize: Impulse and Momentum
\end{center}

\subsection*{Momentum}
\begin{itemize}
	\item Momentum is a vector.  It is symbolized by the letter $\vec{p} $.  
	\item Momentum is defined as mass times velocity: $\vec{p} = m \cdot \vec v $
	\item The units for momentum are $ \frac{kg \cdot m}{s} $
	\item The direction of the momentum is always the same as the direction of the object's velocity. 
	\item Momentum is extremely useful for \textit{collisions} and \textit{explosions}.  
	
\end{itemize}

\subsection*{Impulse}
\begin{itemize}
	\item Impulse is a vector.  It is symbolized by the letter $\vec{J}$ or sometimes $\vec{I} $.  
	\item Impulse is defined as force times time: $\vec{J} = \vec{F} \cdot  t $
\item The units for impulse are $N \cdot s $, which reduce to $ \frac{kg \cdot m}{s} $.
\item The direction of the impulse is always the same as the direction of the force acting on the object. 
\item Impulse causes an object's momentum to change.
	
	
\end{itemize}

	
\subsection*{Conservation of Momentum}
\begin{itemize}
	\item The \textbf{Law of Conservation of Momentum} states that momentum can neither be created, nor destroyed. 
	\begin{itemize}
		\item This means that whatever total momentum a system has at the beginning must equal to the total momentum the system has at the end. 
		\item If Impulse causes an object's momentum to change, it must be accounted for as well.  
		\item A basic equation for the law of conservation of momentum: $\vec{p_i} + \vec{J} = \vec{p_f} $
	\end{itemize}
	\item To apply the law of conservation of momentum to a system:
		\begin{enumerate}
			\item Draw the system in its \textit{before} and \textit{after} states.  
			\item Write a momentum term for each object that is moving before and after.
			\item Determine if there is any impulse on the system.
			\item Plug in the formulas for impulse and momentum to each term.
			\item Manipulate the equation to solve for the variable you want.
			\item Substitute numbers into the equation and calculate the final answer with units. 
		\end{enumerate}
\end{itemize}

	
\subsection*{Impulse and Momentum in 2 Dimensions}
\begin{itemize}

	\item test
\end{itemize}

 



\end{document}
