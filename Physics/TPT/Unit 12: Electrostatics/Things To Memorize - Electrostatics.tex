\documentclass[letterpaper, 12pt]{article}
\usepackage[top=2cm,bottom=1cm,left=0.75in,right=0.75in,headheight=17pt, % as per the warning by fancyhdr
includehead,includefoot,
heightrounded, % to avoid spurious underfull messages
]{geometry}
\addtolength{\topmargin}{-.25in}
\usepackage{fancyhdr}
\pagestyle{fancy}
\usepackage{graphicx}
\usepackage{lastpage}
\usepackage{gensymb}

\begin{document}
\fancyhead[l]{	\includegraphics[height=1.2cm]{../Logo/sp.png} Name:}
\fancyhead[r]{REFERENCE MATERIAL}
\cfoot{\thepage\ of \pageref{LastPage}}
	


\begin{center}Things to Memorize: Electrostatics
\end{center}

\subsection*{Electrical Charge}
\begin{itemize}
	\item Electrical \textbf{Charge} is a scalar related to how many electrons are missing or in excess.  
	\item Electrical charge is symbolized by $q$ and measured in coulombs (C).
	\item Electrons have negative charge and protons have positive charge.  
	
\end{itemize}
\subsection*{Electric Force}
	\begin{itemize}
		\item \textbf{Electrostatic Force} is the force due to the interaction of two charged particles.
		\item Force is measured in newtons (N). 
			\begin{itemize}
				\item When using Coulomb's Law, \textbf{positive} forces are repulsive.
				\item When using Coulomb's Law, \textbf{negative} forces are attractive.
			\end{itemize}
		\item If there are more than two charges, the forces must be combined as vectors.
			\begin{enumerate}
				\item Draw the situation.  
				\item Calculate the force that each pair of charges puts on each other.
				\item Draw the forces on your diagram and assign appropriate signs.
				\item Combine all forces that are colinear. 
				\item Use trigonometry to combine forces in more than one dimension if needed.
			\end{enumerate}
	\end{itemize}
\subsection*{Electric Field}
\begin{itemize}
	\item \textbf{Electric Field} is a vector that shows how much force (and in what direction) a +1C charge would experience at any point.
	\item Units: N/C or V/m (they are the same)
	\item When Drawing Electric Fields:
		\begin{itemize}
			\item Arrows point \textbf{out} of positive charges and \textbf{into} negative charges. 
			\item When lines are \textbf{closer} the electric field is \textbf{stronger.}
			\item Electric field lines are always \textbf{perpendicular} to \textbf{equipotential} lines. 
		\end{itemize}
	\item The electric field inside a conductor is always 0 N/C.
	\item Electric Field is nearly uniform (constant) between two oppositely charged plates.
	\item The size of an electric field is infinite. 
	\item The electric field is zero at an infinite distance from the charged particle. 
	\end{itemize}


	
\subsection*{Electrostatic Potential Energy}
	\begin{itemize}
		\item \textbf{Electrostatic Potential Energy} is a scalar that shows how much energy two charges will have in proximity to each other.  
		\item The units for energy are Joules (J). 
		\item Electrostatic Potential Energy is zero when the two charges are separated by an infinite distance. 
	\end{itemize}

	
	


	
\subsection*{Electrostatic Potential}
\begin{itemize}
\item Electrostatic Potential measures the Energy per charge of a particle.
\item Electrostatic Potential is measured in Volts.
\item Electrostatic Potential is closely related to:
	\begin{itemize}
		\item Voltage
		\item Electromotive Force (EMF)
		\item Potential Difference
	\end{itemize}
\item \textbf{DO NOT} confuse with Electrostatic Potential \textbf{Energy}.  They are two related but different measurements. 
	
\end{itemize}

 



\end{document}
