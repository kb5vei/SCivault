\documentclass[10pt]{examdesign}
\usepackage{amsmath}
\usepackage{enumitem}
\usepackage{amsfonts}
\usepackage{pgfplots}
\usepackage{pifont}
\usepackage{graphicx}
\usepackage{fancyhdr}
\usepackage{cancel}
\usepackage{gensymb}
\SectionFont{\large\sffamily}
\Fullpages
\ContinuousNumbering


\DefineAnswerWrapper{}{}
\NumberOfVersions{1}
%\IncludeFromFile{foobar.tex}
\examname{Quiz: \\ Angular Velocity and Acceleration}
\class{ {\Large Physics}}

\def \namedata {Name: \hrulefill\\ 
	Date: \hrulefill \\
	Period: \hrulefill \\
	Peer Reviewer: \hrulefill \\
	Authentication Code: \hrulefill
	\\
	
	\begin{tabular}{| p{1cm} | p{1cm} | p{1 cm} | p{1cm} |}
		\hline
		+1 & 0 & -1 & $\Sigma$ 
		\\
		\hline
		& & & \vspace{.5cm}
		\\ \hline
		
	\end{tabular}
	\\
	\vspace{-.6in}
	
}




\begin{document}




\begin{multiplechoice} [title={Multiple Choice},
	rearrange=no]
	\textit{Choose the best answer to each question.}
	\begin{question}
		$v$ is related to $\omega$ in the same way that $a$ is related to:
		\choice {$\theta$}
		\choice {$\omega$}
		\choice [!]{$\alpha$}
		\choice {$\Delta$}
	\end{question}

	\begin{question}
	 	A large truck and a small car are both traveling at 15 m/s.  Which wheels have the greater angular velocity? 
		\choice {The large truck's wheels}
		\choice [!]{The small car's wheels}
		\choice {Both have the same angular velocity}
		\choice {It is impossible to tell}
	\end{question}

	\begin{question}
		During its spin cycle, a washing machine starts from a stop and slowly speeds up during the next 15 seconds until it is making 3 complete turns per second.  What is the angular acceleration of the washing machine? 
		\choice {{$0.2 \hspace{0.05in}  \frac{rad}{s^2}$}}
		\choice {{$0.796 \hspace{0.05in} \frac{rad}{s^2}$}}
		\choice [!]{$1.257 \hspace{0.05in} \frac{rad}{s^2}$}
		\choice {{$3 \hspace{0.05in}  \frac{rad}{s^2}$}}
	\end{question}



	\begin{question}
	Is it possible for the angular velocity of an object to increase while the linear velocity of the object decreases? 
		\choice{No, this can never happen.}
		\choice{Yes, if the object is sliding.}
		\choice{yes, if the object's radius is changing.}
		\choice[!]{Both B and C.}
	\end{question}
	
	\begin{question}
	What is the angle between the 10 and 12 on an analog clock?
		\choice{30$\degree$}
		\choice[!]{60$\degree$}
		\choice{90$\degree$}
		\choice{120$\degree$}
	\end{question}
	
	\begin{question}
		An object has an angular acceleration of $\alpha = 0 \hspace{0.05in} rad/s^2$.  Which of the following could describe the motion of the object. 
		\choice{The object is not rotating, and has an angular velocity of $0 \hspace{0.05in} rad/s$ at all times.  }
		\choice{the object is rotating with an angular velocity of $2.3 \hspace{0.05in}$ rad/s at all times.}
		\choice{The object has an angular displacement that is the same at all times.}
		\choice{A and B, but not C.}
		\choice[!]{A, B, and C}
	\end{question}

	\begin{question}
		A fan has a radius of 0.2m.  When the fan is turned on, the blades have a linear acceleration of $a = 2 m/s^2$.  What is the final angular velocity, $\omega_f$, after 3 seconds? 
		\choice{6 rad/s}
		\choice{10 rad/s}
		\choice{15 rad/s}
		\choice[!]{30 rad/s}
	\end{question}

\begin{question}
	A soda can has a radius of 0.03 cm.  If it rolls 12 times, what is the distance the can will roll? 
	\choice[!]{2.62 m}
	\choice{1.13 m}
	\choice{.72m rad/s}
	\choice{.36 m}
\end{question}
	

	\end{multiplechoice}




\end{document}
