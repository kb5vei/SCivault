\documentclass[10pt]{examdesign}
\usepackage{amsmath}
\usepackage{enumitem}
\usepackage{amsfonts}
\usepackage{pifont}
\usepackage{graphicx}
\usepackage{fancyhdr}
\usepackage{cancel}
\SectionFont{\large\sffamily}
\Fullpages
\ContinuousNumbering


\DefineAnswerWrapper{}{}
\NumberOfVersions{4}
%\IncludeFromFile{foobar.tex}
\examname{Common Assessment 6a:\\ Work and Energy}
\class{ {\Large Physics}}

\def \namedata {Name: \hrulefill\\ 
	Date: \hrulefill \\
	Period: \hrulefill
	
}




\begin{document}




\begin{multiplechoice} [title={Multiple Choice},
	rearrange=yes]
	\textit{Choose the best answer to each question.}
	\begin{question}
		Which of the following does \textbf{NOT} contribute to the gravitational potential energy of an object?
		\choice {Height of the object above the earth’s surface.}
		\choice {The acceleration due to gravity of the earth (g).}
		\choice [!]{velocity of the object}
		\choice {mass of the object}
	\end{question}

	\begin{question}
		Which of the following would be the best example of kinetic energy being transformed into potential energy?
		\choice [!] {A ball rolling up a hill}
		\choice {dropping a book}
		\choice {coasting down a hill on a bicycle}
		\choice {starting an automobile engine}
	\end{question}

	\begin{question}
		A ball falls from a height h from a tower. Which of the following statements is true? 
		\choice {The potential energy of the ball is constant as it falls. }
		\choice {The kinetic energy of the ball is constant as it falls. }
		\choice{The difference between the potential energy and kinetic energy is a constant as the ball falls. }
		\choice [!]{The sum of the kinetic and potential energies of the ball is a constant as the ball falls.}
	\end{question}

	\begin{question}
		Which of the following has a meaning closest to that of potential energy? 
		\choice[!]{stored energy}
		\choice{energy at rest}
		\choice{motion energy}
		\choice{gravity energy}
	\end{question}


	\begin{question}
		Which of the following has a meaning closest to that of kinetic energy? 
		\choice{stored energy}
		\choice{potential energy}
		\choice[!]{motion energy}
		\choice{chemical energy}
	\end{question}
	
	\begin{question}
		While riding your bicycle, if you double your speed, your kinetic energy will -
		\choice{be unchanged}
		\choice{increase by a factor of 2}
		\choice[!]{increase by a factor of 4}
		\choice{increase by a factor of 8}
	\end{question}
	
	\begin{question}
		How much work is performed when a 50 kg crate is pushed 15 m with a force of 20 N?
		\choice[!]{300 J}
		\choice{750 J}
		\choice{1,000 J}
		\choice{15,000 J}
	\end{question}

	\begin{question}
		A motor raises a mass of 3.0 kg to a height of 2 meters in 3 seconds.  What is the power provided by the motor? 
		\choice{0.20 Watts }
		\choice{2 Watts}
		\choice{18 Watts}
		\choice[!]{19.6 Watts}
	\end{question}




	

	\end{multiplechoice}



\begin{shortanswer}[title={Free Response},
	rearrange=no]
	\begin{question}
		Calvin and his stuffed tiger, Hobbes, roll down a \word{{45}{25}{65}{75}}m tall hill in a wagon.  The combined mass of Calvin, Hobbes, and the wagon is 35 kg. 
		 
			\includegraphics[width=1in]{../hc.jpg}
			\vspace{-1.25in}
			
		\begin{enumerate}[label=(\alph*),leftmargin=1.25in]		
			\item What is the potential energy of the wagon and its passengers at the top of the hill?
			\vspace{1 in}
			\item b) What is the final velocity of the wagon at the bottom of the hill?  (Assume friction is negligible.)
			\vspace{1 in}
			
		\end{enumerate}
	\end{question}
	\begin{question}
		The space probe Deep Space 1, was launched on October 24, 1998.  It was the first space probe to use an ion engine, that only generates a weak force of 0.056 N, but requires very little fuel.  The probe has a mass of \word{{474 kg} {464 kg} {484kg}{494kg}}.  The engine ran for a long time, causing the probe to move a distance of 2 billion meters.  Assume that the mass of the probe does not change, and no other forces act on the probe.  What is the final speed of the probe?
		\begin{answer}
			
			\end{answer}
		
	
	\end{question}
\end{shortanswer}

\end{document}
