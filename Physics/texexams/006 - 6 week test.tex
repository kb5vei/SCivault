\documentclass[11pt]{examdesign}
\usepackage{amsmath}
\usepackage{enumitem}
\usepackage{amsfonts}
\usepackage{pgfplots}
\usepackage{pifont}
\usepackage{graphicx}
\usepackage{fancyhdr}
\usepackage{cancel}
\usepackage[american]{circuitikz}

\SectionFont{\large\sffamily}
\Fullpages
\ContinuousNumbering
\usepackage{ulem}
\ProportionalBlanks{2}


\DefineAnswerWrapper{}{}
\NumberOfVersions{1}
%\IncludeFromFile{foobar.tex}
\examname{\Large{6 Weeks' Test}}
\class {\Large Physics}

\def \namedata {Name: \hrulefill\\ 
	Date: \hrulefill \\
	Period: \hrulefill \\
	Primary Peer Reviewer: \hrulefill 
	\\
			\begin{tabular}{| p{1cm} | p{1cm} | p{1 cm} | p{1cm} |}
	\hline
		+1 & 0 & -1 & $\Sigma$ 
		\\
		\hline
		& & & \vspace{.25cm}
		\\ \hline
	
	\end{tabular}
	\\
 \vspace{-.65in}
	
}




\begin{document}




\begin{multiplechoice} [title={Multiple Choice},
	rearrange=no]

	
	\begin{question}
	
	Samantha is in a car that is at rest.  She accelerates to 15 m/s in the span of 7.5 seconds.  What was her acceleration?  
	\choice{0.5 m/s\textsuperscript{2}}
	\choice[!]{2 m/s\textsuperscript{2}}
	\choice{7 m/s\textsuperscript{2}}
	\choice{112.5 m/s\textsuperscript{2}}

	\end{question}


\begin{question}
	Daniela is on a ship that is going from London to New York and traveling at 23 m/s, when it spots an iceberg.  The ship slows down at a rate of 0.3 m/s\textsuperscript{2}.  What was the speed of the ship when it strikes the iceberg 37 seconds later?  
	\choice {6.9 m/s}
	\choice [!]{11.9 m/s}
	\choice {34.1 m/s}
	\choice {255.3 m/s}
\end{question}

	\begin{question}
	Travis is driving a car at a speed of 30 m/s when he spots a dinosaur standing on the road. Knowing how dangerous dinosaurs are, he decides to run it over, and speeds up at 3 m/s\textsuperscript{2}. He is traveling at 45 m/s when his collides with the dinosaur.  How far did the car go? 
	\choice [!]{2.5 m}
	\choice {11.25 m}
	\choice {140.625 m}
	\choice {187.5 m}
\end{question}


	\begin{question}
	Enrique drops a rock off a 93 meter building.  If gravity causes the rock to accelerate at a rate of 9.8 m/s\textsuperscript{2}, how long does it take the rock to hit the ground?
	\choice [!]{911.4 s}
	\choice {18.98 s}
	\choice {9.49 s}
	\choice {4.357 s}
\end{question}


\begin{question}
	Two bicycle riders, Lance and Floyd, are racing each other. Lance has a greater top speed of 25 m/s, compared to Floyd's 20 m/s.  However, Floyd has a greater acceleration of 3 m/s\textsuperscript{2}, compared to Lance's 2 m/s\textsuperscript{2}.  Which of the riders will win the race?
	\choice{Lance}
	\choice{Floyd}
	\choice{Either Lance or Floyd can win, depending on who has the more expensive bicycle.}
	\choice{It is impossible to tell without knowing the distance they are riding.} 
	
	
\end{question}


\begin{question}
Luis and Giselle are racing their cars a distance of $\frac{1}{4}$ mile.  Since Giselle is hesitant to race, Luis agrees to give Giselle a head start.  Both cars are identical and have a maximum acceleration of 5 m/s\textsuperscript{2}.  Luis states that she can either have a 2 second head start, or she can start 10 meters in front of him.  Which type of head start gives Giselle a better chance of winning the race?
\choice{The 2 second head start.}
\choice{The 10 meter head start}
\choice{They are the same}
\choice{It cannot be determined.}
	
\end{question}


\begin{question}
Two Rockets are launched at the same time from the same location, and both have the same acceleration.  Rocket 1 accelerates for time $t_1$, and has traveled a distance $d_1$ in that time. Rocket 2 accelerates for 3 times as long, such that $t_2 = 3 t_1$.   What is the distance, $d_2$, that Rocket 2 travels as it accelerates in terms of $d_1$?
\choice{$d_2 = \frac{1}{9} d_1 $}
\choice{$d_2 =  \frac{1}{3} d_1 $}
\choice{$d_2 = d_1 $}
\choice{$d_2 = 3 d_1 $}
\choice{$d_2 = 9 d_1 $}


\end{question}


\end{multiplechoice}


\begin{shortanswer}[title={Free Response},
	rearrange=no]




\begin{question}
A detective is investigating a rear-end traffic collision.  He finds a pair of skid marks on the road, and knows that the type of car involved in the accident is capable of slowing down at a rate of 7.5 m/s\textsuperscript{2} while skidding.  If the speed limit on this road is 20 m/s (about 45 mph),
	\begin{enumerate}
		\item explain how the investigator can determine whether the car was speeding before the collision.
		\vspace{1 in}
		\item If the skid marks are 13.70 m long, determine whether the car was speeding before the collision.
		\vspace{2 in}
	\end{enumerate}
	\end{question}

\begin{question}
	You are working for NASA designing a robotic lander for Mars.  NASA has asked you to design a test to determine whether an airbag landing system will sufficiently protect the robotic lander.  They expect that the retrorocket system would bring the lander to a complete stop approximately 45 meters above the surface of Mars and begin inflating the airbags.  Once the on-board computer detects that the airbags are fully inflated, the rockets will shut off, and the lander will free-fall from a height of 45 meters onto the Martian surface.  The lander is expected to accelerate due to gravity on Mars at 3.7 m/s\textsuperscript{2}, while on earth it accelerates at 9.81 m/s\textsuperscript{2}.
	
	There are several issues that must be kept in mind while testing the landing system, including:
	\begin{itemize}
		\item The gravity of Mars and Earth are not the same.
		\item There is no practical way to test the system on Mars.
		\item The airbags should be kept as light as possible; making them stronger than they need to be will cause the lander to have less scientific equipment due to launch costs.  Launch costs to Mars are approximately \$1,100,000 per kilogram.
		\item A failure of the airbag system would be a waste of hundreds of millions of dollars.
	\end{itemize}
	Write the procedure that should be used in order to determine if the airbag system will work.  You may use diagrams and formulas as needed, but the majority of your response should be a clear, coherently written paragraph.  Be sure to include specific numbers for any distances, times, velocities, and/or accelerations that are included in the procedure.
	
	
	
\end{question}

	\end{shortanswer}



\end{document}


