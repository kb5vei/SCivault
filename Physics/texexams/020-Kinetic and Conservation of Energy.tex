\documentclass[11pt]{examdesign}
\usepackage{amsmath}
\usepackage{enumitem}
\usepackage{amsfonts}
\usepackage{pgfplots}
\usepackage{pifont}
\usepackage{graphicx}
\usepackage{fancyhdr}
\usepackage{cancel}
\usepackage{gensymb}
\usepackage[american]{circuitikz}

\SectionFont{\large\sffamily}
\Fullpages
\ContinuousNumbering
\usepackage{ulem}
\ProportionalBlanks{2}


\DefineAnswerWrapper{}{}
\NumberOfVersions{1}
%\IncludeFromFile{foobar.tex}
\examname{\Large{Conservation of Energy}}
\class {\Large Physics}

\def \namedata {Name: \hrulefill\\ 
	Date: \hrulefill \\
	Period: \hrulefill \\
	Primary Peer Reviewer: \hrulefill 
	\\
			\begin{tabular}{| p{1cm} | p{1cm} | p{1 cm} | p{1cm} |}
	\hline
		+1 & 0 & -1 & $\Sigma$ 
		\\
		\hline
		& & & \vspace{.5cm}
		\\ \hline
	
	\end{tabular}
	\\
 \vspace{-.6in}
	
}




\begin{document}




\begin{multiplechoice} [title={Multiple Corect Multiple Choice},
	rearrange=no]

\textit{For Each queston, chose TWO answers. }
	


\begin{question}
	What kinds of energy are included in mechanical energy? (CHOOSE TWO)
	\choice{Chemical Energy}
	\choice{Kinetic Energy}
	\choice{Potential Energy}
	\choice{Light}
\end{question}

\begin{question}
When a roller coaster cart falls towards the ground, what happens to its kinetic and potential energy? (CHOOSE TWO)
\choice{kinetic energy increases.}
\choice{kinetic energy decreases.}
\choice{kinetic energy remains constant.}
\choice{potential energy increases.}
\choice{potential energy decreases.}
\choice{potential energy remains constant}
\end{question}

\begin{question}
Airplane A has a gravitational potential energy of $2 \times 10^7$ J and a kinetic energy of $4 \times 10^7$ J.  Airplane B is traveling at the same height, but has double the mass and is traveling 3 times faster.  What are the kinetic and gravitational potential energies of Aiplane B? (CHOOSE TWO)
\choice{$U_g = 2 \times 10^7$ J} 
\choice{$U_g = 4 \times 10^7$ J} 
\choice{$U_g = 6 \times 10^7$ J} 
\choice{$K = 1.8 \times 10^8$ J} 
\choice{$K = 4 \times 10^7$ J} 
\choice{$K = 8 \times 10^7$ J} 
\choice{$K = 1.2 \times 10^8$ J} 
\choice{$K = 3.6 \times 10^8$ J} 

\end{question}


\end{multiplechoice} 
\pagebreak



\begin{shortanswer}[title={Free Response}, rearrange=no]
	

	\begin{question} 
		Calvin and his stuffed tiger, Hobbes, roll down a 45 m-tall hill in a wagon.  The combined mass of Calvin, Hobbes, and the wagon is 35 kg.  
		\begin{enumerate}
			\item {What is the potential energy of the wagon and its passengers at the top of the hill?}
				\includegraphics[height=1in]{hc.jpg}
			\vspace{.75in}
			\item{What is the final velocity of the wagon at the bottom of the hill?  (Assume friction is negligible.)}
			\vspace{1.25in}
		\end{enumerate}
	\end{question}

	\begin{question} 
	The space probe Deep Space 1, was launched on October 24, 1998.  It was the first space probe to use an ion engine, that only generates a weak force of 0.056 N, but requires very little fuel.  The probe has a mass of 474 kg.  The engine ran for a long time, causing the probe to move a distance of 2 billion meters.  Assume that the mass of the probe does not change, and no other forces act on the probe.  What is the final speed of the probe?



	\end{question}


	
\end{shortanswer}





\end{document}


