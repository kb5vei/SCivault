\documentclass[12pt]{examdesign}
\usepackage{amsmath}
\usepackage{enumitem}
\usepackage{amsfonts}
\usepackage{pgfplots}
\usepackage{pifont}
\usepackage{graphicx}
\usepackage{fancyhdr}
\usepackage{cancel}
\usepackage{gensymb}
\usepackage[american]{circuitikz}

\SectionFont{\large\sffamily}
\Fullpages
\ContinuousNumbering
\usepackage{ulem}
\ProportionalBlanks{2}


\DefineAnswerWrapper{}{}
\NumberOfVersions{1}
%\IncludeFromFile{foobar.tex}
\examname{\Large{3rd 9 Weeks Test}}
\class {\Large Physics}

\def \namedata {Name: \hrulefill\\ 
	Date: \hrulefill \\
	Period: \hrulefill \\
	Primary Peer Reviewer: \hrulefill 
	\\
			\begin{tabular}{| p{1cm} | p{1cm} | p{1 cm} | p{1cm} |}
	\hline
		+1 & 0 & -1 & $\Sigma$ 
		\\
		\hline
		& & & \vspace{.5cm}
		\\ \hline
	
	\end{tabular}
	\\
 \vspace{-.6in}
	
}




\begin{document}















\begin{multiplechoice} [title={Multiple Choice},
	rearrange=no]
	
	\textit{For Each question, choose the best answer.}
	
	
	Some Formulas: 
	\begin{center}
		$F_c = \frac{mv^2}{r} $	\hspace{1 in} $a_c = \frac{v^2}{r}$ \hspace{1in} $c = 2 \pi r 
		$
		
			\vspace{0.2 cm}	
			
		$	K = \frac{1}{2}mv^2	$ \hspace{1 in} $U_g = mgh$ \hspace{1 in} $W = F \cdot  d \cdot cos (\theta)$
			\vspace{0.2 cm}	
			
			$T_p = 2\pi \sqrt{\frac{l}{g}}$ 
		\vspace{0.1in}
	\end{center}	
	
\begin{question}
	A car is driving on a circular track at a constant speed.  Which of the following statements is true?
	\choice{The car has a constant velocity.}
	\choice{The car's acceleration is 0 m/s\textsuperscript{2}}
	\choice{The magnitude of the force on the car is constant.}
	\choice{Gravity provides the centripetal force.}
	
\end{question}


\begin{question}
	Which of the following has the largest momentum relative to Earth?
	\choice{a tightrope walker crossing Niagara Falls}
	\choice{a pickup truck speeding along a highway}
	\choice{an 18-wheeler parked in a lot}
	\choice{a dog running down the street}
\end{question}


\begin{question}
	A pendulum is made such that it has a period of exactly 1 second.  The pendulum is then sent to the planet Jupiter ($g_{Jupiter} = 24.79 m/s^2$).  What would the period of the pendulum be there?
	\choice{More than 1s}
	\choice{Exactly 1 s}
	\choice{Less than 1 s s}
	\choice{The pendulum will not swing on mars.}	
\end{question}

\begin{question}
	Which of the following would be the best example of kinetic energy being transformed into potential energy?
	\choice{dropping a book}
	\choice{coasting down a hill on a bicycle}
	\choice{starting an automobile engine}
	\choice{A ball rolling up a hill}
\end{question}


\begin{question}
	You do 200 J of work while pushing a box across a frictionless horizontal surface with a force of 10N, directed horizontally.  How far did you push the box?
	\choice{2000 m}
	\choice{20 m}
	\choice{2 m}
	\choice{0.2 m}
	\choice{0 m}
\end{question}

\begin{question}
	A glass of water is at rest on top of a table, and has 4 J of gravitational potential energy.  George picks up the glass of water, doing work on the glass as he raises it to drink.  When the glass has reached its highest position, it has a potential energy of 7 J.  How much work did George do?
	\choice{0 J}
	\choice{3 J}
	\choice{4 J}
	\choice{7 J}
\end{question}

\begin{question}
	Tom drops a rock off of a cliff.  As the rock falls, 
	\choice{the kinetic energy of the rock turns into gravitational potential energy.}
	\choice{the mechanical energy of the rock increases.}
	\choice{the thermal energy of the rock decreases.}
	\choice{the gravitational potential energy of the rock turns into kinetic energy.}
\end{question}



\begin{question}
	An athlete sitting in a wheelchair at rest throws a basketball forward. Since the athlete and the wheelchair have greater mass than the basketball has, the athlete and the wheelchair will — 
	\choice{move backward at a lower speed than the basketball moves forward}
	\choice{travel the same distance as the basketball but in the opposite direction}
	\choice{move backward at a higher speed than the basketball moves forward}
	\choice{move forward faster than the basketball moves forward.}
\end{question}


\begin{question}
	Which of the following objects has its own gravity (MARK ALL THAT APPLY)
	\choice{The Earth}
	\choice{The Moon}
	\choice{An Atom}
	\choice{A 100 kg rock}
	\choice{You}

\end{question}



	
	\end{multiplechoice}

	\pagebreak




\begin{truefalse} [title={True or False},
	rearrange=no]
	\begin{question}
		\answer{false} 10 The Sun's gravity pulls on the Earth more than the Earth's gravity pulls on the Sun.  
	\end{question}
	\begin{question}
		\answer{false} 11. The Moon has no gravity.
	\end{question}

	\begin{question}
	\answer{false} 12. Planets with thin atmospheres have little gravity.
\end{question}

	\begin{question}
	\answer{false} 13. Planets distant from the Sun have less gravity.
\end{question}

	\begin{question}
	\answer{false} 14. Gravity is stronger between the most distant objects.
\end{question}

	\begin{question}
	\answer{false} 15. Space shuttle astronauts are weightless because there is no gravity above earth.
\end{question}


\begin{question}
	\answer{false} 16. Planets revolve around the sun because they are pushed by gravity.
\end{question}



\begin{question}
		\answer{false} 17. The gravitational field of the Earth is infinite in size. 
	\end{question}
		
			\begin{question}
			\answer{True} 18. Doing work on an object causes that object's energy to change.
		\end{question}
		\begin{question}
			\answer{False} 19. All potential energy is due to gravity.
		\end{question}
		\begin{question}
			\answer{True} 20. You do more work playing tag than studying all night for a test.
		\end{question}
		
		\begin{question}
			\answer{False} 21. Force and work are the same thing, just with different units.
		\end{question}
		\begin{question}
			\answer{True} 22. Kinetic energy is best described as energy of motion.
		\end{question}
		
		\begin{question}
			\answer{False} 23. To have potential energy an object must be at rest.
		\end{question}
		\begin{question}
			\answer{True} 24. Potential Energy is best described as stored energy.
		\end{question}
		\begin{question}
			\answer{False} 25. An object traveling at 3 m/s has a kinetic energy of 27J.  This means that the object used up 27J of energy to get going that fast. 
		\end{question}
		
		
		\begin{question}
			\answer{False} 26. When an object is at rest, it has no energy.
		\end{question}
		
		
		\begin{question}
			\answer{False} 27. All energy that exists on Earth can eventually be traced back to the sun.
		\end{question}
		
		\begin{question}
			\answer{True} 28. There are forms of potential energy that are not caused by gravity.
		\end{question}
		
			\begin{question}
			\answer{True} 29. Energy and force are the same thing, just with different units.
		\end{question}
		

		
		
		
		
		
		


\end{truefalse}


		
\pagebreak
\begin{shortanswer}  [title={Essay Questions}, rearrange=no]
Answer each of the following questions in a well-written paragraph of at least\textbf{ 5 sentences.}  
	\begin{question}
	Before being shot, a gun and a bullet have 0 kg m/s of momentum.  When the gun is shot, both the bullet and the gun have momentum (the bullet travels forward at high speed, and the gun travels backward at a much lower speed.).  How can this be if momentum cannot be created or destroyed?
	\vspace{.25 in}
	\hrule
	\vspace{.25 in}
	\hrule
	\vspace{.25 in}
	\hrule
	\vspace{.25 in}
	\hrule
	\vspace{.25 in}
	\hrule
	\vspace{.25 in}
	\hrule
	\vspace{.25 in}
	\hrule
	\vspace{1cm}
		\end{question}
		\begin{question}
		During a scene where Maui and Moana are fighting the Kakamora, Maui is seen pulling on a rope to make his boat turn.  Explain why Maui must pull on the rope and in what direction the boat turns.  
		\vspace{.25 in}
		\hrule
		\vspace{.25 in}
		\hrule
		\vspace{.25 in}
		\hrule
		\vspace{.25 in}
		\hrule
		\vspace{.25 in}
		\hrule
		\vspace{.25 in}
		\hrule
		\vspace{.25 in}
		\hrule
			\vspace{.75cm}
	\end{question}
	
	\begin{question}
	Your 8-year old little brother wants to know how his hot wheels cars can stay on a track with a full loop in it without falling off.  Explain how the cars stay on the track to him in terms he can understand. 
	\vspace{.25 in}
	\hrule
	\vspace{.25 in}
	\hrule
	\vspace{.25 in}
	\hrule
	\vspace{.25 in}
	\hrule
	\vspace{.25 in}
	\hrule
	\vspace{.25 in}
	\hrule
	\vspace{.25 in}
	\hrule
	\vspace{1cm}
\end{question}


	
	
	
	\end{shortanswer}



\end{document}


