\documentclass[10pt]{examdesign}
\usepackage{amsmath}
\usepackage{enumitem}
\usepackage{amsfonts}
\usepackage{pgfplots}
\usepackage{pifont}
\usepackage{graphicx}
\usepackage{fancyhdr}
\usepackage{cancel}
\usepackage{gensymb}
\usepackage[american]{circuitikz}

\SectionFont{\large\sffamily}
\Fullpages
\ContinuousNumbering
\usepackage{ulem}
\ProportionalBlanks{2}


\DefineAnswerWrapper{}{}
\NumberOfVersions{1}
%\IncludeFromFile{foobar.tex}
\examname{\Large{Newton's 2nd Law}}
\class {\Large Physics}

\def \namedata {Name: \hrulefill\\ 
	Date: \hrulefill \\
	Period: \hrulefill \\
	Primary Peer Reviewer: \hrulefill 
	\\
			\begin{tabular}{| p{1cm} | p{1cm} | p{1 cm} | p{1cm} |}
	\hline
		+1 & 0 & -1 & $\Sigma$ 
		\\
		\hline
		& & & \vspace{.5cm}
		\\ \hline
	
	\end{tabular}
	\\
 \vspace{-.6in}
	
}




\begin{document}




\begin{multiplechoice} [title={Multiple Choice},
	rearrange=no]


	
	\begin{question}
Two forces act on a 5 kg object.  One force is 50N to the west.  The other force is 25 N to the east.  What is the acceleration of the object?
\choice{5 m/s\textsuperscript{2} East}
\choice{5 m/s\textsuperscript{2} West}
\choice{15 m/s\textsuperscript{2} East}
\choice{5 m/s\textsuperscript{2} East}
\choice{0 5 m/s\textsuperscript{2}}
	\end{question}


\begin{question}
	A force of 47 N is needed to overcome a frictional force of 6 N to accelerate a 6 kg mass across a floor. What is the acceleration of the mass? 
	\choice{4 m/s\textsuperscript{2}}
	\choice{5 m/s\textsuperscript{2}}
	\choice{7 m/s\textsuperscript{2}}
	\choice{41 m/s\textsuperscript{2}}
	\choice{53 m/s\textsuperscript{2}}
\end{question}


\begin{question}
A man pushes a frictionless shopping cart of mass $m$ with a force $\vec{F}$, causing it to accelerate at $\vec{a}$ .  Suppose the man were to add groceries to the cart such that the new mass of the cart is $3m$.  What force would he need to use in order to have the cart accelerate at the same rate, $a$? 
\choice{$\frac{F}{9}$ }
\choice{$\frac{F}{3}$ }
\choice{$3F$}
\choice{$9F$}
\choice{it is impossible to tell}


\end{question}

\begin{question}
Two cars are racing a distance of 1 mile. One car has a full gas tank and one car has a $\frac{1}{8}$ full tank, but the cars are otherwise identical. Assuming the drivers of the cars are equally skilled, which car has the better chance of winning the race, and why?
\choice{The car with the full gas tank, because a full gas tank has more stored energy than a nearly empty one.}
\choice{The car with the full gas tank, because the car has more inertia.}
\choice{The car with the $\frac{1}{8}$ full tank, because less mass will accelerate more if the same force is applied. }
\choice{The car with the $\frac{1}{8}$ full tank, because when the car accelerates forward, the gas pushes backward on the fuel tank, causing the car to be slower.}
\choice{Both cars will cross the finish line at the same time.}
\end{question}


\end{multiplechoice}

\pagebreak
\begin{multiplechoice} [title={Multiple Correct Choice},
	rearrange=no]
\textit{For each of the following questions choose TWO answers.  No credit will be given for incorrect or partially correct answers.}
\begin{question}
	A space ship is in deep space, far away from any planets or stars.  It is moving forward at a constant speed of 1000 m/s.  Which of the following statements are true? (CHOOSE TWO)
	\choice{There are no forces acting on the space ship.}
	\choice{The engine must be running, providing a forward force.}
	\choice{The spaceship will slowly slow down, eventually coming to a stop.}
	\choice{In order to slow down, the space ship will have to run its engine in reverse.}
	\choice{The space ship cannot speed up or slow down until it comes close to an object.}
\end{question}

\end{multiplechoice}

\begin{shortanswer}  [title={Free Response}, rearrange=no]
	\begin{question}
	In the movie \textit{Sharknado}	a shark is able to change direction while airborne by ``swimming" through the air.  Explain why this cannot happen in real life.  Be sure to include references to applicable laws of physics. 
	\vspace {2 in}
	\end{question}

	\begin{question}
	A tennis ball is at rest when it is hit by a tennis racket.  Its final velocity is 35 m/s, and the force applied to the tennis ball is 10.238 N for 0.2 seconds.  What is the mass of the tennis ball?
	\end{question}
	
	\end{shortanswer}



\end{document}


