\documentclass[12pt]{examdesign}
\usepackage{amsmath}
\usepackage{enumitem}
\usepackage{amsfonts}
\usepackage{pgfplots}
\usepackage{pifont}
\usepackage{graphicx}
\usepackage{fancyhdr}
\usepackage{cancel}
\usepackage[american]{circuitikz}

\SectionFont{\large\sffamily}
\Fullpages
\ContinuousNumbering
\usepackage{ulem}
\ProportionalBlanks{2}


\DefineAnswerWrapper{}{}
\NumberOfVersions{1}
%\IncludeFromFile{foobar.tex}
\examname{\Large{Quiz: Electrostatics}}
\class{ {\Large AP Physics 2}}

\def \namedata {Name: \hrulefill\\ 
	Date: \hrulefill \\
	Period: \hrulefill \\
	Peer Reviewer: \hrulefill \\
	Authentication Code: \hrulefill
	\\
		
	\begin{tabular}{| p{1cm} | p{1cm} | p{1 cm} | p{1cm} |}
	\hline
		+1 & 0 & -1 & $\Sigma$ 
		\\
		\hline
		& & & \vspace{.5cm}
		\\ \hline
	
	\end{tabular}
	\\
 \vspace{-.6in}
	

}




\begin{document}




\begin{multiplechoice} [title={Multiple Choice},
	rearrange=no]
	
	Information: 
	
	Coulomb's Law:$ F_e = \frac{kq_1q_2}{r^2} \hspace{1 in}  k = 9 \times 10^9 Nm^2/C^2$ 
	
	Charge of an electron: $-1.6 \times 10^{-19}$C
	


	\textit{Choose the best answer to each question.} 
	
	\begin{question}
	If an object has a positive charge - 
	 \choice {It has gained extra protons.}
	 \choice {It has gained extra neutrons.}
	 \choice {It has gained extra electrons}
	 \choice [!]{It has lost some of its electrons.}
	\end{question}

	\begin{question}
	If the distance between two objects is doubled, the force between them is  - ?
	\choice {Double the original force.}
	\choice {The same as the original force.}
	\choice {Half the original force.}
	\choice[!]{One fourth the original force.}
	\end{question}

	\begin{question}
	When using Coulomb's law a \textbf{Negative} force indicates - 
	\choice {The force is repulsive.}
	\choice[!] {The force is attractive.}
	\choice {The force is to the left.}
	\choice {The force is downward.}
	\end{question}	

	\begin{question}
	You calculate the charge of an object to be $ -4.2 \times 10^{-20}$C.  You know that this answer is - 
	\choice {wrong because charge cannot be negative.}
	\choice[!] {wrong because the charge is smaller than the elementary charge.}
	\choice {wrong because the charge is faster than the speed of light.}
	\choice{correct, because charges are made of electrons, which are negative.}
	\end{question}		


\begin{question}
	How many electrons would need to be removed from an object for it to have a charge of 0.25 C?  
	\choice{$4 \times 10^{-20}$ electrons}	
	\choice{$6.4 \times 10^{-19}$ electrons}
	\choice[!]{$1.563 \times 10^{18}$ electrons}
	\choice{This is impossible, the object would need to have an excess of electrons.}
\end{question}		






\end{multiplechoice}




\begin{shortanswer}[title={Free Response},
	rearrange=no]
	
	\begin{question}
		 An astronaut has designed a new way to butter his pancakes while in space.  He is able to cause a pancake to have a charge of $1.6 \times 10^{-6}$C, and a blob of butter (m=0.002kg) to have a charge of  $-2.7 \times 10^{-6}$C.  The butter and pancake start off 2 m apart and the pancake is held in place.  
		 \begin{enumerate}
		 	\item What is the electrostatic force that the butter feels?
		 	\vspace{1 in}
		 	\item What is the acceleration of the butter?  (Hint: $F = ma$)
		 	\vspace{1 in}
		 	\item How long will it take for the butter to collide with the pancake? \\ (Hint: $d = v_it + \frac{1}{2}at^2$)
		 	\vspace{1 in}
		 \end{enumerate}

	\end{question}

	
	\end{shortanswer}






\end{document}


