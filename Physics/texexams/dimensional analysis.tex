\documentclass[12pt]{article}
\usepackage{amsmath}
\usepackage{geometry}
\geometry{margin=1in}
\usepackage{array}
\usepackage{multicol}
\usepackage{titlesec}
\usepackage{parskip}

\title{Dimensional Analysis Worksheet: SI Base Units}
\author{}
\date{}

\begin{document}
	
	\maketitle
	\vspace{-5cm}
	\noindent\textbf{Name:} \hrulefill \hfill \textbf{Date:} \hrulefill \hfill \textbf{Class:} \hrulefill
	

\vspace{1.75cm}

	In physics, \textbf{dimensional analysis} helps verify equations, convert units, and understand relationships between physical quantities. All physical quantities can be expressed using \textbf{seven SI base units}.
	
	\section*{The 7 SI Base Quantities}
	
	
	\begin{tabular}{| l | l | c | c |}
		\hline
		\textbf{Quantity} & \textbf{SI Unit} & \textbf{Symbol} & \textbf{Dimensional Symbol} \\
		\hline
		Length & meter & m & [L] \\
		Mass & kilogram & kg & [M] \\
		Time & second & s & [T] \\
		Electric Current & ampere & A & [I] \\
		Temperature & kelvin & K & [\(\Theta\)] \\
		Amount of Substance & mole & mol & [N] \\
		Luminous Intensity & candela & cd & [I] \\
		\hline
	\end{tabular}
	
	\section*{Part A: Basic Dimensions}
	Write the dimensional formula for each quantity below:
	
	\begin{enumerate}
		\item Velocity (m/s): \rule{5cm}{0.4pt}
		\item Acceleration (m/s\textsuperscript{2}): \rule{5cm}{0.4pt}
		\item Force (F = ma): \rule{5cm}{0.4pt}
		\item Energy (E = F·d): \rule{5cm}{0.4pt}
		\item Power (P = E/t): \rule{5cm}{0.4pt}
		\item Pressure (P = F/A): \rule{5cm}{0.4pt}
		\item Charge (Q = I·t): \rule{5cm}{0.4pt}
		\item Potential Difference (V = E/Q): \rule{5cm}{0.4pt}
	\end{enumerate}
	
	\newpage
	\section*{Part B: Unit Conversions Using Dimensional Analysis}
	Convert the following quantities. Show your work using conversion factors.
	
	\begin{enumerate}
		\item Convert 120 km/hr to m/s. \vspace{2cm}
		\item Convert 5.0 grams per cubic centimeter (g/cm\textsuperscript{3}) to kg/m\textsuperscript{3}. \vspace{2cm}
		\item Convert 72,000 seconds into days. \vspace{2cm}
		\item A car travels 60 miles/hour. What is this in meters per second? \vspace{2cm}
	\end{enumerate}
	
	\section*{Part C: Apply Dimensional Analysis}
	
	\begin{enumerate}
		\item The formula for the period of a pendulum is:
		\[
		T = 2\pi \sqrt{\frac{L}{g}}
		\]
		where \(T\) is time, \(L\) is length, and \(g\) is acceleration due to gravity. \\
		\textbf{Verify} the dimensional consistency of this equation. \vspace{2cm}
		
		\item Suppose you are given an equation for force:
		\[
		F = mv
		\]
		Is this dimensionally correct? Explain why or why not. \vspace{2cm}
		
		\item A student proposes that the equation for kinetic energy is:
		\[
		KE = \frac{1}{2}mv^3
		\]
		Use dimensional analysis to determine whether this formula is valid. \vspace{2cm}
		
		\item The Stefan-Boltzmann Law relates the power \(P\) radiated by a blackbody to its temperature \(T\):
		\[
		P = \sigma A T^4
		\]
		where \(\sigma\) is the Stefan-Boltzmann constant, \(A\) is surface area, and \(T\) is temperature. \\
		Use dimensional analysis to determine the dimensions of the constant \(\sigma\). \\
		\textit{Hint: Power has dimensions \([ML^2T^{-3}]\), Area has \([L^2]\), and Temperature has \([\Theta]\).} \vspace{3cm}
	\end{enumerate}
	
\end{document}
