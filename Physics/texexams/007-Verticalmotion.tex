\documentclass[10pt]{examdesign}
\usepackage{amsmath}
\usepackage{enumitem}
\usepackage{amsfonts}
\usepackage{pgfplots}
\usepackage{pifont}
\usepackage{graphicx}
\usepackage{fancyhdr}
\usepackage{cancel}
\usepackage[american]{circuitikz}

\SectionFont{\large\sffamily}
\Fullpages
\ContinuousNumbering
\usepackage{ulem}
\ProportionalBlanks{2}


\DefineAnswerWrapper{}{}
\NumberOfVersions{1}
%\IncludeFromFile{foobar.tex}
\examname{\Large{Graphing Motion}}
\class {\Large Physics}

\def \namedata {Name: \hrulefill\\ 
	Date: \hrulefill \\
	Period: \hrulefill \\
	Primary Peer Reviewer: \hrulefill 
	\\
			\begin{tabular}{| p{1cm} | p{1cm} | p{1 cm} | p{1cm} |}
	\hline
		+1 & 0 & -1 & $\Sigma$ 
		\\
		\hline
		& & & \vspace{.5cm}
		\\ \hline
	
	\end{tabular}
	\\
 \vspace{-.6in}
	
}




\begin{document}




\begin{multiplechoice} [title={Multiple Choice},
	rearrange=no]


	
	\begin{question}
	
Kayla drops a rock off of a 75m high cliff.  How long does it take of the rock to land at the bottom of the cliff?  
	 \choice [!]{1.195 s}
	 \choice {3.910 s}
	 \choice {7.645 s}
	 \choice {15.291 s}
	\end{question}


\begin{question}
	America throws a softball directly upward.  If the ball was in the air for 4.21 seconds, how high did the ball go?
	\choice {41.3 m}
	\choice {21.734 m}
	\choice {20.65 m}
	\choice {There is not enough information to solve this problem.}
\end{question}


\begin{question}
	Javier drops a rock off of a tall building.  Half a second later, he drops another rock.  Assuming air resistance can be ignored, as the rocks fall the distance between them will - 
	\choice{increase}
	\choice{decrease}
	\choice{remain the same}
	\choice{either increase or decrease depending on the height of the building}
	\choice{it cannot be determined without knowing the mass of each rock}
\end{question}


	\begin{question}
A person standing on the surface of the Earth throws an object directly upward with an initial velocity of 25 m/s.  In the absence of air resistance, what is the acceleration of the object at the top of its path?
	\choice {0 m/s\textsuperscript{2}}
	\choice {9.81 m/s\textsuperscript{2} downward}
	\choice {It depends on the mass of the object, because heavier objects accelerate faster.}
	\choice {It depends on the size of the object, because larger objects accelerate faster.}
\end{question}


	\begin{question}
An astronaut is standing on the moon.  He holds a feather in one hand and a hammer in the other at the same height above the moon's surface.  Which statement best describes what happens when the astronaut releases both objects at the same time?
	\choice {Both objects float away.}
	\choice {The feather lands on the moon first.}
	\choice {The hammer lands on the moon first.}
	\choice {Both objects land on the moon at the same time.}
\end{question}

\begin{question}
Beauford and Beaulah are playing frisbee near a 10-meter high cliff when the frisbee goes over the edge of the cliff and becomes stuck on a branch that is protruding from the cliff 5 meters above the ground below.  In an attempt to dislodge the frisbee, Beauford stands at the bottom of the cliff and throws a rock at 15 m/s upward.  At the same time, Beaulah stands at the top of the cliff and throws a second rock downward at 15 m/s.  Who's rock will hit the frisbee first?
\choice{Beauford's rock hits the frisbee first.}
\choice{Beaulah's rock hits the frisbee first.}
\choice{Both rocks hit the frisbee at the same time.}
\choice{It cannot be determined without knowing the mass of the rocks.}
	\end{question}

\begin{question}
In order for a ball to move upwards can its initial velocity be zero?
\choice{No, a ball with zero initial velocity will not move.}
\choice{No, in order for the object to go up its initial velocity must be greater than zero (assuming up is positive).}
\choice{Yes, if the ball is light enough it will move upward. }
\choice{Yes, if the ball is thrown upward its initial velocity could still be zero.}

	
\end{question}


\begin{question}
A man named Jayne* has just robbed one of the most powerful men in the galaxy, and is attempting to flee the planet.  His spaceship is having trouble coping with the weight of his loot, and is stuck hovering at a height of 310 m above a large, muddy swamp.  In an act that the mud-workers will sing songs about for years, he pushes a large crate filled with gold coins out the back of the ship in an attempt to lighten the load.  Jayne gets away, but his loot is quickly gathered up and split by the mud-workers.  If the acceleration due to gravity on this planet was 7.6 m/s\textsuperscript{2}, how long did it take the crate of gold coins to fall from the spaceship to the surface of the planet?  
\choice {81.579 s}
\choice {40.789 s}
\choice {31.600 s}
\choice[!]{9.032 s}
\choice {7.950 s}

*1 Bonus point for anyone who can identify the source of this question.


\end{question}






\end{multiplechoice}

\begin{multiplechoice} [title={Multiple Correct Multiple Choice},
	rearrange=no]
	\textit{For the following question, \textbf{choose two} correct answers.  No credit will be given for incorrect or partially correct answers.  Mark \textbf{both} answers clearly.} 



\begin{question}
	A bullet is shot upward at 250 m/s.  You chose to make upward the positive direction.  What statements are true concerning the bullet? 
	\choice {As the bullet rises, its initial velocity is positive.}
	\choice {As the ball rises, its initial velocity is negative.}
	\choice {As the ball rises, its acceleration is positive.}
	\choice {As the ball rises, its acceleration is negative.}
\end{question}







\end{multiplechoice}

\end{document}



