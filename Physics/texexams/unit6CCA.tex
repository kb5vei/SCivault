\documentclass[10pt]{examdesign}
\usepackage{amsmath}
\usepackage{enumitem}
\usepackage{amsfonts}
\usepackage{pgfplots}
\usepackage{pifont}
\usepackage{graphicx}
\usepackage{fancyhdr}
\usepackage{cancel}
\SectionFont{\large\sffamily}
\Fullpages
\ContinuousNumbering


\DefineAnswerWrapper{}{}
\NumberOfVersions{1}
%\IncludeFromFile{foobar.tex}
\examname{Quiz: Angular Velocity and Acceleration}
\class{ {\Large Physics}}

\def \namedata {Name: \hrulefill\\ 
	Date: \hrulefill \\
	Period: \hrulefill
	
}




\begin{document}




\begin{multiplechoice} [title={Multiple Choice},
	rearrange=no]
	\textit{Choose the best answer to each question.}
	\begin{question}
		The \word {{momentum of}{impulse on}} an object is - 
		\choice {its mass times its acceleration}
		\choice [!]{\word{{its mass times its velocity}{its force times time}}}
		\choice {its force times its acceleration}
		\choice {\word{{its force times time}{its mass times its velocity}}}
	\end{question}

	\begin{question}
	 	A large truck and a small car both moving at 15 m/s.  Which has the greater momentum?
		\choice [!] {The large truck}
		\choice {The small car}
		\choice {Both have the same amount of momentum}
		\choice {It is impossible to tell}
	\end{question}

	\begin{question}
		A train car is rolling on a track when it collides with and sticks to another identical train car that is initially at rest.  Compared to the velocity of the first car before the collision, the velocity of the combined cars will be - 
		\choice {Twice as large.}
		\choice {The same.}
		\choice [!]{One half as large.}
		\choice {0 m/s}
	\end{question}



	\begin{question}
	You are playing golf, where you are about to hit a golf ball.  Which of the following would increase the final momentum of a golf ball?
		\choice{Increase the force acting on it.}
		\choice{Increase the time of contact with the ball.}
		\choice{Increase the initial momentum of the golf club.}
		\choice[!]{All of the above}
	\end{question}
	
	\begin{question}
		When a rifle shoots a bullet, the speed of the rifle’s recoil is small compared to the speed of the bullet.  Why?
		\choice{The force against the rifle is smaller than the force on the bullet.}
		\choice{The impulse on the rifle is less than the impulse on the bullet.}
		\choice[!]{The rifle has a greater mass than the bullet.}
		\choice{The momentum of the rifle does not change.}
	\end{question}
	
	\begin{question}
		In a perfectly \underline{\word{{inelastic}{elastic}}} collision - 
		\choice[!]{\word{{the two objects stick together.}{kinetic energy is conserved.}}}
		\choice{\word{{kinetic energy is conserved.}{the two objects stick together.}}}
		\choice{the initial velocities are equal to the final velocities.}
		\choice{the mass of each object must change.}
	\end{question}

	\begin{question}
		A 600 kg car is rolling at 5 m/s.  What is its momentum?
		\choice{1000 kg m/s}
		\choice{2000 kg m/s}
		\choice[!]{3000 kg m/s}
		\choice{4000 kg m/s}
	\end{question}


	\begin{question}
You kick a ball using a force of 36 N.  You deliver an impulse of 9 kg m/s to the ball.  How long was your foot in contact with the ball?
	\choice{0.2 s}
	\choice[!]{0.25 s}
	\choice{0.4s}
	\choice{4 s}
\end{question}

	\begin{question}
You are pushing on a frictionless shopping cart that has a mass of 9kg and is initially at rest.  If you use a force of 15N and push for 3 seconds, what will the final velocity of the shopping cart be?
	\choice{2 m/s}
	\choice{3 m/s}
	\choice{4 m/s}
	\choice[!]{5 m/s}
\end{question}


	\begin{question}
	10. A 1500 kg dump truck is traveling to the east at 20 m/s when it collides with a 750 kg car that is traveling to the west at 30 m/s.  The two vehicles stick together.  What is the final velocity of the wreck?
	\choice{0.222 m/s West }
	\choice{3.333 m/s East}
	\choice[!]{0.444 m/s East}
	\choice{10 m/s West}
\end{question}

	

	\end{multiplechoice}




\end{document}
