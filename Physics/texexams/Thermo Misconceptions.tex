\documentclass[10pt]{examdesign}
\usepackage{amsmath}
\usepackage{enumitem}
\usepackage{amsfonts}
\usepackage{pgfplots}
\usepackage{pifont}
\usepackage{graphicx}
\usepackage{fancyhdr}
\usepackage{cancel}
\usepackage[american]{circuitikz}

\SectionFont{\large\sffamily}
\Fullpages
\ContinuousNumbering
\usepackage{ulem}
\ProportionalBlanks{2}


\DefineAnswerWrapper{}{}
\NumberOfVersions{1}
%\IncludeFromFile{foobar.tex}
\examname{\Large{Quiz: Thermodynamics}}
\class{ {\Large AP Physics 2}}

\def \namedata {Name: \hrulefill\\ 
	Date: \hrulefill \\
	Period: \hrulefill \\
	Peer Reviewer: \hrulefill \\
	Authentication Code: \hrulefill
	\\
		
	\begin{tabular}{| p{1cm} | p{1cm} | p{1 cm} | p{1cm} |}
	\hline
		+1 & 0 & -1 & $\Sigma$ 
		\\
		\hline
		& & & \vspace{.5cm}
		\\ \hline
	
	\end{tabular}
	\\
 \vspace{-.6in}
	
}




\begin{document}


test

\begin{truefalse}[title={True or False},
	rearrange=yes]
	
	\begin{question}
		\answer{False} Temperature measures the intensity of heat. 
	\end{question}

	\begin{question}
		\answer{False} Nerves in your skin can feel temperature. 
	\end{question}

	\begin{question}
	\answer{False} Perceptions of hot and cold are unrelated to energy transfer. 
\end{question}
	 
	\begin{question}
		\answer{False} Water that has boiled for 15 minutes will be hotter than water that has boiled for 5 minutes. 
	\end{question}
	\begin{question}
	\answer{False} The boiling point of a substance is the maximum temperature that substance can reach.
\end{question}

\begin{question}
	\answer{False} A cold object contains no heat.
\end{question}

\begin{question}
	\answer{False} The temperature of an object depends on its size.
\end{question}

\begin{question}
	\answer{False} No matter how cold something gets, there is always a temperature that is colder. 
\end{question}


\begin{question}
	\answer{False} Heat always travels upward.  
\end{question}

\begin{question}
	\answer{False}  Heat and cold flow like fluids.
\end{question}
\begin{question}
	\answer{False} Temperature can be transferred.  
\end{question}
\begin{question}
	\answer{False} Objects of different temperature that are in contact with each other or in contact with air at different temperature, do not necessarily move toward the same temperature.
\end{question}
\begin{question}
	\answer{False} Hot objects naturally cool down, cold objects naturally warm up.  
\end{question}
\begin{question}
	\answer{False} Heat flows more slowly through conductors making them feel hot. 
\end{question}

\begin{question}
	\answer{False} Temperature is a property of a particular material or object.  
\end{question}

\begin{question}
	\answer{False}  Metal has the ability to attract, hold, intensify or absorb heat and cold.  
\end{question}


\begin{question}
	\answer{False} Objects that readily become warm do not readily become cold.   
\end{question}

\begin{question}
	\answer{False} Different materials hold the same amount of heat.  
\end{question}
\begin{question}
	\answer{False} The boiling point of water always 100°C.  
\end{question}

\begin{question}
	\answer{False} Ice is at 0 $\degreeCelsius$ and cannot change temperature.   
\end{question}

\begin{question}
	\answer{False} Liquid water cannot exist at 0$\degreeCelsius. $
\end{question}

\begin{question}
	\answer{False} Steam is always hotter than 100$\degreeCelsius. $
\end{question}



\begin{question}
	\answer{False} Materials like wool have the ability to warm things up. 
\end{question}

\begin{question}
	\answer{False}  Some materials are difficult to heat: they are more resistant to heating.  
\end{question}
\begin{question}
	\answer{False} Bubbles mean boiling. 
\end{question}
\begin{question}
	\answer{False} The bubbles in boiling water contain either air or pure oxygen.  
\end{question}

\begin{question}
	\answer{False} It takes energy to remove coldness from an object.  
\end{question}

\begin{question}
	\answer{False} On a cold day, you keep the door closed in order to keep the cold out.
\end{question}
	
\begin{question}
\answer{False} Objects with greater specific heat capacity (c) change temerature more easily than those with lower specific heat capacity.  
\end{question}






\end{truefalse}

\end{document}





