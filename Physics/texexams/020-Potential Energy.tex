\documentclass[12pt]{examdesign}
\usepackage{amsmath}
\usepackage{enumitem}
\usepackage{amsfonts}
\usepackage{pgfplots}
\usepackage{pifont}
\usepackage{graphicx}
\usepackage{fancyhdr}
\usepackage{cancel}
\usepackage{gensymb}
\usepackage[american]{circuitikz}

\SectionFont{\large\sffamily}
\Fullpages
\ContinuousNumbering
\usepackage{ulem}
\ProportionalBlanks{2}


\DefineAnswerWrapper{}{}
\NumberOfVersions{1}
%\IncludeFromFile{foobar.tex}
\examname{\Large{Energy}}
\class {\Large Physics}

\def \namedata {Name: \hrulefill\\ 
	Date: \hrulefill \\
	Period: \hrulefill \\
	Primary Peer Reviewer: \hrulefill 
	\\
			\begin{tabular}{| p{1cm} | p{1cm} | p{1 cm} | p{1cm} |}
	\hline
		+1 & 0 & -1 & $\Sigma$ 
		\\
		\hline
		& & & \vspace{.5cm}
		\\ \hline
	
	\end{tabular}
	\\
 \vspace{-.6in}
	
}




\begin{document}




\begin{multiplechoice} [title={Multiple Choice},
	rearrange=no]


	
	\begin{question}
Which of the following has a definition that is closest to \textbf{stored} energy? 
\choice{Kinetic Energy}
\choice{Mechanical Energy}
\choice{Thermal Energy}
\choice{Potential Energy}
	\end{question}



\begin{question}
A box that is lifted to a height \textit{h} has a potential energy of 15 J.  What would the potential energy of the box be if it was lifted a height of \textit{4h}? 
\choice{3.75 J}
\choice{15 J}
\choice{30 J}
\choice{60 J}
\end{question}


\begin{question}
Tom drops a rock off of a cliff.  As the rock falls, 
\choice{the kinetic energy of the rock turns into gravitational potential energy.}
\choice{the mechanical energy of the rock increases.}
\choice{the thermal energy of the rock decreases.}
\choice{the gravitational potential energy of the rock turns into kinetic energy.}
\end{question}

\begin{question}
Which of the following would be the best example of kinetic energy being transformed into potential energy?
	\choice{dropping a book}
	\choice{coasting down a hill on a bicycle}
	\choice{starting an automobile engine}
	\choice{A ball rolling up a hill}
\end{question}

\begin{question}
	A ball falls from a height h from a tower. Which of the following statements is true?
	\choice{The potential energy of the ball is constant as it falls.}
	\choice{The kinetic energy of the ball is constant as it falls.}
	\choice{The difference between the potential energy and kinetic energy is a constant as the ball falls.}
	\choice{The sum of the kinetic and potential energies of the ball is a constant as the ball falls.}
\end{question}






\end{multiplechoice} 
\pagebreak

\begin{truefalse} [title={True or False},
	rearrange=no]
	\textit{Mark Each answer as True or False}
	
	\begin{question}
		\answer{True} Doing work on an object causes that object's energy to change.
	\end{question}
	\begin{question}
		\answer{False} All potential energy is due to gravity.
	\end{question}
	\begin{question}
	\answer{True} You do more work playing tag than studying all night for a test.
	\end{question}

	\begin{question}
	\answer{False} Force and work are the same thing, just with different units.
	\end{question}
	\begin{question}
	\answer{True} Kinetic energy is best described as energy of motion.
	\end{question}

	\begin{question}
	\answer{False} To have potential energy an object must be at rest.
\end{question}
\begin{question}
	\answer{True} Potential Energy is best described as stored energy.
\end{question}
\begin{question}
	\answer{False} An object traveling at 3 m/s has a kinetic energy of 27J.  This means that the object used up 27J of energy to get going that fast. 
\end{question}


\begin{question}
	\answer{False} When an object is at rest, it has no energy.
\end{question}


\begin{question}
	\answer{False} All energy that exists on Earth can eventually be traced back to the sun.
\end{question}

\begin{question}
	\answer{True} There are forms of potential energy that are not caused by gravity.
\end{question}

\end{truefalse}


\begin{shortanswer}[title={Free Response}, rearrange=no]
	\begin{question}
A 0.25 kg ball is dropped from a height of 3.5 m.  
	\begin{enumerate}
		\item {What is the gravitational potential energy of the ball before it is dropped?}
		\vspace{1 in}
		\item {What is the kinetic energy of the ball just before it hits the ground?}
				\vspace{1 in}
		\item{What is the speed the ball will be moving before it hits the ground?}
	\end{enumerate}
	\end{question}
	
\end{shortanswer}





\end{document}


