\documentclass[10pt]{examdesign}
\usepackage{amsmath}
\usepackage{enumitem}
\usepackage{amsfonts}
\usepackage{pgfplots}
\usepackage{pifont}
\usepackage{graphicx}
\usepackage{fancyhdr}
\usepackage{cancel}

\SectionFont{\large\sffamily}
\Fullpages
\ContinuousNumbering
\usepackage{ulem}
\ProportionalBlanks{2}


\DefineAnswerWrapper{}{}
\NumberOfVersions{1}
%\IncludeFromFile{foobar.tex}
\examname{Quiz: Circuits}
\class{ {\Large AP Physics 2}}

\def \namedata {Name: \hrulefill\\ 
	Date: \hrulefill \\
	Period: \hrulefill
	
}




\begin{document}




\begin{multiplechoice} [title={Multiple Choice (5 Points Each)},
	rearrange=no]
	\textit{Choose the best answer to each question.} 
	
	\begin{question}
	 While you are driving on the highway, a bug collides with the windshield of your car.  The impulse on the bug is - 
		\choice [!]{equal to the impulse on the car and in the opposite direction.}
		\choice {less than the impulse on the car and in the same direction.}
		\choice {greater than the impulse on the car and in the opposite direction.}
		\choice {equal to the impulse on the car and in the same direction. }
	\end{question}

	\begin{question}
	The \word{{impulse on}{momentum of}} an object is best defined as - 
	\choice {its mass times its acceleration}
	\choice {\word{{its mass times its velocity}{its force times time}}}
	\choice {its force times its acceleration}
	\choice [!]{\word{{its force times time}{its mass times its velocity}}}
\end{question}

	\begin{question}
	 The Earth orbits the sun at a speed of $3 \times 10^{5}$ m/s and has a mass of $5.972 \times 10^{24}$ kg.  Jupiter orbits the sun at a speed of $1.307 \times 10^5$ m/s and has a mass of $1.898 \times 10^{27}$ kg.  Which planet has the greater momentum?
		\choice [!] {Jupiter}
		\choice {Earth}
		\choice {Both have the same amount of momentum}
		\choice {It is impossible to tell}
	\end{question}

	\begin{question}
		A hunting whale is swimming at 6 m/s when it catches a sleeping giant squid of the same mass in its mouth.  Immediately after catching the squid, how fast will the two be moving?
		\choice {12 m/s}
		\choice {6 m/s}
		\choice [!]{3 m/s}
		\choice {It is impossible to tell. }
	\end{question}



	\begin{question}
		When a spring is stretched 4 cm, it has an elastic potential energy of 4 J.  How much elastic potential energy would the spring have if it were stretched only 2 cm? 
		\choice[!]{1 J}
		\choice{2 J}
		\choice{3 J}
		\choice{4 J}
	\end{question}
	
	\begin{question}
		You find that your grandfather clock runs too slowly (each swing is slightly longer than one second).  What should you do to correct this problem?
		\choice{Increase the mass of the bob.}
		\choice{Increase the angle to which the pendulum swings.}
		\choice[!]{Decrease the length of the pendulum.}
		\choice{Move to somewhere where gravity is weaker.}
	\end{question}
	
	\begin{question}
		Which of the following is an example of an \textbf{\underline{inelastic}} collision.
		\choice{William dribbles a basketball.}
		\choice{Diego kicks a soccer ball.}
		\choice[!]{Victor throws his gum and it sticks to the wall.}
		\choice{Martin punches the wall and makes a hole in it.}
	\end{question}

	\begin{question}
		A hammer falls from a roof of height $h$ and lands moving at a certain speed $v$.  If the hammer fell from a roof of height $4h$ what would be the speed the hammer is moving when it hits the ground? 
		\choice{v/2}
		\choice{v}
		\choice[!]{2v}
		\choice{4v}
	\end{question}


	\begin{question}
	According to physics, which of the following would be the most work? 
	\choice{doing a 100-question physics test}
	\choice[!]{running a marathon}
	\choice{writing a 30-page essay}
	\choice{Sleeping in a tree}
\end{question}

	\begin{question}
	A dog is at the back of an empty boat when he sees an interesting fish jump near the front of the boat.  The dog runs 4 meters east, to the front of the boat, then stops.  The dog has a mass of 30kg, and the boat has a mass of 60 kg.  If there is no friction between the boat and the water, how far does the boat move? 
	\choice{1 m}
	\choice[!]{2 m}
	\choice{3 m}
	\choice{4 m}
\end{question}


\begin{question}
	A truck is rolling on a level, frictionless road with the engine turned off. Rain begins to fall, and over the next few minutes, the bed of the truck fills with water.  According to the Law of Conservation of Momentum, the speed of the truck should - 
	\choice{Increase}
	\choice[!]{Decrease}
	\choice{Remain the Same}
	\choice{It cannot be determined without knowing the mass of the truck.}
\end{question}

\begin{question}
	\word{{Kinetic Energy}{Potential Energy}} is best described as - 
	\choice[!]{\word{{motion energy}{stored energy}}}
	\choice{\word{{stored energy}{motion energy}}}
	\choice{thermal energy}
	\choice{chemical energy}
\end{question}

\begin{question}
	Which statement best describes how energy changes as you go down a large hill on a roller coaster?
	\choice{Kinetic energy becomes gravitational potential energy.}
	\choice[!]{Gravitational potential energy becomes kinetic energy. }
	\choice{Kinetic energy becomes elastic potential energy.}
	\choice{Elastic potential energy becomes kinetic energy.}
\end{question}

\begin{question}
	A TIE starfighter is flying at 15 m/s horizontally through empty space when its reactor goes critical and it explodes.  The center of mass of the explosion will - 
	\choice{be stationary where the fighter exploded.}
	\choice[!]{continue to move at 15 m/s.}
	\choice{move at less than 15 m/s.}
	\choice{move at more than 15 m/s. }
\end{question}


	\end{multiplechoice}

\begin{multiplechoice} [title={Multiple Correct Multiple Choice},
	rearrange=yes]
 	\textit{For each of the following questions, choose TWO answers.  }
	\begin{question}
	In the 1995 movie \textit{Operation Dumbo Drop}, an elephant is dropped out the back of a plane (don't worry, he has a really big parachute).    As the elephant is dropped out of the cargo hold, what types of energy does the elephant have? \textbf{(CHOOSE TWO)}
		\choice{Kinetic Energy}
		\choice[!]{Gravitational Potential Energy}
		\choice[!]{Elastic Potential Energy}
		\choice{Nuclear Energy}
	\end{question}

\begin{question}
Changing which of the following variables can change the period of oscillation of a pendulum? \textbf{(CHOOSE TWO)}
\choice {Amplitude of Oscillation}
\choice{Mass}
\choice[!]{Gravity}
\choice[!]{Length of the Pendulum}
\end{question}





\end{multiplechoice}





\begin{shortanswer}[title={Free Response}, rearrange=NO]

	
	\begin{question}
		A 75 kg wide-receiver is traveling north at 7 m/s.  A 100 kg linebacker runs directly east at 5 m/s. The linebacker attempts to tackle the wide-receiver (and thus the two are stuck together after colliding). 
		\begin{enumerate}
			\item What is the final velocity of the two players?  
			\vspace{.75in}
			\item What is their direction of travel?
			\word{{\vspace{.5in}}{\vspace{.75in}}}
			
		\end{enumerate}
	\end{question}

	\begin{question}
	Roger uses a slingshot to launch a 0.02 kg rock directly up.  The rubber band on the slingshot has a  spring constant of 1200 N/m.  If he pulls the slingshot back 0.12 m, how high does the rock go?
	\word{{\vspace{1.2in}}{\vspace{.75in}}}
	
	\end{question}


	\begin{question}
	You have landed on the planet mercury's surface and wish to determine its gravity.  You have the equipment listed below available:
	\begin{center}
	\begin{tabular}{|c|c|c|c|}
		\hline
		Flashlight & Batteries & Dental Floss & Meter Stick \\
		\hline
		Video Camera & Stopwatch & Computer with Graphing Software & Paper and Pencil \\
		\hline
		Astronaut Ice Cream & Thermometer & Exercise Bicycle & Laser\\
		\hline
		Graph Paper & Duct Tape & Assorted Legos & Shaving Cream \\
		\hline

		 
	\end{tabular}  
	\end{center}	
	\begin{enumerate} 
		\item Circle all the equipment you will use to determine gravity on Mercury.
		\item In a well-written paragraph of at least 5 sentences, describe the process for determining the gravity of mercury.  Be sure to explain how each piece of equipment will be used and include any important formulas.
				\vspace{2 in}
	\end{enumerate}
	\end{question}


	\begin{question}
	\textit{This question requires you to collect data at a lab station.}  Go to a lab station that is unoccupied.
	\begin{enumerate}
		\item Record your lab station number:
		\item Hang a 100 gram mass at the 90-cm mark of the meterstick.  Carefully adjust the meter stick so that it balances on the green base.  Record where the meterstick is balanced:
		\item Place mass, meterstick, and green base where you found them. Return to your desk.  Use the information you gathered to calculate the mass of your meterstick. 
		\vspace{1in}
	\end{enumerate} 

	\end{question}



\end{shortanswer}





\end{document}


