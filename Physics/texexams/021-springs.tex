\documentclass[11pt]{examdesign}
\usepackage{amsmath}
\usepackage{enumitem}
\usepackage{amsfonts}
\usepackage{pgfplots}
\usepackage{pifont}
\usepackage{graphicx}
\usepackage{fancyhdr}
\usepackage{cancel}
\usepackage{gensymb}
\usepackage[american]{circuitikz}

\SectionFont{\large\sffamily}
\Fullpages
\ContinuousNumbering
\usepackage{ulem}
\ProportionalBlanks{2}


\DefineAnswerWrapper{}{}
\NumberOfVersions{1}
%\IncludeFromFile{foobar.tex}
\examname{\Large{Springs and Pendulums}}
\class {\Large Physics}

\def \namedata {Name: \hrulefill\\ 
	Date: \hrulefill \\
	Period: \hrulefill \\
	Primary Peer Reviewer: \hrulefill 
	\\
			\begin{tabular}{| p{1cm} | p{1cm} | p{1 cm} | p{1cm} |}
	\hline
		+1 & 0 & -1 & $\Sigma$ 
		\\
		\hline
		& & & \vspace{.5cm}
		\\ \hline
	
	\end{tabular}
	\\
 \vspace{-.6in}
	
}




\begin{document}




\begin{multiplechoice} [title={Multiple Choice},
	rearrange=no]

\textit{For Each queston, chose the best answer.}

	Some Formulas: 
	\begin{center}
	$F_s = -kx$ \hspace {1in} $	U_s = \frac{1}{2}kx^2 $ \hspace{1in} 	$U_g = mgh$
	\vspace{0.1in}

$k = \frac{1}{2} m v^2$ \hspace{1in} $	T_s = 2 \pi \sqrt{\frac{m}{k}} $ \hspace{1in} $	T_p = 2 \pi \sqrt{\frac{l}{g}} $ 
	\vspace{0.1in}
	\end{center}	
	


\begin{question}
Which is the best definition of the spring constant (k)? 
	\choice{A quantity that measures how stiff a spring is.}
	\choice{Energy stored in a spring.}
	\choice{9.81 m/s\textsuperscript{2}. }
	\choice{A quantity that measures how fast a spring is moving.}
\end{question}

\begin{question}
Which of the following is the best definition of Elastic Potential Energy?
	\choice{A quantity that measures how stiff a spring is.}
\choice{Energy stored in a spring.}
\choice{9.81 m/s\textsuperscript{2}. }
\choice{A quantity that measures how fast a spring is moving.}
\end{question}

\begin{question}
A spring is stretched 0.05m, and stores an elastic potential energy of 5J.  If the spring were stretched 0.15m, how much elastic potential energy would the spring store?
\choice{10 J}
\choice{15 J}
\choice{45 J}
\choice{225 J}


\end{question}


\begin{question}
	A pendulum is made such that it has a period of exactly 1 second.  The pendulum is then sent to the planet mars ($g_mars = 3.711 m/s^2$).  What would the period of the pendulum be there?
	\choice{0.248 s}
	\choice{1 s}
	\choice{1.626 s}
	\choice{The pendulum will not swing on mars.}
	
	
\end{question}

\begin{question}
A 2 kg mass is attached to a spring, and set in oscillatory motion such that the system has a period of 1 second.  If the 2 kg mass were replaced with a 4 kg mass, what would the period of the system be?
\choice{1 second}
\choice{$\sqrt{2}$ seconds}
\choice{2 seconds}
\choice{4 seconds}
	
\end{question}


\end{multiplechoice} 




\begin{shortanswer}[title={Free Response}, rearrange=no]
	

	\begin{question} 
	A spring has an elastic potential energy of 20J when it is stretched a distacne of 0.25m.  
	\begin{enumerate}
		\item What is the spring constant of the spring?
		\vspace{1in}
		\item What is the force that the spring is exerting?
		\vspace{1 in}
		\item A 0.2 kg mass is attached to the spring and allowed to oscillate.  What is the period of oscillation?
		\vspace{1in}
	\end{enumerate}



	\end{question}


\begin{question}
A pendulum has a length of 0.5 meters.  A spring has a spring constant of 45 N/m.  What mass would need to be attached to the spring in order for the period of oscillation of the spring to match the period of the pendulum? 




\end{question}

	
\end{shortanswer}





\end{document}


