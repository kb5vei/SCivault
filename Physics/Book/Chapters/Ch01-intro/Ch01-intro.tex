\chapter{Introduction}
\section{Dimensional Analysis and SI units}
The \textbf{SI}\index{SI system of units} system of units is the standard used by many scientists throughout the world.  There are seven \textit{fundamental} or \textit{base} quantities from which all other measurements are derived.  These quantities are listed below:
 \index{Units, Fundamental}

\begin{center}

	
\begin{table}[ht]\caption{\textbf{SI Units}}% title of Table 
	\centering % used for centering table	
	\begin{tabular}{|c|c|c|}
		\hline \hline
		\textbf{Quantity} & \textbf{Unit} & \textbf{Unit Symbol}\\
		\hline
		time & second & s \\
		\hline
		length & meter & m \\
		\hline
		mass & kilogram & kg \\
		\hline
		electrical current & Ampere & A \\
		\hline
		temperature & Kelvin & K \\
		\hline
		amount of substance & mole & mol \\
		\hline
		luminous intensity & candela & cd \\
		\hline		
	\end{tabular}
	\label{table:nonlin}% is used to refer this table in the text
\end{table}
\end{center}

	Of these quantities, mass, time and length are quite common.  Thus, this system is sometimes called the MKS (meter, kilogram, second) system. In order to use any equations, all measurements must have correct units.  For example, if a time is expressed in hours, it must first be converted into seconds before any calculations can be attempted.  
	
	Dimensional analysis \index{Dimensional Analysis} is the process in which the units associated with quantities create \textit{derived units}. \index{Units, Derived} For instance, when a distance is divided by a time, the units will be $ \frac{m}{s}$ (read \textit{meters per second}).  	
	
	Dimensional analysis is an important part of solving physics problems.  Often, correct dimensional analysis can help you determine if a problem has been solved correctly.  One should not even attempt to calculate an answer to a problem until the correct units have been verified. 



		


\section{Vectors and Scalars}
	A \textbf{Scalar} \index{Scalar} is a quantity that has only a \textbf{magnitude} (that is, a number that measures how strong or big it is).  Mass, time, and temperature are all examples of scalars.  

	\textbf{Vectors} \index{Vector} are quantities that have both a \textbf{magnitude} and a \textbf{direction}.  ``Twenty miles north,'' ``two feet left,'' and ``4.415 meters at a 60$^\circ$ angle'' are all examples of vectors.  All of these measurements have directions associated with them. 
	
	There are a variety of ways a vector can be written.  Variables that represent vectors commonly are written with an arrow over them, such as $\vec{a}$ or $\vec{F}$ or in boldface, such as \textbf{a} or \textbf{F}.  
	
	
	
\section{Vector Mathematics}
	\subsection{Vector Addition}
	\subsection{The Dot Product}
	\subsection{The Cross Product}
	
