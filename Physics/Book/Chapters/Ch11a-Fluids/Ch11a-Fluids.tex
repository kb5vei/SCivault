\chapter{Fluids}
	\section{Density}
	\index{Density}
	\index{Fluids}
	The density of an object measures mass per unit volume.  It is calculated using the following formula:
		\begin{mdframed}[backgroundcolor=orange!20!white]
		\begin{equation}
		\rho = \frac{m}{V}
			\label{equation:density}
		\end{equation}
	\end{mdframed}	
	Where $\rho$ is density in $\frac{kg}{m^3}$, $m$ is mass in kg, and $V$ is volume in $m^3$.  The density of an object depends on the material from which it is made.  The density of pure water is $1000 \frac{kg}{m^3}$.
	
	The table below shows the density of some common materials:
	\begin{center}
		\begin{tabular}{|c|c|}
			\hline
			\textbf{Material} & \textbf{Density} ($\frac{kg}{m^3}$) \\
			\hline
			Ethanol &  789 \\
			\hline
			Olive Oil & 929 \\
			\hline
			Water (pure) & 1000 \\
			\hline 
			Water (ocean) & 1025 \\
			\hline
			
			
			
		\end{tabular}
	\end{center}
	
	
	
	\section{Buoyant Force}
	\index{Buoyant Force}
	\index{Force, Buoyant}
			\begin{mdframed}[backgroundcolor=orange!20!white]
		\begin{equation}
			F_b = \rho V g
			\label{equation:buoyantforce}
		\end{equation}
	\end{mdframed}	

	\subsection{Percent Submerged}
	\index{Percent Submerged}
				\begin{mdframed}[backgroundcolor=orange!20!white]
		\begin{equation}
			\% = \frac{\rho_{solid}}{\rho_{liquid}}
			\label{equation:percentsubmerged}
		\end{equation}
	\end{mdframed}
	
	
	
	
	\section{Pressure}
	\index{Pressure}
	
			\begin{mdframed}[backgroundcolor=orange!20!white]
		\begin{equation}
			P = \frac{F}{A}
			\label{equation:pressure}
		\end{equation}
	\end{mdframed}	

		\subsection{Hydrostatic Pressure}
		\index{Pressure, Hydrostatic}

				\begin{mdframed}[backgroundcolor=orange!20!white]
		\begin{equation}
			P = \rho g h
			\label{equation:hydrostaticpressure}
		\end{equation}
	\end{mdframed}	

\section{Continuity Equation}

\section{Bernouli's Equation}

