\chapter{Modern and Atomic Physics}
\section{The Dual Nature of Light} \index{Light, Dual Nature}
\subsection{Young's Double Slit Experiment} \index{Young's Double Slit Experiment}
Young's Double Slit Experiment proved once and for all that light has a wave nature.  
\subsection{The Photoelectric Effect} \index{Photoelectric Effect}
Thus, Einstein's explanation of the photoelectric effect proved once and for all that light has a particle nature.  When combined with the results from Young's Double Slit Experiment, we find that light can act as either a stream of particles or waves, depending on the experiment and the observations that we make.  


\section{The Dual Nature of Matter} \index{Matter, Dual Nature}
\subsection{Mass-Energy Equivalence} \index{Mass-Energy Equivilance}
\subsection{deBroigle Wavelength} \index{deBroigle Wavelength}

\section{Atomic Physics} \index{Atomic Physics}


For hydrogen, these energy levels are given by the following formula:

\begin{mdframed}[backgroundcolor=orange!20!white]
	\begin{equation}
	E_n = \frac{-13.6 \si{eV}}{n^2}  
	\label{eqn:hydrogenenergy}
	\end{equation}
\end{mdframed}

This energy can be thought of as the electrostatic potential energy of the electron-proton configuration in the atom.  In order to return the electron to zero potential energy (an infinite distance away from the proton), one would need to somehow provide 13.6 eV of energy to the electron.  


\subsection{Energy Level Diagrams and Atomic Spectra} \index{Energy Level Diagram}

\begin{figure}[h]

\begin{mdframed}[backgroundcolor=black!5!white]
	\vspace{0.25 in}
\begin{center}

	\begin{tikzpicture}
	\tikz{\draw (0,0) -- (8,0);
		\node[scale=0.7,left] at (0,0) {$n=1$};
		\node[scale=0.7,right] at (8,0) {$E=-13.6$ eV};
		
		\draw (0,10.2) -- (8,10.2);
		\node[scale=0.7,left] at (0,10.2) {$n=2$};
		\node[scale=0.7,right] at (8,10.2) {$E=-3.4$ eV};
		
		\draw (0,12.089) -- (8,12.089);
		\node[scale=0.7,left] at (0,12.089) {$n=3$};
		\node[scale=0.7,right] at (8,12.089) {$E=-1.511$ eV};
		
		\draw (0,12.75) -- (8,12.75);
		\node[scale=0.7,left] at (0,12.75) {$n=4$};
		\node[scale=0.7,right] at (8,12.75) {$E=-0.85$ eV};
		
		\draw (0,13.056) -- (8,13.056);
		\node[scale=0.45,left] at (0,13.056) {$n=5$};
		\node[scale=0.45,right] at (8,13.056) {$E=-0.544$ eV};
		
		\draw (0,13.222) -- (8,13.222);
		\node[scale=0.45,left] at (0,13.222) {$n=6$};
		\node[scale=0.45,right] at (8,13.222) {$E=-0.3778$ eV};
		
				\draw [thick] (0,13.6) -- (8,13.6);
		\node[scale=0.7,left] at (0,13.6) {$n=\infty$};
		\node[scale=0.7,right] at (8,13.6) {$E=0$ eV};
}
		 
	
	\end{tikzpicture}
\end{center}
\end{mdframed}	

\caption{A simple energy level diagram for Hydrogen}
\end{figure}

\index{Atomic Spectra}



