\chapter{Modern and Atomic Physics}
\label{chap:modern}
\section{The Dual Nature of Light} \index{Light, Dual Nature}
\subsection{Young's Double Slit Experiment} \index{Young's Double Slit Experiment}
Young's Double Slit Experiment proved once and for all that light has a wave nature.  
\subsection{The Photoelectric Effect} \index{Photoelectric Effect}
\index{Photoelectric Effect}

When metals are exposed to light (in a vacuum), electrons are ejected from the metal.  While experimenting with the photoelectric effect, the following observations were made:
\begin{itemize}
	\item Increasing the brightness of the light (the amplitude of the electromagnetic wave) causes more electrons to be ejected, but their maximum speed remains the same.
	\item Increasing the frequency of the light (changing the color toward the violet end of the spectrum) while keeping the intensity the same causes faster electrons to be emitted.
\end{itemize}

These observations are inconsistent with the wave model of light.  Albert Einstein proposed understanding the photoelectric effect as a collision between a particle of light, called a \textit{photon},\index{photon} and an electron.  The energy of a photon is given by:
\index{Energy of a Photon}

	\begin{mdframed}[backgroundcolor=orange!20!white]
	\begin{equation}
		E = h f
		\label{eqn:photonenergy}
	\end{equation}
\end{mdframed}

In this equation $f$ is the frequency of the light, and $h$ is Planck's Constant.  Planck's constant is: 
\index{Planck's Constant}

	\begin{mdframed}[backgroundcolor=green!20!white]
	\begin{equation*}
		h = 6.626 \times 10^{34} \si{J \cdot s} = 4.136 \times 10^{-15} \si{eV \cdot s}
		\label{equation:planck}
	\end{equation*}
\end{mdframed}	



	\begin{mdframed}[backgroundcolor=orange!20!white]
	\begin{equation}
		E = h f - \phi
		\label{eqn:Photoelectric}
	\end{equation}
\end{mdframed}


Thus, Einstein's explanation of the photoelectric effect proved once and for all that light has a particle nature.  When combined with the results from Young's Double Slit Experiment, we find that light can act as either a stream of particles or waves, depending on the experiment and the observations that we make.  


\section{The Dual Nature of Matter} \index{Matter, Dual Nature}
\subsection{Mass-Energy Equivalence} \index{Mass-Energy Equivilance}
The masses of the elementary particles are given in the table below:

\begin{table}[h]
	\caption{Mass of Elementary Particles}

\begin{center}
	\begin{tabular}{|c|c|}
		\hline
		\textbf{Particle} & \textbf{Mass} (u) \\
		\hline
		Proton & 1.00727647 \\
		\hline
		Neutron & 1.008665 \\
		\hline
		Electron & 0.00055 \\
		\hline
		
	\end{tabular}
\end{center}
\end{table}

When the mass of elementary particles had been determined, scientists noticed that the masses of various elements did not equal what would be predicted by adding up constituent particles.  For example, a helium atom is made of two protons, two neutrons, and two electrons.  The combined mass of all these particles is 4.03298294 u, but the mass of a hydrogen atom is 4.002603 u - a difference of a bit more than 0.03 u.  This ``missing" mass is called a \textit{mass defect}.  

Albert Einstein proposed that the missing mass was completely destroyed, and released in the form of energy.  In what is often recognized as his most famous equation, Albert Einstein derived the following expression:

\begin{mdframed}[backgroundcolor=orange!20!white]
	\begin{equation}
		E = mc^2
		\label{eqn:energymass}
	\end{equation}
\end{mdframed}

\index{Law of Conservation of Mass-Energy}
Simultaneously, the Law of Conservation of Mass and the Law of Conservation of Energy are now combined into a single, unified Law of Conservation of Mass-Energy.  Simply put, mass-energy equivalence states that mass and energy are the same thing. 






\subsection{deBroigle Wavelength} \index{deBroigle Wavelength}
\index{deBroigle Wavelength}
\index{Matter Waves}

\begin{mdframed}[backgroundcolor=orange!20!white]
	\begin{equation}
		\lambda = \frac{h}{p}
		\label{eqn:debroigle}
	\end{equation}
\end{mdframed}


\section{Atomic Physics} \index{Atomic Physics}


For hydrogen, these energy levels are given by the following formula:

\begin{mdframed}[backgroundcolor=orange!20!white]
	\begin{equation}
	E_n = \frac{-13.6 \si{eV}}{n^2}  
	\label{eqn:hydrogenenergy}
	\end{equation}
\end{mdframed}

This energy can be thought of as the electrostatic potential energy of the electron-proton configuration in the atom.  In order to return the electron to zero potential energy (an infinite distance away from the proton), one would need to somehow provide 13.6 eV of energy to the electron.  


\subsection{Energy Level Diagrams and Atomic Spectra} \index{Energy Level Diagram}

\begin{figure}[h]

\begin{mdframed}[backgroundcolor=black!5!white]
	\vspace{0.25 in}
\begin{center}

	\begin{tikzpicture}
	\tikz{\draw (0,0) -- (8,0);
		\node[scale=0.7,left] at (0,0) {$n=1$};
		\node[scale=0.7,right] at (8,0) {$E=-13.6$ eV};
		
		\draw (0,10.2) -- (8,10.2);
		\node[scale=0.7,left] at (0,10.2) {$n=2$};
		\node[scale=0.7,right] at (8,10.2) {$E=-3.4$ eV};
		
		\draw (0,12.089) -- (8,12.089);
		\node[scale=0.7,left] at (0,12.089) {$n=3$};
		\node[scale=0.7,right] at (8,12.089) {$E=-1.511$ eV};
		
		\draw (0,12.75) -- (8,12.75);
		\node[scale=0.7,left] at (0,12.75) {$n=4$};
		\node[scale=0.7,right] at (8,12.75) {$E=-0.85$ eV};
		
		\draw (0,13.056) -- (8,13.056);
		\node[scale=0.45,left] at (0,13.056) {$n=5$};
		\node[scale=0.45,right] at (8,13.056) {$E=-0.544$ eV};
		
		\draw (0,13.222) -- (8,13.222);
		\node[scale=0.45,left] at (0,13.222) {$n=6$};
		\node[scale=0.45,right] at (8,13.222) {$E=-0.3778$ eV};
		
				\draw [thick] (0,13.6) -- (8,13.6);
		\node[scale=0.7,left] at (0,13.6) {$n=\infty$};
		\node[scale=0.7,right] at (8,13.6) {$E=0$ eV};
}
		 
	
	\end{tikzpicture}
\end{center}
\end{mdframed}	

\caption{A simple energy level diagram for Hydrogen}
\end{figure}

\index{Atomic Spectra}



