\chapter{Heat and Thermodynamics}
\index{Heat}
\index{Thermodynamics}

	\textit{Heat} is thermal energy that is transferred from one object to another.  Most often, heat energy is symbolized by the variable $Q$.  
	\section{Modes of Heat Transfer}
	\index{Heat Transfer. Modes of}
	There are three modes in which heat is transferred from one object to another.   \begin{itemize}
		\item \textit{Conduction} occurs when objects are physically touching each other.  Metals tend to be good conductors of heat. 
		\item   \textit{Convection} occurs when a moving fluid causes heat to be transferred.  Convection currents in fluids can occur naturally, or they can be forced. 
		\item  \textit{Thermal Radiation} (not to be confused with Ionizing Radiation, which will be studied in chapter \ref{chap:nuclear}) occurs when heat is transferred via electromagnetic waves.  While infrared light is most often associated with heat transfer, all electromagnetic rays transfer energy - and that energy often ends up as thermal energy.  Additionally, thermal radiation is the only mode of heat transfer that does not require a medium - it can happen through empty space.
		\end{itemize}
	
	\section{Laws of Thermodynamics}
	\index{Thermodynamics}
	
	There are a total of four laws of thermodynamics.  Each law of thermodynamics focuses on a particular aspect of how heat can be transferred. 
	
	\subsection{0th Law}
		\index{Thermodynamics, 0th Law of}
	
	The 0th law of thermodynamics deals with the idea of thermal equilibrium.  Let us suppose there are three objects: A, B, and C, and each object has a temperature, $T_A$, $T_B$, and $T_C$, respectively.  If $T_A = T_B$ and $T_B = T_C$, then $T_A = T_C$.  
	
	While this may seem to go without saying, it is important to realize the implications of this law.  First, when objects are in thermal equilibrium (at the same temperature), no heat will flow.  Second, heat energy is fungible, meaning that all heat enery is the same and will have the same effect on the temperature of an object, despite the origin or process that delivers that thermal energy to an object.
	
	\subsection{1st Law}
			\index{Thermodynamics, 1st Law of}
			Heat flows from higher temperature to lower temperature.
			$\Delta U = Q + W $ 
			
	\subsection{2nd Law} 
			\index{Thermodynamics, 2nd Law of}
			Entropy Always increases.
			
	\subsection{3rd Law}
			\index{Thermodynamics, 3rd Law of}
			It is impossible to reach absolute zero.
	
	
	
	\section{Specific Heat Capacity}
	\index{Specific Heat Capacity}
	
	Have you ever gone to the beach and found that the sand is extremely hot, while the water is quite cold?  This is an example of a difference in specific heat capacity.  Different materials respond to heat by warming at different rates; it takes more energy to warm 1 kg of water by $1 \si{\degreeCelsius}$ than it takes to warm 1 kg of sand by $1 \si{\degreeCelsius}$.  Thus, the sand tends to warm up faster than the water.  
	
	The amount of heat, $Q$, needed to cause a mass of $m$ to change its temperature by $\Delta T$ is given by the following formula:
	
					\begin{mdframed}[backgroundcolor=orange!20!white]
		\begin{equation}
			Q = m c \Delta T 
			\label{eqn:specificheat}
		\end{equation}
	\end{mdframed}
	The specific heat capacity of the material is $c$.  A list of specific heat capacities for common materials can be found in Appendix \ref{tab:specificheat} on  \cpageref{tab:specificheat}.
	
	
	\begin{mdframed}[backgroundcolor=blue!10!white]
		\begin{center}
			
			
			\textbf{Example \thesection.1}	
		\end{center}
		
		\textbf{Problem: } How much energy is required to heat 2.5 kg of water from $20 \si{\degreeCelsius}$ to $60 \si{\degreeCelsius}$?
				
		\vspace{0.1in}
		
		\textbf{Solution:} The specific heat of water is $4180 \si{\frac{J}{kg \degreeCelsius}} $.  Applying equation \ref{eqn:specificheat} yields:
		
		\begin{equation*}
			Q = m c \Delta T = (\SI{2.5}{kg}) (4180 \si{\frac{J}{kg \degreeCelsius}}) (60 \si{\degreeCelsius} - 20 \si{\degreeCelsius})= \SI{418000}{J} 
		\end{equation*}
	\end{mdframed}
	
	\section{Phase Changes and Latent Heat}
	\subsection{Latent Heat of Fusion}
	\subsection{Latent Heat of Vaporization}
	\section{Heat Engines}
	


