\chapter{Heat and Thermodynamics}
\index{Heat}
\index{Thermodynamics}

	\textit{Heat} is thermal energy that is transferred from one object to another.  Most often, heat energy is symbolized by the variable $Q$.  
	\section{Modes of Heat Transfer}
	\index{Heat Transfer. Modes of}
	There are three modes in which heat is transferred from one object to another.   \begin{itemize}
		\item \textit{Conduction} occurs when objects are physically touching each other.  Metals tend to be good conductors of heat. 
		\item   \textit{Convection} occurs when a moving fluid causes heat to be transferred.  Convection currents in fluids can occur naturally, or they can be forced. 
		\item  \textit{Thermal Radiation} (not to be confused with Ionizing Radiation, which will be studied in chapter \ref{chap:nuclear}) occurs when heat is transferred via electromagnetic waves.  While infrared light is most often associated with heat transfer, all electromagnetic rays transfer energy - and that energy often ends up as thermal energy.  Additionally, thermal radiation is the only mode of heat transfer that does not require a medium - it can happen through empty space.
		\end{itemize}
	
	\section{Laws of Thermodynamics}
	\index{Thermodynamics}
	
	There are a total of four laws of thermodynamics.  Each law of thermodynamics focuses on a particular aspect of how heat can be transferred. 
	
	\subsection{0th Law}
		\index{Thermodynamics, 0th Law of}
	
	The 0th law of thermodynamics deals with the idea of thermal equilibrium.  Let us suppose there are three objects: A, B, and C, and each object has a temperature, $T_A$, $T_B$, and $T_C$, respectively.  If $T_A = T_B$ and $T_B = T_C$, then $T_A = T_C$.  
	
	While this may seem to go without saying, it is important to realize the implications of this law.  First, when objects are in thermal equilibrium (at the same temperature), no heat will flow.  Second, heat energy is fungible, meaning that all heat enery is the same and will have the same effect on the temperature of an object, despite the origin or process that delivers that thermal energy to an object.
	
	\subsection{1st Law}
			\index{Thermodynamics, 1st Law of}
		
		\textbf{The first law of thermodynamics} can be thought of as a restatement of the Law of Conservation of Energy. The change in internal thermal energy of a system ($\Delta U$) will be equal to the amount of energy that entered or left the system.  Energy can enter or leave a system as either work ($W$) or heat ($Q$).  Thus,
		
		\begin{mdframed}[backgroundcolor=orange!20!white]
		\begin{equation}
				\Delta U = Q + W
				\label{eqn:firstlawofthermo}
		\end{equation}
		\end{mdframed}
			
This equation is often printed with various sign conventions that can sometimes make it harder to understand.  Rather than memorizing sign conventions, the author of this book suggest developing an understanding that energy entering the system should increase the internal energy and energy exiting the system should decrease the internal energy.  Positives and negatives can then be chosen based on the situation.  

	
	\subsection{2nd Law} 
			\index{Thermodynamics, 2nd Law of}
			\textbf{The Second Law of Thermodynamics} states that systems tend to become more chaotic over time.  That is, the \textit{entropy} of a closed system will always increase.  
			
			This concept is fairly easy to understand.  The probability of an organized stack of books falling down into a disorganized pile is much greater than the probability of a disorganized pile of books spontaneously arranging themselves into an organized stack.  Thus, the Second Law of Thermodynamics merely states that events that are more probable also tend to be more disorganized.  
			
			There are two important implications of the Second Law of Thermodynamics:
			\begin{enumerate}
				\item Heat always flows from objects of higher temperature to objects of lower temperature.  
				\item The direction of time can be determined by the increase in entropy of closed systems.  
			\end{enumerate}
			
			
			
	\subsection{3rd Law}
			\index{Thermodynamics, 3rd Law of}
 \textbf{The Third Law of Thermodynamics} states that the entropy of a system approaches a constant value as the temperature of a system approaches absolute zero.  
 
 
 
 When combined with the 2nd law of thermodynamics, which states that the entropy of a closed system always increases, this means that the temperature of absolute zero can never be reached.  
	
	
	
	\section{Specific Heat Capacity}
	\index{Specific Heat Capacity}
	
	Have you ever gone to the beach and found that the sand is extremely hot, while the water is quite cold?  This is an example of a difference in specific heat capacity.  Different materials respond to heat by warming at different rates; it takes more energy to warm 1 kg of water by $1 \si{\degreeCelsius}$ than it takes to warm 1 kg of sand by $1 \si{\degreeCelsius}$.  Thus, the sand tends to warm up faster than the water.  
	
	The amount of heat, $Q$, needed to cause a mass of $m$ to change its temperature by $\Delta T$ is given by the following formula:
	
					\begin{mdframed}[backgroundcolor=orange!20!white]
		\begin{equation}
			Q = m c \Delta T 
			\label{eqn:specificheat}
		\end{equation}
	\end{mdframed}
	The specific heat capacity of the material is $c$.  A list of specific heat capacities for common materials can be found in Appendix \ref{tab:specificheat} on  \cpageref{tab:specificheat}.
	
	
	\begin{mdframed}[backgroundcolor=blue!10!white]
		\begin{center}
			
			
			\textbf{Example \thesection.1}	
		\end{center}
		
		\textbf{Problem: } How much energy is required to heat 2.5 kg of water from $20 \si{\degreeCelsius}$ to $60 \si{\degreeCelsius}$?
				
		\vspace{0.1in}
		
		\textbf{Solution:} The specific heat of water is $4180 \si{\frac{J}{kg \degreeCelsius}} $.  Applying equation \ref{eqn:specificheat} yields:
		
		\begin{equation*}
			Q = m c \Delta T = (\SI{2.5}{kg}) (4180 \si{\frac{J}{kg \degreeCelsius}}) (60 \si{\degreeCelsius} - 20 \si{\degreeCelsius})= \SI{418000}{J} 
		\end{equation*}
	\end{mdframed}
	
	\section{Phase Changes and Latent Heat}
	\subsection {The States of Matter}
		There are 4 common states of matter that exist abundantly in the universe:
		\begin{enumerate}
			\item \textbf{Solid} - in which matter has both a definite shape and a definite volume.  Subatomically, atoms are packed together tightly, causing them to be stuck in place.  While they can vibrate, they are unable to move significantly.  
			\item \textbf{Liquid} - in which matter has a definite volume, but not a definite shape.  Atoms are still packed together relatively tightly, but are free to move around. 
			\item \textbf{Gas} - in which matter has no definite shape or size; gasses will expand to fill their container.  Individual atoms are far apart and rarely interact with each other.  
			\item \textbf{Plasma} - in which atoms become ionized.  That is, electrons and nuclei are completely separated from each other.  Like gasses, plasmas have no definite shape or volume.  
		\end{enumerate}
		It is important to note that there are other states of matter, such as Bose-Einstein condensates, that are beyond the scope of this text.  
		
		\begin{figure}
			
			\definecolor{topgreen}{RGB}{242,255,225}
			\definecolor{botgreen}{RGB}{220,253,174}
			\definecolor{bordercol}{RGB}{158,174,125}
			\definecolor{arrowcolor1}{RGB}{74,126,186}
			\definecolor{arrowcolor2}{RGB}{255,126,74}
			\tikzset{
				box/.style={draw=bordercol, rectangle, font=\sffamily, top color=topgreen, bottom color=botgreen, drop shadow, minimum width=2cm, inner ysep=4pt},
				myarr/.style={-{Straight Barb[angle=60:2pt 5]}, arrowcolor1},
				myarr2/.style={-{Straight Barb[angle=60:2pt 5]}, arrowcolor2},
				nodarr/.style={midway, fill=white, anchor=center, text=black, font=\sffamily}
			}
		\centering
		\begin{tikzpicture}[auto, node distance=8cm]
			
			\node[box] (Solid) {Solid};
			\node[box, right = of Solid] (Liquid) {Liquid};
			\node[box, above = of Solid] (Gas) {Gas};
			\node[box, above = of Liquid] (Plasma) {Plasma};
			\node[below left = -.15cm of Plasma] (bp){};
			\node[below right = -.15cm of Gas] (bg){};
			\node[above left = -.15cm of Plasma] (ap){};
			\node[above right = -.15cm of Gas] (ag){};
			\node[below right = -.15cm of Solid] (bs){};
			\node[below left = -.15cm of Liquid] (bl){};
			\node[above right = -.15cm of Solid] (as){};
			\node[above left = -.15cm of Liquid] (al){};
			\node[below left = -.15cm of Gas] (bgl){};
			\node[above left = -.15cm of Solid] (asl){};
			\node[above right = -.15cm of Liquid] (alr){};
			\node[below = -.15cm of Gas] (bgc){};
			
			\draw[myarr2] (ag) -- (ap) node[nodarr] {Ionization};
			\draw[myarr] (bp) -- (bg) node[nodarr] {Recombination};
			
			\draw[myarr2] (as) -- (al) node[nodarr] {Melting};
			\draw[myarr] (bl) -- (bs) node[nodarr] {Freezing};
			
			\draw[myarr2] (alr) -- (ag) node[nodarr] {Vaporization};
			\draw[myarr] (bgl) -- (bl) node[nodarr] {Condensation};
			
			
			\draw[myarr2] (as) -- (bg) node[nodarr] {Sublimation};
			\draw[myarr] (bgl) -- (asl) node[nodarr] {Deposition};
		\end{tikzpicture}
		\caption{The Phase Changes of Matter}
		\end{figure}
	\subsection{Latent Heat of Fusion}
	\subsection{Latent Heat of Vaporization}
	\section{Heat Engines}
	


