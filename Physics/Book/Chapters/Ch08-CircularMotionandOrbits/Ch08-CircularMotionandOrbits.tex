\chapter{Circular Motion and Orbits}
	\section{Centripetal Forces and Accelerations}
	\subsection{Centripetal Force} \index{Force, Centripetal}
	We have already learned that an object in motion will continue to move in a straight line, assuming no forces are acting on the object.  In order for an object to move along a circular path, there must therefore be a force acting on the object to keep it from moving in a straight line.  If you whirl a mass on a string around in a circle, tension in the string keeps the mass from continuing to move in a straight line.  As the Moon orbits the Earth, the gravitational attraction between the Moon and the Earth keeps the Moon in its orbit around the Earth.  Any force that keeps an object moving along a circular path is called a \textbf{Centripetal Force} (Centripetal literally translates from Latin as ``center seeking'').  Any centripetal force can be described as:
	
	\begin{mdframed}[backgroundcolor=orange!20!white]
	\begin{equation}
	F_c = \frac{mv^2}{r}
	\label{equation:centripetalforce}
	\end{equation}
		
	\end{mdframed}
	

	The direction of a centripetal force is always toward the center of the circle.  

	
	\subsection{Centripetal Acceleration} \index{Acceleration, Centripetal}
	
	
	
	When an object moves in a circle, even if its speed remains constant, its velocity is constantly changing due to its constant change in direction of motion.  Thus the object must be constantly accelerating in a direction toward the center of the circle.  Using Equation \ref{equation:centripetalforce} and Newton's Second Law, it is possible to prove that centripetal acceleration is given by:			
	\begin{mdframed}[backgroundcolor=orange!20!white]
	\begin{equation}
	a_c = \frac{v^2}{r}
	\end{equation}
	\end{mdframed}
	Just as Centripetal force is always directed toward the center of the circle, centripetal acceleration is also always directed toward the center of the circle.  
	
	
	\begin{mdframed}[backgroundcolor=blue!10!white]
		\begin{center}
			\textbf{Example \thesubsection}
		\end{center}
		\textbf{Problem:} A children's toy consists of a 0.5kg ball attached to the end of a light 0.3 meter rope.  A child grabs the toy from the end of the rope and swings the ball around in a circle above his head.  What is the tension in the rope? 
		
		\vspace{0.1in}
		
		\textbf{Solution:}
		
		\end{mdframed}
	
	
	\section{Kepler's Laws of Planetary Motion}
	\subsection{Kepler's First Law}
	\subsection{Kepler's Second Law}
	\subsection{Kepler's Third Law}
	
	
	
	
	
	\section{Newton's Law of Universal Gravitation}
	
	\section{Orbital Motion}
	

		


	


