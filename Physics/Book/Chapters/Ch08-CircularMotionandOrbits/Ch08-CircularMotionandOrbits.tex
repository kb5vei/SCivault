\chapter{Circular Motion and Orbits}
	\section{Centripetal Forces and Accelerations}
	\subsection{Centripetal Force} \index{Force, Centripetal}
	We have already learned that an object in motion will continue to move in a straight line, assuming no forces are acting on the object.  In order for an object to move along a circular path, there must therefore be a force acting on the object to keep it from moving in a straight line.  If you whirl a mass on a string around in a circle, tension in the string keeps the mass from continuing to move in a straight line.  As the Moon orbits the Earth, the gravitational attraction between the Moon and the Earth keeps the Moon in its orbit around the Earth.  Any force that keeps an object moving along a circular path is called a \textbf{Centripetal Force} (Centripetal literally translates from Latin as ``center seeking'').  Any centripetal force can be described as:
	
	\begin{mdframed}[backgroundcolor=orange!20!white]
	\begin{equation}
	F_c = \frac{mv^2}{r}
	\label{equation:centripetalforce}
	\end{equation}
		
	\end{mdframed}
	

	The direction of a centripetal force is always toward the center of the circle.  

	
	\subsection{Centripetal Acceleration} \index{Acceleration, Centripetal}
	
	
	
	When an object moves in a circle, even if its speed remains constant, its velocity is constantly changing due to its constant change in direction of motion.  Thus the object must be constantly accelerating in a direction toward the center of the circle.  Using Equation \ref{equation:centripetalforce} and Newton's Second Law, it is possible to prove that centripetal acceleration is given by:			
	\begin{mdframed}[backgroundcolor=orange!20!white]
	\begin{equation}
	a_c = \frac{v^2}{r}
	\end{equation}
	\end{mdframed}
	Just as centripetal force is always directed toward the center of the circle, centripetal acceleration is also always directed toward the center of the circle.  
	
	
	\begin{mdframed}[backgroundcolor=blue!10!white]
		\begin{center}
			\textbf{Example \thesubsection}
		\end{center}
		\textbf{Problem:} A children's toy consists of a 0.5kg ball attached to the end of a light 0.3-meter-long rope.  A child grabs the toy from the end of the rope and swings the ball around in a circle above his head with a tangential speed of 2 m/s.  What is the tension in the rope? 
		
		\vspace{0.1in}
		
		\textbf{Solution:} In order to keep moving in a circle, tension in the rope must act as the centripetal force.  Therefore, the tension in the rope is given by:
	
		\begin{equation*}
			F_T = F_c = \frac{mv^2}{r} = \frac{0.5 kg \cdot (2 m/s )^2}{0.3m} \approx \boxed{6.667 N}
		\end{equation*}
		
		\end{mdframed}
	
	
		\begin{mdframed}[backgroundcolor=blue!10!white]
		\begin{center}
			\textbf{Example \thesubsection.2}
		\end{center}
		\textbf{Problem:} A toy car travels around a loop of diameter 0.3 meters.  What is the minimum speed the car needs to travel in order to make it around the loop?
		
		\vspace{0.1in}
		
		\textbf{Solution:} 
		
		\begin{equation*}
		a
		\end{equation*}
		
	\end{mdframed}
	
	
	
	\section{Kepler's Laws of Planetary Motion} \index{Kepler's Laws of Planetary Motion} \index{Planetary Motion, Kepler's Laws of}
	\subsection{Kepler's First Law}
	    Kepler's first law stathes that the orbit of a planet is an ellipse with the Sun at one of the two foci:
	    


		The Eccentricity of an elipse can be found using: 
		
	
	\subsection{Kepler's Second Law}
		A line segment joining a planet and the Sun sweeps out equal areas during equal intervals of time.
		
		
		
	\subsection{Kepler's Third Law}
			The square of the orbital period of a planet is directly proportional to the cube of the semi-major axis of its orbit.
	
	
	
	
	\section{Newton's Law of Universal Gravitation} \index{Universal Gravitation, Law of}
	
	Newton's Law of Universal Gravitation is a way of calculating the gravitational force between any two objects with mass.  It is given by: 
	\begin{mdframed}[backgroundcolor=orange!20!white]
		\begin{center}
\textbf{Newton's Law of Universal Gravtation}
		\end{center}
		\begin{equation}
		F_g = \frac{G m_1m_2}{r^2}
		\label{equation:UniversalGravitation}
		\end{equation}
	\end{mdframed}
	where G is the Universal Gravitational Constant, ($G = 6.67 \times 10^{-11} \frac{Nm^2}{kg^2}$), $m_1$ and $m_2$ are the masses of the objects, and $r$ is the distance between these objects.   
	
	
	Sometimes, you will see equation \ref{equation:UniversalGravitation} written with a negative sign in order to make it consistent with some of the laws of electrostatics, studied in \color{red} INSERT REFERENCE HERE\color{black}. However, you will likely end up assigning a sign to the force, according to the coordinate system you have defined for the problem. 
	
	
	
	\section{Orbital Motion} \index{Orbits}
	Whenever a object orbits another that has a much larger mass, if the orbit can be approximated by a circle, the gravitational attraction between the two bodies acts as a centripetal force.  Thus, a fundamental realization of orbital mechanics is:
	

		


	


