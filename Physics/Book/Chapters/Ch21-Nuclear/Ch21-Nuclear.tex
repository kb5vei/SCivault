\chapter{Nuclear Physics}
\label{chap:nuclear}

\section{Elements, Isotopes, and Ions} \index{Periodic Table of Elements} \index{Element} \index{Isotope} \index{Ion}
You may remember some of the information contained on the periodic table of elements from chemistry.  A sample box from the periodic table is shown below: 

\begin{figure}[h]
	\begin{center}
	\begin{tabular}{|c| }
		\hline
		\color{red}11\color{black}\\
		\huge{Na}\\
		Sodium\\
		\color{blue}22.989769\color{black}\\
		\hline
	\end{tabular}
\caption{An example of a box from the Periodic Table of Elements.}
	\end{center}
\end{figure}

The number of protons in an atom determines what element it is.  In the example above, the atomic number of sodium is \color{red}11\color{black}.  This means that any atom that has 11 protons is a sodium atom.  The element number is written to the bottom-left of the elements symbol when writing nuclear equations: $_{11}$Na, though this is redundant, since the symbol tells you what element it is.

An atom's isotope is determined by the number of protons and neutrons in that atom.  Collectively, protons and neutrons are called \textit{nucleons}\index{nucleons}.  Thus, increasing the number of neutrons present in an atom will change its isotope without changing its element.  While Sodium only has one stable isotope with 23 nucleons, other elements sometimes have multiple isotopes that are stable.  Thus, when referring to a specific isotopes, we often state the element's name and isotope number, such as Sodium-29 or Carbon-12.  The isotope is often written to the upper-left of the isotope's symbol: $^{23}$Na.    

A neutral atom has the same number of protons and electrons, but not all atoms are electrically neutral.  An atom that has lost or gained one or more electrons is called an ion.  Ions are indicated by placing a charge on the upper-right of the element symbol: Na$^{2+}$.

In nuclear reactions, it is often useful to see element, isotope, and ion information at the same time.  The symbol $^{59}_{27}$Co$^{-}$ would refer to a cobalt-59 atom that has one extra electron.  









\section{Radioactive Decay}
\index{Radioactive Decay}

\subsection{Types of Radioactive Decay}
\subsubsection{Alpha Decay}
\index{Alpha Decay}
Alpha decay is when an atom emits an $\alpha$ (alpha) particle.  \textit{$\alpha$-particles} are really just a helium nucleus.  That is, in alpha decay, a group of two protons and two neutrons spontaneously breaks off of the nucleus.  

\begin{mdframed}[backgroundcolor=blue!10!white]
	\begin{center}
		
		
		\textbf{Example \thesection.1}	
	\end{center}
	
	\textbf{Problem: }A Uranium-238 atom undergoes alpha decay.  
	\begin{enumerate}
		\item Write a formula for this reaction.  
		\item How much energy is released in this reaction?  
	\end{enumerate}
	
	\vspace{0.1in}
	
	\textbf{Solution:} 
	We know that the Uranium-238 atom will break up into an alpha particle and another unknown atom, so we begin by writing the equation with a placeholder element:
	
	\begin{center}
		$^{238}_{92} U \longrightarrow ^{4}_{2}\alpha^{2+} + X$
	\end{center}
	
Applying conservation laws allows us to determine the unknown element's isotope and atomic number.  This element must have a mass-number of 234, as well as 90 protons.  Likewise, since the initial atom was electrically neutral, the final product must have a charge of -2 in order to cancel the positive charge of the alpha particle.  Since Element 90 is Thorium, we can write: 

	\begin{center}
		$\boxed{^{238}_{92} U \longrightarrow ^{4}_{2}\alpha^{2+} + ^{234}_{90}Th^{2-}}$
\end{center}  

In order to determine the amount of energy released in this process, we must find the difference in mass between the two sides of the equation.  

\begin{equation*}
m_{U238} = 238.05078826 \si{u}
\end{equation*}
\begin{equation*}
m_{He4} = 4.002602 \si{u}
\end{equation*}
\begin{equation*}
m_{Th234} = 234.0436 \si{u} 
\end{equation*}

Subtracting the two sides shows: 
\begin{equation*}
\Delta m = m_{U238} - (m_{He4} + m_{Th234}) = 238.05078826 \si{u} - (4.002602 \si{u} + 234.0436 \si{u} ) = 0.00458626 \si{u}
\end{equation*}

We can convert atomic mass units to kilograms since we know $1 \si{u} = 1.6605402 \times 10^{-27} \si{kg}$:
\begin{equation*}
\Delta m =  0.00458626 \si{u} \times \frac{1.6605402 \times 10^{-27} \si{kg}}{1\si{u}} = 7.616 \times 10^{-30} \si{kg}
\end{equation*}

Applying $E = mc^2$ yields:

\begin{equation}
E = m c ^2 = 7.616 \times 10^{-30} \si{kg} \cdot (3 \times 10^8 \si{m/s})^2  = \boxed{6.854\times 10^{-13} \si{J}}
\end{equation}
	
\end{mdframed}




\subsubsection{Beta Decay}
\subsubsection{Gamma Decay}
\index{Gamma Decay} 
\textit{Gamma decay} is a type of radioactive decay that occurs when an atomic nucleus emits a gamma ray - that is, a high-energy photon.  Gamma decay can occur on its own or as a result of other types of radioactive decay, such as alpha or beta decay.

During gamma decay, the atomic nucleus undergoes a transition from a higher energy state to a lower energy state, releasing energy in the form of a gamma ray. Gamma rays are extremely energetic and can penetrate through many materials, including human tissue, making them potentially harmful to living organisms and hard to shield against.  However, they are also useful in many areas, such as medical imaging, radiation therapy, and nuclear physics.

Gamma decay does not change the atomic mass or atomic number of the nucleus, because gamma rays have no mass or charge. However, it can lead to the emission of other particles, such as electrons, as a result of the interaction between the gamma ray and the surrounding matter. Gamma decay is often accompanied by other types of radioactive decay, such as alpha or beta decay, as the unstable nucleus decays toward a more stable configuration.


\subsection{Half-Life}
Half-life represents the time it takes for the number of radioactive atoms in a sample to decrease by half, due to the decay of those atoms, and it is usually denoted with the symbol $t_{1/2}$.  Each isotope has its own specific half-life, some of which can be found in (Insert reference here).  These half-lifes can range from fractions of a second to many billions of years.  Some isotopes with short half-lives are used in medical imaging, while others with long half-lives are used in geology and astrophysics.

The half-life of a radioactive isotope can be used to determine the age of a rock or other geological material by measuring the ratio of the parent isotope to its decay products, since the isotope will decay according to the following formula: 

	\begin{mdframed}[backgroundcolor=orange!20!white]
	\begin{equation}
		A = A_0 e^{\frac{-\ln(2)\cdot t}{t_{1/2}}}
		\label{eqn:halflife}
	\end{equation}
\end{mdframed}



\section{Fission and Fusion}
\subsection{Fission}
\subsection{Fusion}



	


