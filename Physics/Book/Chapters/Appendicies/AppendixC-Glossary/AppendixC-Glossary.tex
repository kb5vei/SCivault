
\newglossaryentry{displacement}{
	name={Displacement},
	description={A vector quantity representing the change in position of an object, with both magnitude and direction}
}

\newglossaryentry{velocity}{
	name={Velocity},
	description={The rate of change of displacement with respect to time, a vector quantity}
}

\newglossaryentry{acceleration}{
	name={Acceleration},
	text={acceleration},
	description={The rate of change of velocity with respect to time}
}

\newglossaryentry{speed}{
	name={Speed},
	description={A scalar quantity representing the distance traveled per unit of time}
}

\newglossaryentry{freefall}{
	name={Free Fall},
	text={free fall},
	description={The motion of an object under the influence of gravitational force only}
}

\newglossaryentry{uniformmotion}{
	name={Uniform Motion},
	description={Motion at a constant speed in a straight line}
}

\newglossaryentry{projectilemotion}{
	name={Projectile Motion},
	description={The motion of an object thrown or projected into the air, subject to only the acceleration of gravity}
}

\newglossaryentry{kinematicequationsa}{
	name={Kinematic Equations},
	description={A set of equations used to describe the motion of an object under constant acceleration}
}

\newglossaryentry{instantaneousvelocity}{
	name={Instantaneous Velocity},
	description={The velocity of an object at a particular moment in time}
}

\newglossaryentry{averagevelocity}{
	name={Average Velocity},
	description={The total displacement divided by the total time taken}
}
\newglossaryentry{fundamentalforces}{
	name={Fundamental Forces},
	text={fundamental forces},
	description={The four basic forces that govern interactions in the universe: gravitational force, electromagnetic force, strong nuclear force, and weak nuclear force. These forces are responsible for the behavior of objects at both macroscopic and subatomic levels}
}

\newglossaryentry{force}{
	name={Force},
	text={Force},
	description={A vector quantity that represents a push or pull on an object, causing it to accelerate, decelerate, or maintain its motion. It is typically measured in newtons (N) and described by both its magnitude and direction}
}

\newglossaryentry{relativemotion}{
	name={Relative Motion},
	description={The calculation of the motion of an object with regard to some other moving object}
}

\newglossaryentry{frameofreference}{
	name={Frame of Reference},
	description={A coordinate system used to represent and measure properties of objects such as position, orientation, and movement}
}

\newglossaryentry{siunits}{
	name={SI System of Units},
	description={The International System of Units, a standardized system of measurement based on seven base units including the meter, kilogram, second, and ampere}
}

\newglossaryentry{dimensionalanalysis}{
	name={Dimensional Analysis},
	description={A mathematical technique used to check the consistency of equations and to derive relationships between physical quantities by analyzing their dimensions}
}

\newglossaryentry{conversionfactor}{
	name={Conversion Factor},
	text={conversion factor},
	description={A ratio used to convert a quantity expressed in one unit to another unit}
}

\newglossaryentry{scalar}{
	name={Scalar},
	text={scalar},
	description={A physical quantity that has only magnitude and no direction, such as temperature or mass}
}

\newglossaryentry{vector}{
	name={Vector},
	description={A physical quantity that has both magnitude and direction, such as velocity or force}
}

\newglossaryentry{magnitude}{
	name={Magnitude},
	text={magnitude},
	description={The size or quantity of a vector, representing its length regardless of direction}
}

\newglossaryentry{unitvector}{
	name={Unit Vector},
	text={Unit vectors},
	description={A vector with a magnitude of one, used to indicate direction without scale}
}

\newglossaryentry{resultant}{
	name={Resultant},
	text={resultant},
	description={A vector that represents the sum of two or more vectors}
}

\newglossaryentry{dotproduct}{
	name={Dot Product},
	text={dot product},
	description={A scalar product of two vectors, equal to the product of their magnitudes and the cosine of the angle between them}
}

\newglossaryentry{crossproduct}{
	name={Cross Product},
	text={cross product},
	description={A vector product of two vectors, resulting in a vector that is perpendicular to both of the original vectors}
}

\newglossaryentry{determinant}{
	name={Determinant},
	text={determinant},
	description={A scalar value that can be computed from the elements of a square matrix and encodes certain properties of the matrix}
}

\newglossaryentry{righthandrulefirst}{
	name={Right Hand Rule, First},
	text={first right hand rule},
	description={A rule used to determine the direction of the cross product of two vectors. Point the index fingers of the right hand in the direction of the first vector and the middle in the direction of the second; the thumb points in the direction of the resultant vector}
}

\newglossaryentry{gravity}{
	name={Acceleration due to Gravity},
	text={acceleration due to gravity},
	description={The acceleration of an object caused by a planet's gravity.  Earth's gravity is typically measured as $\SI{9.81}{m/s^2}$}
}
\newglossaryentry{friction}{
	name={Friction},
	text={Friction},
	description={A resistive force that occurs when two surfaces move or attempt to move across each other. It opposes motion and acts parallel to the surfaces in contact, with two main types: static friction (preventing motion) and kinetic friction (opposing ongoing motion)}
}


\newglossaryentry{freebodydiagram}{
	name={Free Body Diagram},
	text={free body diagram},
	description={A graphical representation used to visualize the forces acting on an object, where the object is typically represented as a point or a simple shape, and the forces are represented as arrows showing both magnitude and direction}
}

\newglossaryentry{distance}{
	name={Distance},
	text={distance},
	description={A scalar quantity that represents the total path length traveled by an object, regardless of direction}
}
\newglossaryentry{slopeinterceptformat}{
	name={Slope-Intercept Format},
	text={slope-intercept format},
	description={A way of expressing the equation of a straight line in the form \( y = mx + b \), where \( m \) is the slope and \( b \) is the y-intercept}
}

\newglossaryentry{slope}{
	name={Slope},
	text={slope},
	description={A measure of the steepness or incline of a line, calculated as the ratio of the vertical change to the horizontal change between two points on the line}
}

\newglossaryentry{yintercept}{
	name={y-Intercept},
	text={y-intercept},
	description={The point where a line crosses the y-axis of a graph, representing the value of \( y \) when \( x \) is zero}
}

\newglossaryentry{jerk}{
	name={Jerk},
	text={jerk},
	description={The rate of change of acceleration with respect to time, often referred to as the third derivative of position with respect to time}
}

\newglossaryentry{energy}{
	name={Energy},
	text={Energy},
	description={The capacity of a system to perform work or produce change, typically measured in joules (J). Energy exists in various forms, including kinetic, potential, thermal, and chemical energy, and can be transferred or transformed but not created or destroyed (according to the law of conservation of energy)}
}
\newglossaryentry{joule}{
	name={Joule},
	text={joule},
	description={The SI unit of energy, represented by the symbol \( J \). One joule is defined as the amount of work done when a force of one newton displaces an object by one meter in the direction of the force, equivalent to \( 1 \, \text{J} = 1 \, \text{N} \cdot \text{m} \)}
}
\newglossaryentry{mechanicalenergy}{
	name={Mechanical Energy},
	text={mechanical energy},
	description={The sum of kinetic and potential energy in a system, representing the energy associated with the motion and position of an object. Mechanical energy is conserved in an isolated system with no non-conservative forces, such as friction}
}
\newglossaryentry{kineticenergy}{
	name={Kinetic Energy},
	text={Kinetic energy},
	description={The energy an object possesses due to its motion, calculated as \( KE = \frac{1}{2}mv^2 \), where \( m \) is the object's mass and \( v \) is its velocity. Kinetic energy depends on both mass and speed, and it increases with the square of the speed}
}

\newglossaryentry{gravitationalpotentialenergy}{
	name={Gravitational Potential Energy},
	text={Gravitational potential energy},
	description={The energy an object possesses due to its position in a gravitational field, commonly calculated as \( PE = mgh \), where \( m \) is the object's mass, \( g \) is the acceleration due to gravity, and \( h \) is the height above a reference point}
}

\newglossaryentry{elasticpotentialenergy}{
	name={Elastic Potential Energy},
	text={elastic potential energy},
	description={The energy stored in an elastic material, such as a spring, when it is stretched or compressed. Elastic potential energy is calculated as \( PE = \frac{1}{2}kx^2 \), where \( k \) is the spring constant and \( x \) is the displacement from the equilibrium position}
}
\newglossaryentry{potentialenergy}{
	name={Potential Energy},
	text={Potential energy},
	description={The stored energy in a system due to the position, arrangement, or state of its components. Potential energy can take various forms, including gravitational potential energy (due to height in a gravitational field) and elastic potential energy (due to stretching or compressing an elastic material)}
}
\newglossaryentry{hookeslaw}{
	name={Hooke's Law},
	text={Hooke's law},
	description={A principle describing the behavior of springs and other elastic materials, stating that the force exerted by a spring is proportional to its displacement from the equilibrium position. Mathematically, \( F = -kx \), where \( F \) is the restoring force, \( k \) is the spring constant, and \( x \) is the displacement}
}

\newglossaryentry{work}{
	name={Work},
	text={Work},
	description={A measure of energy transfer that occurs when an object is moved by a force. Calculated as \( W = Fd \cos \theta \), where \( W \) is the work done, \( F \) is the applied force, \( d \) is the displacement, and \( \theta \) is the angle between the force and displacement directions}
}