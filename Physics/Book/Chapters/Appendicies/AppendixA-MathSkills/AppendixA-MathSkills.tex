



\chapter{Math Skills}

\section{Scientific Notation}


	\begin{itemize}
		\item Scientific Notation always has three parts: the \textit{coefficient}, the \textit{base}, and the \textit{exponent}:
		\begin{center}
			\color{blue} Coefficient $\rightarrow \color{black} 6.022 \times 10^{23 \color{red}\leftarrow}$ \textsuperscript{\color{red}Exponent} \color{black}
			
			\hspace{.7in} \color{orange}$\uparrow$
			
			\hspace{.7in} Base \color{black}
		\end{center}
		\item In scientific notation the \color{orange} base \color{black}is always 10. 
		\item A negative in front of the  \color{blue} coefficient \color{black} means the whole number is negative. 
		\item  A negative \color{red} exponent \color{black} means the number is very small (close to zero). \color{black}
		\item  The \color{red} exponent \color{black} counts how many places the decimal moved, NOT the number of zeroes.	
		\item When comparing numbers in scientific notation, look at (in order): 
		\begin{enumerate}
			\item  Negatives in front of the \color{blue} coefficient. \color{black}
			\item \color{red}Exponents \color{black}
			\item \color{blue}Coefficients \color{black}
		\end{enumerate}
		
		\item To multiply, multiply coeffients, then ADD exponents.
		\item To divide, divide coefficients, then SUBTRACT exponents.
		\item To raise to a power, raise the coefficient to the power, then MULTIPLY exponents.
		\item To enter scientific notation on most calculators use the ``EE" key. $6.022 \times 10^{23}$ is entered as 6.022\scriptsize E\normalsize23.  Calculator notation should \underline{\textbf{never}} be handwritten. 
		\item Metric Prefixes are really just scientific notation:
		\begin{center}
			\begin{tabular}{|c|c|c|}
				\hline
				Prefix & Letter & Power of 10 \\
				\hline
				nano & n &  $ \times 10^{-9}$ \\
				\hline
				micro & $\mu$ &  $ \times 10^{-6}$ \\
				\hline
				milli & m & $ \times 10^{-3}$ \\
				\hline
				centi & c & $ \times 10^{-2}$ \\
				\hline
				deci & d & $ \times 10^{-1}$ \\
				\hline
				Deka & D & $ \times 10^{1}$ \\
				\hline
				Hecto & H & $ \times 10^{2}$ \\
				\hline
				Kilo & k & $ \times 10^{3}$ \\
				\hline
				Mega & M & $ \times 10^{6}$ \\
				\hline
				Giga & G & $ \times 10^{9}$ \\
				\hline
				
			\end{tabular}	
		\end{center}
		
		
	\end{itemize}

\section{Algebra}
\section{Trigonometry}
\section{Arc Length and Radians}