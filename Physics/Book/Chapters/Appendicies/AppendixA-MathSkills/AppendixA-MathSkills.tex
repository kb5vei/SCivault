



\chapter{Math Skills}

\section{Scientific Notation}


	\begin{itemize}
		\item Scientific Notation always has three parts: the \textit{coefficient}, the \textit{base}, and the \textit{exponent}:
		\begin{center}
			\color{blue} Coefficient $\rightarrow \color{black} 6.022 \times 10^{23 \color{red}\leftarrow}$ \textsuperscript{\color{red}Exponent} \color{black}
			
			\hspace{.7in} \color{orange}$\uparrow$
			
			\hspace{.7in} Base \color{black}
		\end{center}
		\item In scientific notation the \color{orange} base \color{black}is always 10. 
		\item A negative in front of the  \color{blue} coefficient \color{black} means the whole number is negative. 
		\item  A negative \color{red} exponent \color{black} means the number is very small (close to zero). \color{black}
		\item  The \color{red} exponent \color{black} counts how many places the decimal moved, NOT the number of zeroes.	
		\item When comparing numbers in scientific notation, look at (in order): 
		\begin{enumerate}
			\item  Negatives in front of the \color{blue} coefficient. \color{black}
			\item \color{red}Exponents \color{black}
			\item \color{blue}Coefficients \color{black}
		\end{enumerate}
		
		\item To multiply, multiply coeffients, then ADD exponents.
		\item To divide, divide coefficients, then SUBTRACT exponents.
		\item To raise to a power, raise the coefficient to the power, then MULTIPLY exponents.
		\item To enter scientific notation on most calculators use the ``EE" key. $6.022 \times 10^{23}$ is entered as 6.022\scriptsize E\normalsize23.  Calculator notation should \underline{\textbf{never}} be handwritten. 
		\item Metric Prefixes are really just scientific notation:
		\begin{center}
			\begin{tabular}{|c|c|c|}
				\hline
				Prefix & Letter & Power of 10 \\
				\hline
				nano & n &  $ \times 10^{-9}$ \\
				\hline
				micro & $\mu$ &  $ \times 10^{-6}$ \\
				\hline
				milli & m & $ \times 10^{-3}$ \\
				\hline
				centi & c & $ \times 10^{-2}$ \\
				\hline
				deci & d & $ \times 10^{-1}$ \\
				\hline
				Deka & D & $ \times 10^{1}$ \\
				\hline
				Hecto & H & $ \times 10^{2}$ \\
				\hline
				Kilo & k & $ \times 10^{3}$ \\
				\hline
				Mega & M & $ \times 10^{6}$ \\
				\hline
				Giga & G & $ \times 10^{9}$ \\
				\hline
				
			\end{tabular}	
		\end{center}
		
		
	\end{itemize}

\section{Algebra}
\section{Trigonometry}

\newpage
\section {Radians and Arc Length}
\subsection{Radians} \index{Radians}

Just like there are many different units that measure distance (meters, feet, inches, miles, furlongs, etc.), there are also different ways of measuring angles.  You are probably already familiar with degrees. A right angle is $90 \degree$, and a full circle is $360 \degree$.  When calculating arc length or using the angular kinematic equations, the standard units for angles are \textit{radians}\footnote{It should be noted that strictly speaking, radians are not a unit; since the definition of a radian has to do with a ratio, all numbers with radians as the unit are actually unitless.}  There are $2 \pi$ radians in a complete circle, so,

	\begin{mdframed}[backgroundcolor=green!20!white]
	\begin{equation*}
		360 \degree = 2 \pi \, \si{radians}
		\label{conversion:radiandegree}
	\end{equation*}
\end{mdframed}	

This equation can be used to convert an angle from radians to degrees or vice-versa.  

 \begin{mdframed}[backgroundcolor=blue!10!white]
	\begin{center}	
		\textbf{Example \thesection.1}	
	\end{center}
	
	\textbf{Problem:} An angle measures $34 \degree$.  What is this angle in radians? 
	
	\vspace{0.2 in} 
	\textbf{Solution}: Begin by creating a conversion factor.  In this case, since we have degrees and want radians, we create a fraction with $2\pi$ radians on top of the fraction, and $360\degree$ on the bottom.  Multiplying by this fraction gives:
	
\begin{equation*}
	34 \degree \times \frac{2 \pi \, \si{rad}}{360 \degree} = \frac{17 \pi}{90}\si{rad} \approx 0.593 \, \si{rad}  
\end{equation*}
	
	It should be noted that the fraction $\frac{2 \pi \, \si{rad}}{360 \degree}$ can easily be reduced to $\frac{ \pi \, \si{rad}}{180 \degree}$ .  Using this as your conversion factor will yield the same results.  It is also often easier and more accurate to leave measurements involving radians in terms of $\pi$.  
	
\end{mdframed}
   
   
    \begin{mdframed}[backgroundcolor=blue!10!white]
   	\begin{center}	
   		\textbf{Example \thesection.2}	
   	\end{center}
   	
   	\textbf{Problem:} An angle measures $\frac{\pi}{2} \si{rad}$.  What is this angle in degrees? 
   	
   	\vspace{0.2 in} 
   	\textbf{Solution}: Again, we begin by creating a conversion factor.  Because we have a measurement in radians and are asked to find degrees, we create a fraction with $360\degree$ on top of the fraction, and  $2\pi$ radians on the bottom:
   	
   	\begin{equation*}
   		\frac{\pi}{2}  \times \frac{360 \degree}{2 \pi \, \si{rad}} = 90 \degree
   	\end{equation*}
 
   	
   \end{mdframed}

\newpage
   \subsection{Arc Length} \index{Arc Length}
   
   The distance along the circumference of a circle that corresponds to an internal angle of the circle is called \textit{Arc Length}.  Though arc-length is a measurement of length, the symbol used for Arc Length is $s$. 
 \begin{figure}[h]
 	  \begin{center}
   	 	\begin{tikzpicture}
   		% the origin
   		\coordinate (O) at (0,0);
   		% the circle and the dot at the origin
   		\draw (O) node[circle,inner sep=1.5pt,fill] {} circle [radius=3cm];
   		% the ``\theta'' arc
   		\draw 
   		(3 cm,0) coordinate (xcoord) -- 
   		node[midway,below] {$r$} (O) -- 
   		(60:3 cm) coordinate (slcoord)
   		pic [draw,->,angle radius=1cm,"$\theta$"] {angle = xcoord--O--slcoord};
   		% the outer ``s'' arc
   		\draw[|-|]
   		(3 cm +10pt,0)
   		arc[start angle=0,end angle=60,radius=3cm+10pt]
   		node[midway,fill=white] {$s$};
   	\end{tikzpicture}
   \end{center}
 \caption{The relationship between arc-length, radius, and angle}
 \end{figure}
   	
  
   	

   
   
   
   
   
   
   
   
   
    Arc Length can be found using the following equation:
   
   	\begin{mdframed}[backgroundcolor=orange!20!white]
   	
   	\begin{equation}
   		s = r \theta
   		\label{equation:arclength}
   	\end{equation}
   \end{mdframed}
where $r$ is the radius of the circle and $\theta$ is the internal angle, measured in radians. Additionally, while meters are the proper SI units for these measurements, this formula will work with any units of length as long as both arc-length, $s$, and radius, $r$ are measured using the same units.  




\newpage

  \begin{mdframed}[backgroundcolor=blue!10!white]
	\begin{center}	
		\textbf{Example \thesection.2}	
	\end{center}
	
	\textbf{Problem:} Find the arc length of $30\degree$ of a circle with a radius of 0.2 meters.  
	 
	
	\vspace{0.2 in} 
	\textbf{Solution}: Begin by drawing a diagram:
	
	
	 	  \begin{center}
		\begin{tikzpicture}
			% the origin
			\coordinate (O) at (0,0);
			% the circle and the dot at the origin
			\draw (O) node[circle,inner sep=1.5pt,fill] {} circle [radius=3cm];
			% the ``\theta'' arc
			\draw 
			(3 cm,0) coordinate (xcoord) -- 
			node[midway,below] {$0.2 \si{m}$} (O) -- 
			(30:3 cm) coordinate (slcoord)
			pic [draw,->,angle radius=1.6cm,"$30 \degree$"] {angle = xcoord--O--slcoord};
			% the outer ``s'' arc
			\draw[|-|]
			(3 cm +10pt,0)
			arc[start angle=0,end angle=30,radius=3cm+10pt]
			node[midway,fill=blue!10!white] {$s$};
		\end{tikzpicture}
	\end{center}

	In order to find arc length, we must first convert the angle from degrees to radians:
	\begin{equation*}
	\theta = 30 \degree \times \frac{2 \pi \si{rad}} {360 \degree} = \frac{\pi}{6} \si{rad}
	\end{equation*}
	
	Now, arc length can be found using equation \ref{equation:arclength}.
	\begin{equation*}
		s = r \theta = (0.2 \si{m})(\frac{\pi}{6} \si{rad}) \approx 0.105 \si{m}
	\end{equation*}
	
	
\end{mdframed}
  