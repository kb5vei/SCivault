\chapter{Optics}
	\index{Optics} Optics is the study of light and how that light interacts with matter.  There are two general fields of Optics: \textit{Geometric Optics} and \textit{Physical Optics}. Geometric optics studies how rays of light travel, while physical optics studies how multiple rays of light interact as waves.  
	
	The speed of light in empty space is exactly:	\index{Speed of Light}
	\begin{mdframed}[backgroundcolor=green!20!white]
		\begin{equation*}
		c = \SI{299792458}{m/s}
		\label{equation:speedoflight}
		\end{equation*}
	\end{mdframed}	
	
	The symbol used is \textit{c} because the speed of light is constant in all frames of reference.  This fact will be discussed in more detail in Insert REFREENCE HERE.
	
	\section{Geometric Optics} \index{Geometric Optics}
	
	\subsection{Refraction} \index{Refraction}
	
	\textit{Refraction} is a phenomenon associated with how light changes direction as it moves from one medium to another.  This is often demonstrated by looking at a pencil in a glass of water, or a fish in a pond.  
	
	
		\subsubsection{Snell's Law} \index{Snell's Law}
		\textit{Snell's Law}, named for the Dutch physicist Willebrord Snell, explains that light always takes the path of least time between two points.  
		
		
					\paragraph{Total Internal Reflection}
		\subsubsection{Lenses}

	\subsection{Reflection}
		\subsubsection{The Law of Reflection}
		\subsubsection{Mirrors}
		
	\section{Physical Optics}
		\subsection{Diffraction}
		\subsection{Young's Double Slit Experiment}
		\subsection{Thin Film Interference}
		
		
		
	

	


