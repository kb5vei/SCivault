\chapter{Work and Energy}
	\section{Work} 
	\index{Work}
	\textbf{Work} is the amount of energy that is transferred into or out of an object due to the application of a force that results in a displacement.  The formula for work is:
		
		\begin{mdframed}[backgroundcolor=orange!20!white]
		\begin{equation}
		W = \vec{F}\cdot\vec{d}  
		\label{eqn:work}
		\end{equation}
	\end{mdframed}
	
	Note that both force and displacement are vectors, and work is the dot product of the two vectors.  Thus, referencing equation \ref{equation:dotproduct}
	\begin{mdframed}[backgroundcolor=orange!20!white]
	\begin{equation}
		W = \vec{F}\cdot\vec{d}  = |\vec{F}| |\vec{d}| cos (\theta)
		\label{eqn:workdotproduct}
	\end{equation}
\end{mdframed}	
	It should also be noted that for work to be done, there must be a non-zero displacement.  No work is done on an object that does not move.  
	
	
	\section{Energy}
	\index{Energy}
	\subsection{Kinetic Energy} \index{Kinetic Energy} \label{Kinetic Energy} \index{Energy, Kinetic}
	\textbf{Kinetic Energy} is energy of motion.  Any object that is moving has kinetic energy.  Kinetic energy cannot be negative.  The formula for kinetic energy is: 
		\begin{mdframed}[backgroundcolor=orange!20!white]
		\begin{equation}
		K = \frac{1}{2}mv^2 
		\label{eqn:kineticenergy}
		\end{equation}
	\end{mdframed}
	
	
	\subsection{Potential energy} \index{Potential Energy} \index{Energy, Potential}
	\textbf{Potential Energy} is energy that is stored.  Though there are many forms of potential energy, We will concentrate on two forms of stored energy in this section: gravitational potential energy and elastic potential energy.  In this text, potential energy is symbolized by the variable $U$, though it is quite common to find potential energy symbolized by the letters $PE$ as well. 
	
	\subsubsection{Gravitational Potential Energy} \index{Gravitational Potential Energy} \index{Potential Energy, Gravitational} \index{Energy, Gravitational Potential}
	\textbf{Gravitational Potential Energy} is energy that is stored due to the interaction of an object with gravitational field.  When a relatively small object is in a uniform gravitational field (such as near the surface of a planet), the gravitational potential energy is given by: 
			\begin{mdframed}[backgroundcolor=orange!20!white]
		\begin{equation}
		U_g = mgh
		\label{eqn:gravitationalpotentialenergy}
		\end{equation}
	\end{mdframed}
	
 
	
	
	
	\subsubsection{Elastic Potential Energy} \index{Elastic Potential Energy} \index{Potential Energy, Elastic}
	
	\index{Potential Energy, Gravitational}
	Springs and other stretchable materials that return to their original shape when not subjected to external forces store \textbf{Elastic Potential Energy}.  For an ideal spring, the force the spring exerts is proportional to the distance it has been stretched.  This is called Hooke's Law: 
	
		\begin{mdframed}[backgroundcolor=orange!20!white]
		\begin{equation}
		\overrightarrow{F_s} = -k\vec{x}
		\label{eqn:hookeslaw}
		\end{equation}
	\end{mdframed}
	
	Some calculus can show that the energy stored in a spring that obeys Hooke's Law is: 
	
	\begin{mdframed}[backgroundcolor=orange!20!white]
		\begin{equation}
		U_s = \frac{1}{2}kx^2
		\label{eqn:elasticpotentialenergy}
		\end{equation}
	\end{mdframed}

	While all real springs convert mechanical energy into heat when stretching or compressing, the amount of energy lost as heat (a process called hysteresis) is usually negligible.  However, some stretchable objects, such as rubber bands, lose significant amounts of energy as heat.  Thus, equations \ref{eqn:hookeslaw} and \ref{eqn:elasticpotentialenergy} are not applicable to these objects.  
	

	
	\section{The Work-Energy Theorem}
	The \textit{Work-Energy Theorem} states that doing work on an object causes that object's energy to change by the same amount as the work done.  This means that an object has 8 Joules of energy, and 2 Joules of work is done on the object, the object will have 10 Joules of work at the end of the process.  	While this is often associated with a change in kinetic energy, the energy change associated with work can also be associated with gravitational potential energy, thermal energy, or any other form of energy.  
	
		\begin{mdframed}[backgroundcolor=orange!20!white]
		\begin{equation}
			W = \Delta E
			\label{equation:workenergy}
		\end{equation}
	\end{mdframed}
	
		\section{Power}
	\index{Power}
	
	
	Power is defined as how quickly work is done.  Power is given by:
	
	\begin{mdframed}[backgroundcolor=orange!20!white]
		\begin{equation}
			P = \frac{W}{t}
			\label{equation:power}
		\end{equation}
	\end{mdframed}
	
	Since the work-energy theorem states that work is a change in energy, power is also used to describe how fast energy is flowing into or out of a system.
	
	\begin{mdframed}[backgroundcolor=orange!20!white]
		\begin{equation}
			P = \frac{\Delta E}{t}
			\label{equation:poweralt}
		\end{equation}
	\end{mdframed}
	

	
	
	\section{The Law of Conservation of Energy}
	\textit{The Law of Conservation of Energy} states that energy cannot be created or destroyed (this isn't entirely true.  This law will be tweaked in chapter \ref{chap:modern}.
	
	\section{Springs}
	
		\begin{mdframed}[backgroundcolor=orange!20!white]
		\begin{equation}
		T_p = 2 \pi \sqrt{\frac{m}{k}}
		\label{eqn:springperiod}
		\end{equation}
	\end{mdframed}
	
	\section{Pendulums}
	A pendulum is any weight on the end of an arm that is free to swing back and forth.  You may have seen pendulums in old-fashioned clocks, and the swings at a park also act like a pendulum.  When a pendulum swings at a small angle ($\theta \lessapprox 5 \deg)$, the period of a pendulum is given by:
	
	\begin{mdframed}[backgroundcolor=orange!20!white]
		\begin{equation}
			T_p = 2 \pi \sqrt{\frac{l}{g}}
			\label{eqn:pendulumperiod}
		\end{equation}
	\end{mdframed}
	
	Notice that the period of a pendulum does not depend on the mass of the bob.  The only two variables that affect its period (assuming a small angle) are the length of the arm and gravity.  
	
	
	 
	

		


	


