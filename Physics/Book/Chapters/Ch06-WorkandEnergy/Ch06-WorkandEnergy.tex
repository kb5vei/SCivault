\chapter{Work and Energy}
	\section{Work}
	
		\begin{mdframed}[backgroundcolor=orange!20!white]
		\begin{equation}
		W = \vec{F}\cdot\vec{d}  
		\label{eqn:work}
		\end{equation}
	\end{mdframed}
	
	\section{Energy}
	\subsection{Kinetic Energy} \index{Kinetic Energy} \label{Kinetic Energy}
		\begin{mdframed}[backgroundcolor=orange!20!white]
		\begin{equation}
		K = \frac{1}{2}mv^2 
		\label{eqn:kineticenergy}
		\end{equation}
	\end{mdframed}
	
	\subsection{Potential energy}
	\subsubsection{Gravitational Potential Energy} \index{Gravitational Potential Energy} \index{Potential Energy, Gravitational}
	
			\begin{mdframed}[backgroundcolor=orange!20!white]
		\begin{equation}
		U_g = mhg
		\label{eqn:gravitationalpotentialenergy}
		\end{equation}
	\end{mdframed}
	
	\subsubsection{Elastic Potential Energy} \index{Elastic Potential Energy} \index{Potential Energy, Elastic}
	
	\index{Potential Energy, Gravitational}
	
		\begin{mdframed}[backgroundcolor=orange!20!white]
		\begin{equation}
		\overrightarrow{F_s} = -k\vec{x}
		\label{eqn:hookeslaw}
		\end{equation}
	\end{mdframed}
	
	
	\begin{mdframed}[backgroundcolor=orange!20!white]
		\begin{equation}
		U_s = \frac{1}{2}kx^2
		\label{eqn:elasticpotentialenergy}
		\end{equation}
	\end{mdframed}

	While all real springs convert mechanical energy into heat when stretching or compressing, the amount of energy lost as heat (a process called hysteresis) is usually negligible.  However, some stretchable objects, such as rubber bands, lose significant amounts of energy as heat.  Thus, equations \ref{eqn:hookeslaw} and \ref{eqn:elasticpotentialenergy} are not applicable to these objects.  
	
	\section{The Work-Energy Theorem}
	The \textit{Work-Energy Theorem} states that doing work on an object causes that object's energy to change by the same amount as the work done.  This means that an object has 8 Joules of energy, and 2 Joules of work is done on the object, the object will have 10 Joules of work at the end of the process.  	While this is often associated with a change in kinetic energy, the energy change associated with work can also be associated with gravitational potential energy, thermal energy, or any other form of energy.  
	
	\section{The Law of Conservation of Energy}
	States that Energy cannot be created or destroyed (this isn't entirely true.  This law will be tweaked in chapter \color{red} INSERT REFERENCE HERE. \color{black}
	 
	

		


	


