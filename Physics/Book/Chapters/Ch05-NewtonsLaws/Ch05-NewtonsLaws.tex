\chapter{Newtons Laws} \index{Newton's Laws of Motion}
	\section{Newton's First Law} \index{Newton's First Law} \index{Law of Inertia}
	\begin{tabular}{p{.75in} p{4.5in} p{.75in}}
		 & \textit{Corpus omne perseverare in statu suo quiescendi vel movendi uniformiter in directum, nisi quatenus a viribus impressis cogitur statum illum mutare.} &  \\
		  & & \\
		 & \textit{Every body continues in its state of being at rest or moving uniformly in a direction, except insofar as it is compelled to change its state by means of an imparted force. } & \\
		 & & \\

		&  -- Newton, Isaac.  \textit{Philosophiae Naturalis Principia Mathematica}.  tr. J. Williamson & \\
		& & \\
	\end{tabular}

	

	
  You may have heard Sir Isaac Newton's first law of physics stated in different ways than the above.  Often in grade school, students are taught a phrase beginning with ``objects in motion...''.  	Sometimes this law is called the ``Law of Inertia''.   This is a very basic understanding of the complexity of this law.  In fact, all non-accelerating systems are governed by this law.  As long as the vector sum of the forces upon an object is zero, the object will continue in a state of uniform motion (remaining at rest is a type of uniform motion) until something causes the equilibrium of the system to be lost.  
	Likewise, if an object is known to have an acceleration of zero, we can state that the vector sum of the forces is equal to zero.  We can use this law to characterize non-accelerating systems:
	
	
	
	\section{Newton's Second Law}
		\begin{tabular}{p{.75in} p{4.5in} p{.75in}}
		& \textit{Mutationem motus proportionalem esse vi motrici impressae, et fieri secundum lineam rectam qua vis illa imprimitur.} &  \\
		& & \\
		& \textit{The change in motion is proportional to the amount of force of motion imparted, and according to the straight line made by the force impressed. } & \\ 
		& & \\
		 & {-Newton, Isaac.  \textit{Philosophiae Naturalis Principia Mathematica}.  tr. J. Williamson} & \\
		 
		
		
	\end{tabular}
	
	\section{Newton's Third Law}
		\begin{tabular}{p{.75in} p{4.5in} p{.75in}}
		&  &  \\
		& & \\
		& \textit{Actioni contrariam semper et aequalem esse reactionem: sive corporum duorum actiones in se mutuo semper esse aequales et in partes contrarias dirigi. } & \\
		& & \\
		& \textit{For an action there is always an equal and opposite reaction: or the two bodies on each other are always equal and in opposite directions. } & \\
		& & \\
		 & {-Newton, Isaac.  \textit{Philosophiae Naturalis Principia Mathematica}.  tr. J. Williamson} & \\
		
	\end{tabular}
	
	\section{Applications of Newton's Laws}
		\subsection{Friction}
		\subsection{Elevators}
		\subsection{Pulleys}
		


	


