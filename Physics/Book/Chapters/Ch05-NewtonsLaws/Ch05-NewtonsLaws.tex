\chapter{Newtons Laws}
	\section{Newton's First Law}
	\begin{tabular}{p{1in} p{4in} p{1in}}
		 & \textit{Corpus omne perseverare in statu suo quiescendi vel movendi uniformiter in directum, nisi quatenus a viribus impressis cogitur statum illum mutare.\footnote{Newton, Isaac.  Philosophiæ Naturalis Principia Mathematica.  tr. J. Williamson}} &  \\
		  & & \\
		 & \textit{Every body continues in its state of being at rest or moving uniformly in a direction, except insofar as it is compelled to change its state by means of an imparted force. } &
		 
	\end{tabular}
	

	
	You may have heard Sir Isaac Newton's first law of physics stated in different ways than the above.  Often in grade school, students are taught a phrase beginning with “objects in motion...”.  This is a very basic understanding of the complexity of this law.  In fact, all non-accelerating systems are governed by this law.  As long as the vector sum of the forces upon an object is zero, the object will continue in a state of uniform motion (remaining at rest is a type of uniform motion) until something causes the equilibrium of the system to be lost.  
	Likewise, if an object is known to have an acceleration of zero, we can state that the vector sum of the forces is equal to zero.  We can use this law to characterize non-accelerating systems:
	
	
	
	\section{Newton's Second Law}
		\begin{tabular}{p{1in} p{4in} p{1in}}
		& \textit{Mutationem motus proportionalem esse vi motrici impressæ, et fieri secundum lineam rectam qua vis illa imprimitur.\footnote{Newton, Isaac.  Philosophiæ Naturalis Principia Mathematica.  tr. J. Williamson}} &  \\
		& & \\
		& \textit{The change in motion is proportional to the amount of force of motion imparted, and according to the straight line made by the force impressed. } &
		
	\end{tabular}
	
	\section{Newton's Thrid Law}
		\begin{tabular}{p{1in} p{4in} p{1in}}
		& \textit{Insert Third Law Here\footnote{Newton, Isaac.  Philosophiæ Naturalis Principia Mathematica.  tr. J. Williamson}} &  \\
		& & \\
		& \textit{Insert Third Law Here } &
		
	\end{tabular}
	
	\section{Applications of Newton's Laws}
		\subsection{Friction}
		\subsection{Elevators}
		\subsection{Pulleys}
		


	


