\chapter{Magnetic Forces and Fields} \label{chap:MagneticForcesandFields}
	\section{Introduction to Magnetism} \index{Magnetism}
		All magnetic fields are created by moving electrically charged particles.  Since all atoms contain protons that can rotate or oscillate, as well as electrons that can move around outside the nucleus, all atoms have magnetic fields.  Most of the time, the effects of many atoms cancel out due to their random orientations.  However, sometimes atoms in areas can align, creating permanent magnetic properties in an object. 

	\section{Types of Magnetism} 
		\subsection{Permanent Magnetism}
		Permanent magnetism occurs when an object exhibits magnetic properties without any external voltages or currents being applied.  There are three basic types of permanent magnetism: 
			\subsubsection{Ferromagnetism} \index{Ferromagnetism} Ferromagnetism is the type of magnetism most of us are familiar with.  It occurs in iron, nickle, cobalt, and some types of rare-earth metals, such as neodymium.  Ferromagnetic materials are strongly attracted to external magnetic fields.  In addition, ferromagnetic materials become magnetized themselves when exposed to an external magnetic field.  That magnetization can remain even when the external magnetic field is removed.  
			
			
			\subsubsection{Paramagnetism} \index{Paramagnetism} Paramagnetic materials are weakly attracted to magnets, but do not become magnetized themselves.  Examples of paramagnetic materials include aluminum, tungsten, and platinum.  

			\subsubsection{Dimagnetism} \index{Diamagnetism} Diamagnetic materials have a weak and opposite response to magnetic fields, meaning they are repelled by a magnet. Examples of diamagnetic materials include copper, silver, and gold.
			
			
		\subsection{Electromagnetism}
	\section{Magnetic Force on a Charged Particle}
	\section{Magnetic Force on a Current-Carrying Wire}
	\section{Magnetic Field Produced by a Current-Carrying Wire}
	
	

		
	
	
	

	


