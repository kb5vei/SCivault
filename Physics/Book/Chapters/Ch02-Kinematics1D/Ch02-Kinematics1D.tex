\chapter{Kinematics in One Dimension}
\section{Distance and Displacement} \index{Distance}

You are probably already familiar with the concept of \textbf{distance} - you might get in your car and drive a total of 1.2 miles to school, turning right after 0.45 miles, according to your car's odometer.  Distance is a scalar that tells you how far something traveled.  The symbol d usually represents distance.

While you may have traveled a total distance of 1.2 miles from your school, you are significantly less than 1.2 miles away from home; in fact, you are approximately 0.874 miles from home, following a direct path directly from your home to the school, at an angle of $59^\circ $(not worrying that this path might take you through someone's back yard or kitchen).


  \textbf{Displacement} \index{Displacement} is a vector that tell you how far something is from the origin, and is independent of the path taken to get there.  The displacement vector is commonly symbolized by $\vec{r}$ though sometimes it may be written as $\vec{d}$ or $\vec{x}$. 

\section{Average and Instantaneous Speed and Velocity}

\textbf{Speed} is a scalar value that represents the change in distance per change in time of an object.  Speed is usually represented with the symbol $v$, without the vector sign.  You are probably already familiar with this quantity, since the speedometer on your family car measures speed.  For the purposes of physics, speed has little value because it is a scalar that tells us nothing of direction.  Much more useful is the concept called velocity.  Velocity and speed are related much like distance and displacement.  

\textbf{Velocity} is the change in displacement of an object per unit time, and as such is a vector.  Positive velocities indicate that the object is moving forward, relative to the axis in question, and negative velocities generally mean that the object is moving backward, relative the the axis.  The average velocity of an object is given by:
\begin{mdframed}[backgroundcolor=orange!20!white]
	\begin{equation}
	\overrightarrow{v_{avg}} = \frac{\Delta\vec{r}}{\Delta t} 
	\end{equation}
\end{mdframed}

Average velocity is useful if an object's velocity is not changing.  However, many times it is more useful to talk about instantaneous velocity.  Instantaneous velocity tells us how fast an object is moving at a given instant in time.  In order to calculate instantaneous velocity, we must allow our time interval in the above formula to become infinitesimally small.  In this case, a little calculus proves:
\begin{mdframed}[backgroundcolor=orange!20!white]
	\begin{equation}
	\vec{v} = \frac{d\vec{r}}{dt}
	\end{equation}
\end{mdframed}	
	
	Calculation of average velocity is rather straightforward, assuming you know both distance traveled and the time it took.  If an object is not speeding up or slowing down during a specific time interval, the instantaneous velocity at any time during this interval is equal to the average velocity.  If the object does speed up or slow down during the time interval in question, the average velocity and the instantaneous velocity at a certain time during the interval are not necessarily the same. 
	
\begin{mdframed}[backgroundcolor=blue!10!white]
		\begin{center}


		\textbf{Example \thesection.1}	
	\end{center}

\textbf{Problem: }You ride your bicycle in a straight line for a distance of 73 meters in 12.5 second.  What is your average speed?
\vspace{0.1in}

\textbf{Solution:} 
Begin by drawing a diagram:


\begin{equation*}
 v_{avg}  = \frac{d}{t} = \frac{73m \hspace{0.05in} \hat{i}}{12.5s}  = 5.84 \frac{m}{s} \hspace{0.05in} \hat{i}
\end{equation*}

\end{mdframed}
	\vspace{0.1in}
	
\begin{mdframed}[backgroundcolor=blue!10!white]
	\begin{center}
		
		
		\textbf{Example \thesection.2}	
	\end{center}
	
	\textbf{Problem: }A bicyclist rides his bike to the east.  His position (in meters) is given by the following expression:
	\begin{equation*}
	\vec{r} = (0.5 t^2 + 4t) \hat{i}
	\end{equation*}
	\begin{enumerate}
		\item What is his average velocity from t = 0 to t = 5 seconds?
		\item What is his instantaneous velocity at t=3 seconds?
	\end{enumerate}
	\vspace{0.1in}
	\textbf{Solution:} \begin{enumerate}
		\item The total displacement (in meters) after five seconds is given by: 
		\begin{equation*}
		\hat{r} = (0.5 \times (5s)^2+4\times 5s) \hspace{0.05in} m \hspace{0.05in} \hat{i} = 32.5 \hspace{0.05in} m \hspace{0.05in} \hspace{0.05in} \hat{i}
		\end{equation*}
		Thus, the average velocity is - 
	\begin{equation*}
	\overrightarrow{v_{avg}}  = \frac{\vec{d}}{t} = \frac{32.5 m \hspace{0.05in} \hat{i}}{5s}  = \boxed{6.5 \frac{m}{s} \hspace{0.05in} \hat{i}}
	\end{equation*}
	
	
	
	\item The instantaneous velocity of an object is found using a derivative with respect to time.  Thus,
	
	\begin{equation*}
	\vec{v} = \frac{d\vec{r}}{dt} = \frac{d}{dt} (0.5 t^2 + 4t) \hat{i} = (t + 4) \hat{i} 
	\end{equation*}
	
	Evaluating this at t=3s yields:
	
	\begin{equation*}
	\vec{v} = (3 + 4) \hat{i} = 7 \frac{m}{s} \hat{i}
	\end{equation*}
	\end{enumerate}
	
\end{mdframed}

	
	
	
	

\section{Relative Motion at Constant Velocity}


\section{Acceleration}
\subsection{Average Acceleration}
Velocity is not always constant.  For instance, when you are driving through a city, there are times when you might be going 30 mph, and there are times when you might be stopped at a streetlight.  City driving requires you to speed up at some times, and slow down at other times – your velocity changes as a function of time.  Change in velocity per change in time is called acceleration.  
Keep in mind, both speeding up and slowing down are forms of acceleration.  In the case that an object is traveling with a positive velocity, slowing down causes negative acceleration (or deceleration).
Like velocity, acceleration comes in two basic types – average and instantaneous.  To find average velocity, we calculate the change in velocity per change in time:

\begin{mdframed}[backgroundcolor=orange!20!white]
	\begin{equation}
	\vec{a} = \frac{\Delta \vec{v}}{\Delta t} 
	\end{equation}
\end{mdframed}	
Keep in mind that this can be expressed as:
\begin{mdframed}[backgroundcolor=orange!20!white]
	\begin{equation}
	\vec{a} = \frac{\overrightarrow{v_f} - \overrightarrow {v_i}}{\Delta t}
	\end{equation}
\end{mdframed}	


where $v_f$ and $v_i$ are final velocity and initial velocity, respectively.  Sometimes final velocity is expressed as $v$ and initial velocity is symbolized as $v_0$. 

\subsection{Instantaneous Acceleration}



\section{The Kinematic Equations}
\section{Vertical Motion and Gravity}


