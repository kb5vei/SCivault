\documentclass[letterpaper, 12pt]{article}
\usepackage[top=2cm,bottom=1cm,left=0.75in,right=0.75in,headheight=17pt, % as per the warning by fancyhdr
includehead,includefoot,
heightrounded, % to avoid spurious underfull messages
]{geometry}
\addtolength{\topmargin}{-.25in}
\usepackage{fancyhdr}
\pagestyle{fancy}
\usepackage{graphicx}
\usepackage{lastpage}
\usepackage{gensymb}

\begin{document}
\fancyhead[l]{	\includegraphics[height=1.2cm]{{"../../Logos/VersionC"}.png} Name:}
\fancyhead[r]{Date \hspace{ 1in}}
\cfoot{\thepage\ of \pageref{LastPage}}
	

	
\begin{center}Newton's Laws of Quiz
\end{center}

\textbf{Multiple Choice} \textit{ Choose the best answer for each of the questions.}
\begin{enumerate}
	\item You are in an airplane when the engines fail.  Which of the following will likely happen?
		\begin{enumerate}
			\item Your plane will immediately fall out of the sky.
			\item Your plane will continue to glide as it slowly loses velocity and altitude.
			\item Your plane will go into a dive immediately.
			\item Your plane will catch fire and burn up.  
		\end{enumerate}

	
	\item A net external force of 50 Newtons acts on a 10-kilogram object.  The object must be - 
		\begin{enumerate}
			\item accelerating 
			\item moving at a constant speed
			\item at rest
			\item moving backward.
			\item it is impossible to tell.
			
		\end{enumerate}


	\item Billy-Bob is hit by a train.  Which of the following forces is greatest? 
	\begin{enumerate}
		\item The force the train puts on Billy-Bob.
		\item The force that Billy-Bob puts on the train.
		\item Both forces are equal.
		\item it is impossible to tell.
	
	\end{enumerate}


	\item Friction is best defined as - 
			\begin{enumerate}
			\item Heat.
			\item A force that causes objects to slow down.
			\item The destruction of energy.
			\item the process by which objects return to their natural resting state.
			
		\end{enumerate}
	
	\item Sebastian is in space.  Which of the following is true?
\begin{enumerate}
	\item He has weight but no mass.
	\item He has mass but not weight.
	\item He has neither weight nor mass.
	\item He has both weight and mass.
	
\end{enumerate}



\end{enumerate}
		
		\vspace{1.2in}
		
\textbf{True or False} \textit{Mark each question as either TRUE or FALSE}
\begin{enumerate}
	\item [\rule{0.4in}{0.01in} 1.]When you kick a ball in an open field, there is a force that keeps the ball moving.
    \item [\rule{0.4in}{0.01in} 2.]When a man is walking forward, the frictional force acting on the man is always backward.
    \item [\rule{0.4in}{0.01in} 3.]If an object is at rest, there must be no forces acting on the object.
    \item [\rule{0.4in}{0.01in} 4.]The force of gravity on earth is 9.81 m/s\textsuperscript{2}
    \item [\rule{0.4in}{0.01in} 5.]A rocket takes off because its exhaust gasses push against the ground. 
    \item [\rule{0.4in}{0.01in} 6.]Mike the alien has a mass of 50kg on his planet, but his planet's gravity is only half as strong as the Earth's.  On Earth, he will have a mass of 100 kg.
    \item [\rule{0.4in}{0.01in} 7.] If there is no force acting on an object, the object must be at rest.
    \item [\rule{0.4in}{0.01in} 8.]An unbalanced, external force will always cause an object to speed up or slow down.
    \item [\rule{0.4in}{0.01in} 9.]The normal force is always equal to the gravitational force.
    \item [\rule{0.3in}{0.01in} 10.]Mass and weight measure the same thing, just in different units.
    
    
    
\end{enumerate}

\vspace{2in}
\hrule
\vspace{0.5in}
Total Points Earned: \rule{1in}{0.01in}

\vspace{1in}
Corrected by: \rule{3in}{0.01in}



 



\end{document}
