\documentclass[letterpaper, 12pt]{article}
\usepackage[top=2cm,bottom=1cm,left=0.75in,right=0.75in,headheight=17pt, % as per the warning by fancyhdr
includehead,includefoot,
heightrounded, % to avoid spurious underfull messages
]{geometry}
\addtolength{\topmargin}{-.25in}
\usepackage{fancyhdr}
\pagestyle{fancy}
\usepackage{graphicx}
\usepackage{lastpage}
\usepackage{gensymb}

\begin{document}
\fancyhead[l]{	\includegraphics[height=1.2cm]{{"../Logos/VersionC"}.png} Name:}
\fancyhead[r]{REFERENCE MATERIAL}
\cfoot{\thepage\ of \pageref{LastPage}}
	


\begin{center}Things to Memorize: Motion in One Dimension
\end{center}

\subsection*{Vectors and Scalars}
\begin{itemize}
	\item \textbf{Magnitude} is a number that measures how big or strong something is.
	\item A \textbf{vector} has both magnitude and direction.
	\item A \textbf{scalar} has magnitude only (no direction).	
	\item Vectors are written with lines over them ($ \vec{A} $).  Scalars are not ($A$).
\end{itemize}

\subsection*{Speed and Velocity}
\begin{itemize}
	\item \textbf{Distance} ($d$) is a scalar that tells you how far something moved.
	\item \textbf{Displacement} ($\vec{d} $) is a vector that tells you how far it is from where something started to where it ended up, regardless of its path. 
	\item \textbf{Speed} ($v$) is a scalar that tells you how fast something is going.
	\item \textbf{Velocity} ($\vec{v}$) is a vector that tells you how fast something is going and in what direction.
	\item Speed and velocity tell you how far an object travels in one second. 
	
	
\end{itemize}

	
\subsection*{Frames of Reference and Relative Motion}
\begin{itemize}
	\item Relative motion problems can be solved by changing your frame of reference:
		\begin{enumerate} 
			\item Instead of seeing the problem from a 3rd person point of view, put yourself in the situation.
				\begin{itemize}
					\item Velocities that are directed in opposite directions in the 3rd person point of view will add.
					\item Velocities that are in the same direction in the 3rd person point of view will subtract.
				\end{itemize}
			\item Calculate the time in the 1st person point of view.
			\item Use the time to calculate distances in the 3rd person point of view.
			
		\end{enumerate}
	\item Relative motion problems can be solved by graphing.
	\item Relative motion problems can be solved by solving a system of equations.
	
	
\end{itemize}

	
\subsection*{Acceleration}
\begin{itemize}
	\item \textbf{Acceleration} tells you how much an object's speed changes in one second.
	\item When an object speeds up, its acceleration is in the same direction as its motion.
	\item When an object slows down, its acceleration is in the direction  opposite to its motion.
	\item \textbf{Average speed} ($v_{avg}$) and \textbf{average velocity} ($\vec{v_{avg}}$) tell how fast something was moving during a period of time.
	\item \textbf{Instantaneous speed} ($v$) and \textbf{instantaneous velocity} ($\vec{v}$) tell you how fast something is moving at a specific time.
	
\end{itemize}

\subsection*{The Kinematic Equations}
\begin{itemize}
	\item There are 5 kinematic variables and 4 kinematic equations.  If you know 3 of the variables, you can find the other 2.  Which makes for 1 happy physics student.\footnote{ [Flipping Physics].  (2015, March 2) \textit{AP Physics 1: Kinematics Review} [Video File] retrieved from https://www.youtube.com/watch?v=8G1oc5Qq90U }
	\item To solve an algebraic kinematic equation: 
		\begin{enumerate}
			\item Define a positive direction.  Label that direction clearly with an arrow: $\longrightarrow + $
			\item Indicate in words what portion of motion your are considering, (like ``motion from launch to the peak of the flight.”)
			\item Fill out a chart, including signs and units, of the five kinematics variables:
			\begin{center}
			\begin{tabular} {| c | c | }
				\hline
				$d$ or $\Delta x$  & \hspace{0.4in} \\
				\hline
				$v_i$ or $v_0$ & \hspace{0.4in} \\
				\hline
				$v_f$ or $v $ & \hspace{0.4in} \\
				\hline
				$a$ & \hspace{0.4in} \\
				\hline
				$t$ & \hspace{0.4in} \\
				\hline
			\end{tabular} 
			\end{center}
		\item Pick an equation that has only \textbf{ONE} unknown variable.  
		\item Manipulate the equation to isolate the unknown variable (if needed).
		\item Plug in the numbers.
		\item Write your final answer with units. 
			
		\end{enumerate}
	
\end{itemize}
\vspace{1in}



\subsection*{Vertical Motion}
\begin{itemize}
	\item An object is in \textbf{free fall} when gravity is the only force acting on it.  
	\begin{itemize}
		\item Objects that are falling under the influence of gravity are in free fall.
		\item Objects that are \textit{rising} can be in free fall if the only force on them is gravity.
	\end{itemize}
	\item The acceleration of objects in free-fall is $g$.  
		\begin{itemize} 
			\item On earth $g_{earth} = 9.81 m/s^2$ 	
			\item Other planets, moons, asteroids, comets, etc. have their own gravity.  Don't use $g_{earth}$ for them. 
		\end{itemize}
	\item If an object lands at the same height it was launched from, the rising time is equal to the falling time. 
	
\end{itemize}
 



\end{document}
