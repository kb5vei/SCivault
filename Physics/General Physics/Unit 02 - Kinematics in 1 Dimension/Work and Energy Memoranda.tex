\documentclass[letterpaper, 12pt]{article}
\usepackage[top=2cm,bottom=1cm,left=0.75in,right=0.75in,headheight=17pt, % as per the warning by fancyhdr
includehead,includefoot,
heightrounded, % to avoid spurious underfull messages
]{geometry}
\addtolength{\topmargin}{-.25in}
\usepackage{fancyhdr}
\pagestyle{fancy}
\usepackage{graphicx}
\usepackage{lastpage}
\usepackage{gensymb}

\begin{document}
\fancyhead[l]{	\includegraphics[height=1.2cm]{{"../Logos/VersionC"}.png} Name:}
\fancyhead[r]{Date \hspace{ 1in}}
\cfoot{\thepage\ of \pageref{LastPage}}
	


\begin{center}Things to Memorize: Motion in One Dimension
\end{center}

\subsection*{Vectors and Scalars}
\begin{itemize}
	\item \textbf{Magnitude} is a number that measures how big or strong something is.
	\item A \textbf{vector} has both magnitude and direction.
	\item A \textbf{scalar} has magnitude only (no direction).	
	\item Vectors are written with lines over them ($ \vec{A} $).  Scalars are not ($A$).
\end{itemize}

\subsection*{Speed and Velocity}
\begin{itemize}
	\item \textbf{Distance} ($d$) is a scalar that tells you how far something moved.
	\item \textbf{Displacement} ($\vec{d} $) is a vector that tells you how far it is from where something started to where it ended up, regardless of its path. 
	\item \textbf{Speed} is a scalar that tells you how fast something is going.
	\item \textbf{Velocity} is a vector that tells you how fast something is going and in what direction.
	\item Speed and velocity tell you how far an object travels in one second. 
	
	
\end{itemize}

	
\subsection*{Frames of Reference and Relative Motion}
\begin{itemize}
	\item Relative motion problems can be solved by changing your frame of reference:
		\begin{enumerate} 
			\item Instead of seeing the problem from a 3rd person point of view, put yourself in the situation.
				\begin{itemize}
					\item Velocities that are directed in opposite directions in the 3rd person point of view will add.
					\item Velocities that are in the same direction in the 3rd person point of view will subtract.
				\end{itemize}
			\item Calculate the time in the 1st person point of view.
			\item Use the time to calculate distances in the 3rd person point of view.
			
		\end{enumerate}
	\item Relative motion problems can be solved by graphing.
	\item Relative motion problems can be solved by solving a system of equations.
	
	
\end{itemize}

	
\subsection*{Acceleration}
\begin{itemize}
	\item \textbf{Acceleration} tells you how much an object's speed changes in one second.
	\item When an object speeds up, its acceleration is in the direction of its motion.
	\item When an object slows down, its acceleration is opposite the direction of motion.
	\item \textbf{Average speed} and \textbf{average velocity} tell how something was moving during a period of time.
	\item \textbf{Instantaneous speed} and \textbf{instantaneous velocity} tell you how fast something is moving at a specific time.
	
\end{itemize}

\subsection*{The Kinematic Equations}
\begin{itemize}
	\item florb
	
\end{itemize}

\subsection*{Vertical Motion}
\begin{itemize}
	\item brolf
	
\end{itemize}
 



\end{document}
