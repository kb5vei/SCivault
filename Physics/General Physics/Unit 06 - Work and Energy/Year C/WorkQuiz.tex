\documentclass[letterpaper, 12pt]{article}
\usepackage[top=2cm,bottom=1cm,left=0.75in,right=0.75in,headheight=17pt, % as per the warning by fancyhdr
includehead,includefoot,
heightrounded, % to avoid spurious underfull messages
]{geometry}
\addtolength{\topmargin}{-.25in}
\usepackage{fancyhdr}
\pagestyle{fancy}
\usepackage{graphicx}
\usepackage{lastpage}
\usepackage{gensymb}

\begin{document}
\fancyhead[l]{	\includegraphics[height=1.2cm]{{"../../Logos/VersionC"}.png} Name:}
\fancyhead[r]{Date \hspace{ 1in}}
\cfoot{\thepage\ of \pageref{LastPage}}
	

	
\begin{center}Work and Energy Quiz
\end{center}

\textbf{Multiple Choice} \textit{ Choose the best answer for each of the questions.}
\begin{enumerate}
	\item You push on a wall for a long time.  Even though you feel tired, your physics teacher states that you have done no work. He states this because - 
		\begin{enumerate}
				\vspace{-.1 in}
			\item The distance the wall moved is zero.
			\item The force applied is zero.
			\item cos $ \theta $ is zero.
			\item He is wrong.
		\end{enumerate}

	
	\item In which of the following situations is no work done?
		\begin{enumerate}
			\vspace{-.1 in}
			\item You push a box across a floor
			\item You lift an orange.
			\item You apply a centripetal force to a ball on a rope.
			\item You push a heavy couch up a ramp into a moving van.

			
		\end{enumerate}


	\item A kid pushes a chair at a constant speed with a force of 30 N.  The chair moves a distance of 2 meters.  How much work was done?
			\begin{enumerate}
					\vspace{-.1 in}
			\item 0 J
			\item 15 J
			\item 30 J
			\item 60 J
			
		\end{enumerate}
	
	\item A traveler is pulling a suitcase along a horizontal floor by directing a force 35$\degree $ above horizontal.  If the traveler uses a for of 70N and pulls the briefcase a distance of 45m, how much work did he do?
\begin{enumerate}
		\vspace{-.1 in}
	\item 2450 J
	\item 2580.329 J
	\item 3150 J
	\item 110250 J
	
\end{enumerate}


	\item A machine does 400J of work while moving a heavy box 5m using a force of 100N.  What is the angle between the force applied and the distance?
	\begin{enumerate} 
			\vspace{-.1 in}
		\item  $0 \degree $
		\item  $ 0.644 \degree $
		\item  $36.87 \degree $
		\item  $90 \degree $
		
	\end{enumerate}
	

\end{enumerate}
	
	\vspace{1 in} 		
	\textbf{True or False} \textit{Mark each question as either TRUE or FALSE}
	\begin{enumerate}
		\item [\rule{0.4in}{0.01in} 1.] Doing work on an object causes that object's energy to change.
		\item [\rule{0.4in}{0.01in} 2.] All potential energy is due to gravity.
		\item [\rule{0.4in}{0.01in} 3.] Work and energy have the same units.
		\item [\rule{0.4in}{0.01in} 4.] Force and work are the same thing, just with different units.
		\item [\rule{0.4in}{0.01in} 5.] Kinetic energy is best described as energy of motion.
		\item [\rule{0.4in}{0.01in} 6.] You do more work playing tag than studying all night for a test.
		\item [\rule{0.4in}{0.01in} 7.] Potential Energy is best described as stored energy.
		\item [\rule{0.4in}{0.01in} 8.] To have potential energy an object must be at rest.
		\item [\rule{0.4in}{0.01in} 9.] Double the force over the same distance will double the work.
		\item [\rule{0.3in}{0.01in} 10.]  It takes less work to push a couch up a ramp than to lift it straight up.
		
		
		
	\end{enumerate}	


\vspace{2in}
\hrule
\vspace{0.5in}
Total Points Earned: \rule{1in}{0.01in}

\vspace{1in}
Corrected by: \rule{3in}{0.01in}



 



\end{document}
