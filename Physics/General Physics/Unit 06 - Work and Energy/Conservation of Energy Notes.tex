\documentclass[letterpaper, 12pt]{article}
\usepackage[top=2cm,bottom=1cm,left=0.75in,right=0.75in,headheight=17pt, % as per the warning by fancyhdr
includehead,includefoot,
heightrounded, % to avoid spurious underfull messages
]{geometry}
\addtolength{\topmargin}{-.25in}
\usepackage{fancyhdr}
\pagestyle{fancy}
\usepackage{graphicx}
\usepackage{lastpage}
\usepackage{gensymb}

\begin{document}
\fancyhead[l]{	\includegraphics[height=1.2cm]{{"../Logos/VersionC"}.png} Name:}
\fancyhead[r]{Date \hspace{ 1in}}
\cfoot{\thepage\ of \pageref{LastPage}}
	

	
\begin{center}{\Large Notes: The Law of Conservation of Energy}
\end{center}
\hangindent=0.5in
\section*{Conservation of Energy}
The word "Conservation" in physics has a very specific meaning. It

The Law of Conservation of Energy states that energy cannot be created, nor can it be destroyed.  That means that whatever energy a system started with it must end with.  If energy flows into or out of the system, that energy must be accounted for as work.  



\section*{Energy} 

\subsection*{Mechanical Energy}


\subsubsection*{Potential Energy}
Potential Energy means stored energy. \textbf{Gravitational potential energy} is stored energy due to a gravitational field. The formula for gravitational potential energy in a uniform gravitational field (like near a planet's surface) is $ U_{g} = m g h $.  \textbf{Elastic Potential Energy} is energy stored by compressing, stretching, or otherwise deforming an object that will naturally return to its shape. Elastic potential energy will be discussed in greater depth in the section on springs. 



\subsubsection*{Kinetic Energy}



 



\end{document}
