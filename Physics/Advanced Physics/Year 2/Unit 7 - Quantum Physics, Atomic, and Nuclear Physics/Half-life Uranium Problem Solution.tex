\documentclass[letterpaper, 12pt]{article}
\usepackage[top=.5in,bottom=.5in,left=.75in,right=.75in,headheight=30pt, % as per the warning by fancyhdr
includehead,includefoot,
heightrounded, % to avoid spurious underfull messages
]{geometry}
\addtolength{\topmargin}{-.25in}
\usepackage{fancyhdr}
\usepackage{cancel}
\usepackage{gensymb}
\usepackage{amsmath} 
\usepackage{xcolor}
\usepackage{tikz}
\usetikzlibrary{angles, quotes}
\usepackage{amssymb}
\pagestyle{fancy}
\usepackage{graphicx}
\usepackage{lastpage}
\usepackage{multicol}
\usepackage{siunitx}
\newcommand{\assnum}{Assignment 7.03}
\newcommand{\assname}{Question 5 \color{red} - Solution\color{black}}

\begin{document}


\fancyfoot[c]{\thepage\ of \pageref{LastPage}}
\fancyfoot[r]{\assnum}	


\begin{center} \assnum{} - \assname{}
\end{center}
\textit{The Uranium-235 and Uranium-238 on earth are assumed to have been formed in a supernova long before the earth was formed.  Currently, only 0.72\% of the uranium in earth is \textsuperscript{235}U while the remaining uranium is \textsuperscript{238}U.  If we assume that the two isotopes were created in equal proportions, how long ago did this supernova take place?}
\\

We know the following information from the question and the table of half-lives:
\\

$t_{1/2-235} = 703.8$ Million Years$ = 7.038 \times 10^8$ Years

$t_{1/2-238} = 4.468$ Billion Years$ = 4.468 \times 10^9$ Years

$A_{235} = 0.72\% = 0.0072 $

$A_{238} = 99.28\% = 0.9928$
\\

The question states that we should assume that the two isotopes were created in equal proportions.  Thus:  
\begin{equation}
A_{0-235} = A_{0-238}
\label{equal}
\end{equation}

The radioactive decay equation states:
\begin{equation}
A = A_0 e^{\frac{-\ln(2)\cdot t}{t_{1/2}}}
\end{equation}

Solving this equation for $A_0$ yields:


\begin{equation}
A_0 = \frac{A}{e^{\frac{-\ln(2)\cdot t}{t_{1/2}}}}
\end{equation}

Plugging this into both sides of equation \ref{equal} gives:

\begin{equation}
\frac{A_{235}}{e^{\frac{-\ln(2)\cdot t}{t_{1/2-235}}}} = \frac{A_{238}}{e^{\frac{-\ln(2)\cdot t}{t_{1/2-238}}}}
\end{equation}

Next, we take the natural log of both sides:
\begin{equation}
\ln\left(\frac{A_{235}}{e^{\frac{-\ln(2)\cdot t}{t_{1/2-235}}}}\right) = \ln\left(\frac{A_{238}}{e^{\frac{-\ln(2)\cdot t}{t_{1/2-238}}}}\right)
\label{naturallog}
\end{equation}

Remembering that: \begin{equation}
\ln\left(\frac{A}{B}\right) = \ln(A) - \ln(B)
\end{equation}

Equation \ref{naturallog} can be written as:

\begin{equation}
\ln(A_{235}) - \ln \left(e^{\frac{-\ln(2)\cdot t}{t_{1/2-235}}}\right) = \ln(A_{238}) - \ln \left(e^{\frac{-\ln(2)\cdot t}{t_{1/2-238}}}\right)
\end{equation}

Since taking the natural log of a number and raising e to a power are inverse operations, this can be simplified to:
\begin{equation}
\ln(A_{235}) - \frac{-\ln(2)\cdot t}{t_{1/2-235}} = \ln(A_{238}) -  \frac{-\ln(2)\cdot t}{t_{1/2-238}}
\end{equation}

Combining negatives gives:
\begin{equation}
\ln(A_{235}) + \frac{\ln(2)\cdot t}{t_{1/2-235}} = \ln(A_{238}) +  \frac{\ln(2)\cdot t}{t_{1/2-238}}
\end{equation}

We then move terms with a $t$ to the left side, and other terms to the right side:
\begin{equation}
\frac{\ln(2)\cdot t}{t_{1/2-235}} - \frac{\ln(2)\cdot t}{t_{1/2-238}} = \ln(A_{238}) - \ln(A_{235})
\end{equation}

Factoring $t$ from the left side of the equation gives:
\begin{equation}
\left(\frac{\ln(2)}{t_{1/2-235}} - \frac{\ln(2)}{t_{1/2-238}}\right) t = \ln(A_{238}) - \ln(A_{235})
\end{equation}

Finally, we divide by the expression in parenthesis:
\begin{equation}
t = \frac{\ln(A_{238}) - \ln(A_{235})}{\left(\frac{\ln(2)}{t_{1/2-235}} - \frac{\ln(2)}{t_{1/2-238}}\right)}
\end{equation}

While this expression is a bit messy and could be made a little nicer, our calculator should be able to handle it.  So, substituting numbers gives:  
\begin{equation}
t = \frac{\ln(99.28\%) - \ln(0.72\%)}{\left(\frac{\ln(2)}{7.038 \times 10^8\si{yrs}} - \frac{\ln(2)}{4.468 \times 10^9\si{yrs}}\right)}
\end{equation}


Evaluating this gives the final answer of:
\begin{equation}
\boxed{t=5.937 \times 10^9 \si{yrs}}
\end{equation}

Since the solar system is estimated to be $4.6 \times 10^9$ years old, this is a reasonable estimate.  While this problem assumes that the two isotopes were created in equal amounts, this may not be a reasonable assumption.  A better understanding of the probability of the creation of each isotope is necessary in order to refine this estimate.  


\end{document}
