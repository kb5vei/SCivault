\documentclass[letterpaper, 11pt]{article}
\usepackage[top=2cm,bottom=1cm,left=0.75in,right=0.75in,headheight=17pt, % as per the warning by fancyhdr
includehead,includefoot,
heightrounded, % to avoid spurious underfull messages
]{geometry}
\addtolength{\topmargin}{-.25in}
\usepackage{fancyhdr}
\pagestyle{fancy}
\usepackage{graphicx}
\usepackage{lastpage}
\usepackage{pgfplots}
\pgfplotsset{width=10cm,compat=1.9}
\usepackage{enumitem,amssymb}

\begin{document}
\fancyhead[l]{	\includegraphics[height=1cm]{"../../Templates/Year C/club".png} Name:}
\fancyhead[r]{Due Date \hspace{ 1in}}
\cfoot{\thepage\ of \pageref{LastPage}}
	

	
\begin{center}Assignment 1.02: Buoyant Force
\end{center}

\begin{enumerate}




	\item A log is floating in a river where it is 50\% submerged.  The log reaches the end of the river and enters the ocean.  When the log enters the ocean, the level of the water, as measured from the bottom of the log will be
	\begin{enumerate}
		\item higher than in the river, because of the greater density of ocean water.
		\item higher than in the river, due to the higher atmospheric pressure at sea-level.
		\item lower than in the river, because ocean water contains a greater amount of dissolved minerals.
		\item lower than in the river, because ocean water has a greater temperature than river water.
		\item The same height as in the river, because the volume of the log does not change.
	\end{enumerate}
 
	\item Scientists believe that mars once had lakes that were very similar to lakes on earth.  The gravity on Mars is approximately 3.7 m/s\textsuperscript{2}.  If a boat that is designed for use in Lake Michigan were placed in a lake on Mars, the boat would - 
	\begin{enumerate}
		\item float higher in the water than on Earth.
		\item float lower in the water than on Earth.
		\item float the same level as on Earth.
		\item sink due to no buoyant force existing on Mars.
		\item It is impossible to tell without knowing the atmospheric pressure of Mars.
	\end{enumerate}
	\item A piece of water-ice (r = 900 kg/m\textsuperscript{3})is dropped into a liquid.  The ice will - 
	\begin{enumerate}
		\item Float, with 9\% of the ice cube submerged.
		\item Float with 90\% of the ice cube submerged.
		\item Be neutrally buoyant.
		\item Sink to the bottom of the container.
		\item It is impossible to tell without knowing the density of the liquid.
	\end{enumerate}

	\item A jellyfish is neutrally buoyant in seawater (ρ = 1020 kg/m\textsuperscript{3}).  The mass of the jellyfish is 10.2 kg.  What is its volume?
	\begin{enumerate}
		\item 1 x 10\textsuperscript{3} m\textsuperscript{3}
		\item 1 x 10\textsuperscript{2} m\textsuperscript{3}
 		\item 1 x 10\textsuperscript{-2} m\textsuperscript{3}
 		\item 1 x 10\textsuperscript{-3} m\textsuperscript{3}
 		e) It cannot be determined without knowing the density of the jellyfish.
	\end{enumerate}
 	\vspace{.15in}
 	\item A floating object displaces 0.6 m\textsuperscript{3} of water. Calculate the buoyant force on the object.
 	\vspace{.95in}
 	
 	\item A ball has a mass of 10 kg and a volume of 0.15 m\textsuperscript{3}.  It is held in place, submerged in water.
	\begin{enumerate}
		\vspace{-.1in}
		\item Calculate the weight of the ball.
		\vspace{.35in}
		\item Calculate the buoyant force on the ball.
		\vspace{.35in}
 		\item The ball is released.  Describe the subsequent motion of the ball in terms of forces acting on the ball.
 		\vspace{.35in}
 		
\end{enumerate}
 
 \item A canoe has a mass of 10 kg.  A 70kg man sits in the canoe.  If the canoe has a volume of 1.1 m\textsuperscript{3}, what is the maximum mass of supplies the man could bring in the canoe before it sinks?
 \vspace{.4in}
 
 \item A beaker of water is placed on a scale.  A cube of metal with volume V=2.7 x 10\textsuperscript{-5} m\textsuperscript{3}  is hung from a string and carefully lowered into the beaker so that it is completely submerged without touching the sides or bottom of the beaker.  How does the reading on the scale change as the metal is submerged? 

	\begin{enumerate} 
		\vspace{-.1in}
	 \item Choose one: \hspace{.25in}
	 $ \square $ Increases \hspace{.25in} $ \square $ Decreases $ \hspace{.25in} \square$ Remains constant
	 
	 \item Justify your answer.
	 \vspace{.4in}
	 \item Perform an experiment to verify your answer.  If the reading does change, determine by how much, and explain this change. 
	 \vspace{.45in}
	 
	  \end{enumerate}
  
	\item The SS Edmund Fitzgerald  sank in Lake Superior on November 10, 1975, with the loss of the entire crew of 29. It is the largest ship to have ever sank in the great lakes.  The ship had a mass of 8686 tons, and carried a load of 26,000 tons of Iron Ore.  (hint: 1 ton = 1000 kg). 
	\begin{enumerate}
		\item   What is the volume of water that had to be displaced by the ship?
		\vspace{.4in}
		\item The ship was roughly a rectangular prism, measuring 222 m x 23 m x 12 meters.  How far would the water line be from the top of the ship?
		\vspace{0.4in}
		\item Why did the Edmond Fitzgerald sink?
		\vspace{0.4in}
		
	\end{enumerate}
	 
	 

\end{enumerate}

 

 
 
 \begin{center}
 	
 	\textit{
 			 The legend lives on from the Chippewa on down, of the big lake they call Gitchigumi.
 		The lake, it's said, never gives up her dead when the skies of November turn gloomy.}
 		
 		--Gordon Lightfoot, \textit {The Wreck of the Edmund Fitzgerald}
 
 	  
 \end{center}

 
	





\end{document}
