\documentclass[letterpaper, 12pt]{article}
\usepackage[top=2cm,bottom=1cm,left=0.75in,right=0.75in,headheight=17pt, % as per the warning by fancyhdr
includehead,includefoot,
heightrounded, % to avoid spurious underfull messages
]{geometry}
\addtolength{\topmargin}{-.25in}
\usepackage{fancyhdr}
\pagestyle{fancy}
\usepackage{graphicx}
\usepackage{lastpage}
\usepackage{gensymb}

\begin{document}
\fancyhead[l]{	\includegraphics[height=1.2cm]{"../../Templates/Year C/club".png} Name:}
\fancyhead[r]{Due Date \hspace{ 1in}}
\cfoot{\thepage\ of \pageref{LastPage}}
	

	
\begin{center}Assignment 5.02: Magnetic Force on Wires
\end{center}

\begin{enumerate}




	\item The magnetic field due to a long wire carrying current is 0.2 T at a distance of 1 cm from the wire. What would the magnetic field be at a distance of 3 cm from the wire?
	\vspace{1in}
	\item A horizontal power line of length 58 m carries a curent of 2200A toward the north.  The earth's magnetic field at this location is 5 x 10\textsuperscript{-5} T directed toward the north, $65 \degree$  below horizontal. 
	\begin{enumerate}
		\item Find the magnitude of the magnetic force on the power line.
		\vspace{.8in}
		\item What is the direction of the force on the power line?
		\vspace{.8in}
	\end{enumerate}

	\item The average nerve impulse has a current of approximately 16 pA.  You are standing in a magnetic field directed to the left.  You want to move your small toe, a distance of 1.75 meters from your brain. What strength of magnetic field would cause a barely-noticable force of 0.1 N to be exerted on your nerve?  
	
	\vspace{.9in}
	
	\item A 10-meter long conductor carrying a current of I=15A is directed along the positive x-axis, perpendicular to a uniform magnetic field.  The force on the conductor is 1.2 N in the negative y direction.  Determine the magnitude and direction of the magnetic field. 
	\vspace{1.5in}
	
	\item Two horizontal wires carry current to the right.  The top wire carries a current of 8A, and the bottom wire carries a current of 2A.  How far below the top wire is the magnetic field zero?
	
	\vspace{1.25in}
	
	
	\item A wire has a mass of 0.1 kg, and a length of 3 meters, and is carrying a current of 25 amps.  It is placed on top of a table, oriented in the X direction, carrying current in the positive X direction.  The coefficient of kinetic friction is 0.2. The wire slides horizontally to the north at a constant speed.  Calculate the magnitude and direction of the magnetic field needed for this to happen.
		
	\vspace{1.25in}
	\item A current-carrying wire is placed in a magnetic field of 2T.  The wire experiences a force of 0.4 N.    The length of the wire is 2 meters.
	
	\begin{enumerate}
		\item Calculate the minimum possible amount of current the wire could be carrying.
		\vspace{1.5in}
		\item Calculate the maximum possible amount of current the wire could be carrying.
	\end{enumerate}
	
\end{enumerate}
 



\end{document}
